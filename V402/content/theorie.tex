\section{Theorie}
\label{sec:Theorie}

Zur Herleitung der Dispersionsgleichung wird die Wechselwirkung zwischen den Ladungen der Materie
und den elektromagnetischen Wellen untersucht.
Die Ladungsträger, wie Ionenrümpfe und Elektronen,
werden durch das elektrische Feld der Welle zu Schwingungen um ihre Ruhelage abgeregt.
Dabei ist zu beachten, dass die Wellenlängen des Lichts nicht kleiner als die des sichtbaren Spektrums sind,
da sonst quantenmechanische Effekte beachtet werden müssten.
Außerdem sollte die Wellenlänge nicht im Resonanzbereich der Materie sein,
da es hier zu ungewollten und nicht vernachlässigbaren Absorptionen kommt.
Die rücktreibende Kraft $\vec{F}_\text{r,h}$, die auf einen Ladungsträger wirkt, ist proportional zur Auslenkung $\vec{x}_\text{h}$
\begin{equation}
  \vec{F}_\text{r,h} = a_\text{h} \vec{x}_\text{h}
\end{equation}
und die Lichtwelle erzeugt die äußere Kraft
\begin{equation}
  \vec{F}_\text{e} = q_\text{h} \vec{E}_0 e^{i \omega t}.
\end{equation}
So ergibt sich mit einem Dämpfungsterm ergänzt die Gleichung eines harmonischen Oszillators:
\begin{equation}
  m_\text{h} \frac{\text{d}^2 \vec{x}_\text{h}}{\text{d} t^2} + f_\text{h} \frac{\text{d} \vec{x}_\text{h}}{\text{d} t} + a_\text{h} \vec{x}_\text{h} = q_\text{h} \vec{E}_0 e^{i \omega t}
  \label{eqn:Oszillator}
\end{equation}
Mit der Polarisation $\vec{P} = \sum N_\text{q} q_\text{h} \vec{x}_\text{h} $ und der Maxwellschenrelation $ n^2 = \epsilon $ wird die Gleichung \eqref{eqn:Oszillator} so umgeschrieben,
dass sich ein Ausdruck für den komplexen Brechungsindex $\tilde n = n(1 + ik)$ aufstellen lässt:
\begin{equation}
  \tilde n = 1 + \sum_h \frac{1}{\omega^2_\text{h} - \omega^2 + i \frac{f_\text{h}}{m_\text{h}}\omega} \frac{N_\text{q} q^2_\text{h}}{m_\text{h} \epsilon_0}
\end{equation}
Dabei ist $q_\text{h}$ die Ladung, $N_\text{q}$ die Anzahl der Ladungsträger pro Volumeneinheit und $\omega^2_\text{h}$ die Resonanzfrequenz.
Der Realteil dieser Funktion beschreibt die Brechung des Lichts und der Imaginärteil die Absorption des Lichts.
Deshalb wird nur der Realteil betrachtet, der sich mit folgender Näherung vereinfachen lässt:
\begin{equation*}
  n^2 k = 0
\end{equation*}
Dies Näherung kann angenommen werden, da die Dispersionskurve nur weit außerhalb der Resonanzbereiche betrachtet wird.
Wird zusätlich noch die Frequenz $\omega$ noch durch die Wellenlänge $\lambda$ ersetzt, ergibt sich folgender Ausdruck:
\begin{equation}
  n^2(\lambda) = 1 + \sum_h \frac{N_\text{q} q^2_\text{h}}{4 \pi^2 c^2 \epsilon_0 m_\text{h}} \frac{\lambda^2 \lambda^2_\text{h}}{\lambda^2 - \lambda^2_\text{h}}
  \label{eqn:n}
\end{equation}
Unter der Annahme, dass die Materie nur eine Absorptionsstelle $\lambda_1$ besitzt, kann die Funktion \eqref{eqn:n} in einer Taylor-Reihe entwickelt werden.

\subsection{Fall 1: \texorpdfstring{$\lambda >> \lambda_1$}{g}}

Für $n^2$ folgt die Gleichung
\begin{equation}
  n^2(\lambda) = A_0 + \frac{A_2}{\lambda^2} + ...
\end{equation}
und für das zugehörige Abweichungsquadrat
\begin{equation}
 s^2_n = \frac{1}{z-2} \sum_{i=1}^z \left(n^2(\lambda_i) - A_0 - \frac{A_2}{\lambda^2_i}\right)^2
 \label{eqn:gl1}
\end{equation}

\subsection{Fall 2: \texorpdfstring{$\lambda << \lambda_1$}{k}}

Für $n^2$ folgt die Gleichung
\begin{equation}
  n^2(\lambda) = A'_0 - \frac{A'_2}{\lambda^2} + ...
\end{equation}
und für das zugehörige Abweichungsquadrat
\begin{equation}
  s^2_n = \frac{1}{z-2} \sum_{i=1}^z \left(n^2(\lambda_i) - A'_0 + \frac{A'_2}{\lambda^2_i}\right)^2
  \label{eqn:gl2}
\end{equation}
