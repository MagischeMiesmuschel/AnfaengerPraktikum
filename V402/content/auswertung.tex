\section{Auswertung}
\subsection{Brechungsindizes}
\label{sec:brech}

\label{sec:Auswertung}
\begin{table}
  \centering
  \caption{$\phi$ Messwerte.}
  \label{tab:phi}
\begin{tabular}{c c c c}
  \toprule
  & $\phi_1 in ^\circ$ & $\phi_2 in ^\circ$ & $(\phi_1 - \phi_2)/2$ \\
  \midrule
  Messung 1  &  205,4  &  325,4  &  60,00 \\
  Messung 2  &  206,6  &  326,9  &  60,15 \\
  Messung 3  &  204,0  &  324,0  &  60,00 \\
  Messung 4  &  202,6  &  323,0  &  60,20 \\
  Messung 5  &  192,1  &  312,0  &  59,95 \\
  Messung 6  &  195,0  &  315,0  &  60,00 \\
  Messung 7  &  198,2  &  318,2  &  60,00 \\
  \bottomrule
\end{tabular}
\end{table}
\FloatBarrier

Mit den Messwerten aus der Tabelle \ref{tab:phi} berechnet sich der Mittelwert von $\phi$ zu

\begin{align*}
  \phi = 60,04 \pm 0,09
\end{align*}


Mit den Messwerten in Tabelle \ref{tab:omega} für die $\Omega$ Winkel, dem Mittelwert der $\phi$ Winkel und den Formeln

\begin{equation}
  \eta = 180^\circ - (\Omega_{links} - \Omega_{rechts})
  \label{eqn:glausw1}
\end{equation}

und

\begin{equation}
  n = \frac{\text{sin}\left(\frac{\eta + \phi}{2}\right)}{\text{sin} \left(\phi\right)}
  \label{eqn:glausw1}
\end{equation}

werden die Brechungsindizes in Tabelle \ref{tab:omega} berechnet.

\begin{table} 
  \centering
  \caption{$\Omega$ Messung.}
  \label{tab:omega}
\begin{tabular}{c c c c}
  \toprule
  Wellenlänge in nm & $\Omega_{links} in ^\circ$ & $\Omega_{rechts} in ^\circ$ & $n$\\ 
  \midrule
  643,85  &  327,8  &  205,5  &  1,711  \\
  578,02  &  327,5  &  205,8  &  1,716  \\
  546,07  &  327,1  &  206,2  &  1,723  \\
  491,61  &  326,7  &  206,5  &  1,723 \\
  467,03  &  326,2  &  206,9  &  1,737 \\
  435,83  &  326,1  &  207,0  &  1,739  \\
  407,78  &  325,3  &  208,1  &  1,755  \\
  \bottomrule
\end{tabular}
\end{table}
\FloatBarrier

Durch auftragen der berechneten $n$ zum Quadrat gegen die passend herausgesuchten Wellenlängen des Quecksilber-Cadmium Spektrums \cite{ref1} aus Tabelle \ref{tab:wellen}

\begin{table}
  \centering
  \caption{Wellenlängen des Spektrums.}
  \label{tab:wellen}
  \begin{tabular}{c c}
    \toprule
    Wellenlänge in nm & Farbe der Linie \\
    \midrule
    643,85 & rot \\
    578,02 & orange \\
    546,07 & gelb-grün \\
    491,61 & grün \\
    467,03 & blau \\
    435,83 & blau-violett \\
    407,78 & violett \\
    \bottomrule 
  \end{tabular}
\end{table}
\FloatBarrier

und durch das Fitten mit den Funktionen

\begin{equation}
  f(x) = A_0 + \frac{A_2}{x^2}
  \label{eqn:glf}
\end{equation}

und 

\begin{equation}
  g(x) = A_0' + A_2' \cdot x^2
  \label{eqn:glg}
\end{equation}

wird der Plot \ref{fig:plot1} erzeugt.

\begin{figure}
  \centering
  \includegraphics{plot1.pdf}
  \caption{Kurvenvergleich der Messwerte.}
  \label{fig:plot1}
\end{figure}
\FloatBarrier

Die Fits liefern folgende Koeffizienten:

\begin{align*}
  f(x): \\
  A_0 &= 2,828947 \pm 0,000169 \\
  A_2 &= 40172,43 \pm 38,03 \\
  \\
  g(x): \\
  A_0' &= 3,143541 \pm 0,000593 \\
  A_2' &= 5,62 \pm 2,04 
\end{align*}

Es ist ein klarer Trend zu erkennen, dass der Fit der Funktion $f(x)$ \eqref{eqn:glf} besser zum Verlauf der Messwerte passt.
Um diesen Verdacht zu überprüfen wird das Abweichungsquadrat beider Funktionen gebildet.
Für $f(x)$ mit Gleichung \eqref{eqn:gl1}, für $g(x)$ mit Gleichung \eqref{eqn:gl2} und mit den jeweiligen Koeffizienten aus den Fits.
Mit den gegebenen Werten berechnen sich die Abweichungsquadrate zu

\begin{align*}
  s_{n} &= 9,15 \cdot 10^{-5} \\
  s_{n'} &= 0,14
\end{align*}

Weil $s_n << s_{n'}$ ist, ist bestätigt, dass der Fit mit $f(x)$ \eqref{eqn:glf} der zutreffende ist.

\subsection{Abelsche Zahl}

Mit der jetzt gegebenen Funktion \eqref{eqn:glf} wird die abelsche Zahl berechnet.
Dazu müssen zuerst die Brechungsindizes der Frauenhoferschen Linien

\begin{align*}
  \lambda_C = 656 \text{nm} \\
  \lambda_D = 589 \text{nm} \\
  \lambda_F = 486 \text{nm}
\end{align*}

mit der Funktion \eqref{eqn:glf} berechnet werden.
Es ergeben sich die Werten
\begin{align*}
  n_C:  1,70947 \pm 0,00006 \\
  n_D:  1,71603 \pm 0,00006 \\
  n_F:  1,73177 \pm 0,00007
\end{align*}

Damit kann nun die Abelsche Zahl nach der Formel 

\begin{equation}
  \nu = \frac{n_D-1}{n_F-n_C}
  \label{eqn:abel}
\end{equation}

berechnet werden und der Fehler berechnet sich mit der Formeln

\begin{align*}
  \Delta \nu &= \sqrt{\left(\frac{\delta}{\delta n_D}\delta \nu \right)^2 \cdot \left(\Delta n_D\right)^2 + \left(\frac{\delta}{\delta n_F}\delta \nu \right)^2 \cdot \left(\Delta n_F\right)^2+\left(\frac{\delta}{\delta n_C}\delta \nu \right)^2 \cdot \left(\Delta n_C\right)^2} \\
  \Delta \nu &= \sqrt{\left(\frac{1}{n_F - n_C}\right)^2\cdot \left(\Delta n_D\right)^2 + \left(\frac{1-n_D}{\left(n_F-n_C\right)^2}\right)^2\cdot \left(\Delta n_F\right)^2 + \left(\frac{n_D - 1}{\left(n_F-n_C\right)^2}\right)^2\cdot \left(\Delta n_F\right)^2}
\end{align*}

Nach dem Einsetzen der Werte ergibt sich für $\nu$

\begin{align*}
  \nu = 32,113 \pm 0,028
\end{align*}

Im Vergleich zum Literaturwert für das optische Glas SF18 von $\nu_{lit} = 29,30$ \cite{ref2} ist nur eine geringe Abweichung zu erkennen.

\subsection{Auflösungsvermögen}

Zur Berechnung des Auflösungsvermögens des Prismenspektralapparates  ist die Basislänge des Prismas notwendig.
Diese ist gegeben als $b = 3$cm.
Es soll die Auflösung für die Wellenlängen $\lambda_C$ und $\lambda_F$ berechnet werden.
Dazu wird die Formel

\begin{equation}
  \frac{\lambda}{\Delta \lambda} = b \cdot \frac{\text{dn}}{\text{d}\lambda}
  \label{eqn:aufl}
\end{equation}

Zuerst wird die Ableitung der Brechungsindize-Funktion benötigt.
Dazu wird die Wurzel aus der Funktion \eqref{eqn:glf} gezogen und abgeleitet.
Daraus folgt

\begin{equation}
  \frac{\text{dn}}{\text{d}\lambda} = - \frac{A_2}{\lambda^3 \sqrt{A_0 + \frac{A_2}{\lambda^2}}}
  \label{eqn:ableitung}
\end{equation}

Der dazugehörige Fehler zum Auflösungsvermögen errechnet sich mit

\begin{align*}
  \Delta \left(\frac{\lambda}{\Delta \lambda}\right) = \sqrt{\left(\frac{A_2}{2 \lambda^3 \left(A_0 + \frac{A_2}{\lambda^2}\right)^{3/2}}\right) \cdot \left(\Delta A_0^2\right) + \left(\frac{1}{\lambda^3 \left(A_0 + \frac{A_2}{\lambda^2}\right)} - \frac{A_2}{2 \lambda^5 \left(A_0 + \frac{A_2}{\lambda^2}\right)^{3/2}}\right)^2 \cdot \left(\Delta A_2^2\right)}
\end{align*}

Durch einsetzen der Wellenlängen $\lambda_C$ und $\lambda_F$ in Gleichung \eqref{eqn:aufl} ergeben sich die Werte

\begin{align*}
  \frac{\lambda_C}{\Delta \lambda} &= 2497.3 \pm 2.3 \\
  \frac{\lambda_F}{\Delta \lambda} &= 6062 \pm 6
\end{align*}

\subsection{Absorptionsstelle}

Aus den Parametern der Dispersionsgleichung \eqref{eqn:glf} berechnet sich die am nächsten gelegene Absorptionsstelle.
Dazu muss die Gleichung \eqref{eqn:n}, aber zunächst wie folgt umgestellt werden, damit für die Absorptionstselle $\lambda_1$ ein reiner Zusammenhang zwischen $A_0$ und $A_2$ besteht.

\begin{align*}
  n^2 &= 1 + \frac{N_1 q_1^2 \lambda_1^2}{4 \pi^2 c^2 \epsilon_0 m_1}\cdot\left(1+\left(\frac{\lambda_1}{\lambda}\right)^2\right) \\
  n^2 &= 1 + \frac{N_1 q_1^2 \lambda_1^2}{4 \pi^2 c^2 \epsilon_0 m_1} + \frac{N_1 q_1^2 \lambda_1^2}{4 \pi^2 c^2 \epsilon_0 m_1} \cdot \left(\frac{\lambda_1}{\lambda}\right)^2 \\
  \Rightarrow A_0 &= 1 + \frac{N_1 q_1^2 \lambda_1^2}{4 \pi^2 c^2 \epsilon_0 m_1} \\
  \Rightarrow A_2 &= \frac{N_1 q_1^2 \lambda_1^2}{4 \pi^2 c^2 \epsilon_0 m_1} \cdot \left(\frac{\lambda_1}{\lambda}\right)^2 \\
\end{align*}
\begin{equation}
  \Rightarrow \lambda_1 = \sqrt{\frac{A_2}{A_0-1}}
  \label{eqn:ausw1}
\end{equation}

Mit dieser Relation und den errechneten Parametern berechnet sich $\lambda_1$ mit der Formel

\begin{align*}
  \Delta \lambda_1 = \sqrt{\left(\frac{\left(\frac{A_2}{A_0-1}\right)^{3/2}}{2A_2}\right)^2 \cdot \left(\Delta A_0\right)^2 + \left(\frac{\sqrt{\frac{A_2}{A_0-1}}}{2 A_2}\right)^2 \cdot \left(\Delta A_2\right)^2}
\end{align*}

für den Fehler, zu

\begin{align*}
  \lambda_1 = 148,21 \pm 0,07 \text{nm}
\end{align*}

Die nächstegelegne berechnete Absorbtionstelle liegt somit im ultravioletten Bereich