\section{Diskussion}
\label{sec:Diskussion}
Die Erholungszeit ist nicht sehr aussagekräftig, weil der Oszillograph ein nicht stehendes Bild erstellen konnte.
Die Wellenberge waren nur Bruchteile von Sekunden zu sehen und in dieser kurzen Zeit nur schwer auszumessen.
Deswegen weicht der Wert unseres Versuchs $T_E=\SI{0.7}{ms}$ und der Wert des parallel laufenden Versuchs $T_E=\SI{240}{\micro\second}$ weit voneinander ab mit 191,7\%. \\
Die Steigung des Plateus hängt von der Nachentladung ab, welche stattfanden.
Jedoch mit 2 $\frac{\%}{100V}$ ist diese noch recht gering.
Die Abweichung der gemessenen Totzeit von 80 $\mu$s und der errechneten von ca. $8 \cdot 10^{-7}$ beträgt 2 Zehnerpotenzen.
Dieser Wert ist sehr hoch und sie befinden sich somit nicht mehr in der selben Größenordnung.
Diese Abweichung kann durch schweres Ablesen am Oszilographen entsanden sein, durch das flackernde Bild.
Im letzten Teil wurde noch die Ladungsmenge pro Teilchen bestimmt.
Diese hatte den zu erwartenden linearen Zusammenhang mit der Spannung, was auch in der Abbildung \ref{fig:plot3} ersichtlich wird.