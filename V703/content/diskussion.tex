\section{Diskussion}
\label{sec:Diskussion}
Die Erholungszeit ist nicht sehr aussagekräftig, weil der Oszillograph ein nicht stehendes Bild erstellen konnte.
Die Wellenberge waren nur Bruchteile von Sekunden zu sehen und in dieser kurzen Zeit nur schwer auszumessen.
Deswegen weicht der Wert unseres Versuchs $T_E=\SI{0.7}{ms}$ und der Wert des parallel laufenden Versuchs $T_E=\SI{240}{\micro\second}$ weit voneinander ab mit 191,7\%. \\
Die starke Steigung des Plateus hängt von der Nachentladung ab, welche stattfanden.
