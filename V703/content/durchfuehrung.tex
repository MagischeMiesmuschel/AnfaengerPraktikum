\section{Durchführung}
\label{sec:Durchführung}

\subsection{Aufnahme der Charakteristik}

Ein $\beta$-Strahler wird vor dem Zählrohr platziert und die Zählrate in Abhängigkeit der äußeren Spannung $U$ gemessen.
Dafür werden die Impulse, die in einer Minute auftreten, bei konstanter Spannung gezählt.
Die Messung wird für Spannungen zwischen 300 und 700 Volt durchgeführt und in 10er Schritten erhöht.

\subsection{Messung der Tot- und Erholungszeit mit einem Oszillographen}

Auf derm Schirm des Oszillographen wird die Zeitablenkung gegen die Anstiegsflanke des Zählrohrimpulses aufgetragen.
Bei bekannter Ablenkgeschwindigkeit des Kathodenstrahls können Tot- und Erholungszeit abgeschätzt werden.

\subsection{Zwei-Quellen-Methode}

Mit Hilfe von zwei radioaktiven Präparaten kann die Totzeit bestimmt werden.
Die Zählrate der einzelnen Präparate und die Zählrate beider Präparate zusammen wird gemessen, aus denen die Totzeit bestimmt werden kann.


\subsection{freigesetzte Ladungsmenge}

Der mittlere Zählrohrstrom wird gemessen, um daraus die pro Teilchen vom Zählrohr freigesetzten Ladungsmenge zu bestimmen.
