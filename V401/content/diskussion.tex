\section{Diskussion}
\label{sec:Diskussion}

Die berechnete Wellenlänge des Lasers weicht um $\SI{7.09}{\percent}$ gegenüber der tatsächlichen Wellenlänge von $\SI{635}{\nano\meter}$ nach oben ab.
Die tatsächliche Wellenlänge liegt erst im 5. Fehlerintervalls der berechneten Wellenlänge,
der berechnete Wert liegt jedoch noch im Spektrum des roten Lichts.
Diese große Abweichung scheint für die sonst hohe Genauigkeit des Interferometers überraschend zu sein.
Dass die aus den Messwerten bestimmte Wellenlänge deutlich größer ist, kann auf das veraltete Zählwerk zurückgeführt werden.
Das Zählwerk registriert nicht jedes Interferenzmaxima, deshalb fällt die Wellenlänge größer aus als sie sollte.
Der Literaturwert des Brechungsindex in Luft beträgt auf Meeresniveau durchschnittlich 1,00029. \cite{brech}
Damit weicht der berechnete Wert nur um $\SI{2.60e-3}{\percent}$ im Vergleich zum Literaturwert ab.
Die relative Abweichung fällt so gering aus, da die Änderungen des Brechungsindex in Luft erst in der 4. Nachkommastelle bemerkbar wird.
Zusätzlich liegt der Literaturwert auch im zweiten Fehlerintervalls des berechneten Werts.
Dieses Ergebnis mit geringen Unsicherheiten spricht für die hohe Präzession des Interferometers.
