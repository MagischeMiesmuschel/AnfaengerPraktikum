\section{Diskussion}
\label{sec:Diskussion}

Die berechnete Wellenlänge des Lasers weicht um $\SI{7.09}{\percent}$ gegenüber der tatsächlichen Wellenlänge von $\SI{635}{\nano\meter}$ nach oben ab.
Dass die aus den Messwerten bestimmte Wellenlänge größer ist, kann auf das veraltete Zählwerk zurückgeführt werden.
Das Zählwerk registriert nicht jedes Interferenzmaxima, deshalb fällt die Wellenlänge größer aus als sie sollte.
Der Literaturwert des Brechungsindex in Luft beträgt auf Meeresniveau durchschnittlich 1,00029. \cite{brech}
Damit weicht der berechnete Wert nur um $\SI{2.60e-3}{\percent}$ ab.
Dieses Ergebnis spricht für die hohe Präzession des Interferometers.
