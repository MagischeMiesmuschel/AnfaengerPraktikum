\section{Auswertung}
\label{sec:Auswertung}

Aus Gleichung \eqref{eqn:gl1} kann die Wellenlänge des Lasers berechnet werden, diese muss jedoch aufgrund der Hebelübersetzung modifiziert werden.
Bevor die Gleichung nach $\lambda$ umgestellt wird, wird die rechte Seite mit dem Faktor $u = 5,046$ multipliziert.
Die gemessenen Zählraten und die Start- und Endpunkte, die zusammen $\increment d = |d_1-d_2|$ ergeben, können nun eingesetzt werden, um $\lambda$ zu bestimmen.
In Tabelle \ref{tab:welle} sind die Messwerte und die dazugehörige Wellenlänge $\lambda$ aufgelistet.
\begin{table}
  \centering
  \caption{Messwerte zur Berechnung der Wellenlänge des Lasers.}
  \label{tab:welle}
  \begin{tabular}{c c c c}
    \toprule
    $z$ & $d_1$ /\si{\milli\meter} & $d_2$ /\si{\milli\meter} & $\lambda$ /\si{\nano\meter} \\
    \midrule
    3004 & 14,0 &   8,9 & 672,90 \\
    3001 &  9,0 & 14,16 & 681,50 \\
    3005 & 14,0 &  8,93 & 668,72 \\
    3008 &  9,0 &  3,91 & 670,69 \\
    3000 &  4,0 &   9,3 & 700,22 \\
    3150 &  9,0 &  14,5 & 692,05 \\
    3003 & 14,5 &  9,26 & 691,61 \\
    3002 &  9,0 & 14,18 & 683,91 \\
    3000 & 14,0 &  8,95 & 667,20 \\
    3001 &  9,0 & 14,13 & 677,54 \\
    \bottomrule
  \end{tabular}
\end{table}
\FloatBarrier
Aus den Ergebnissen lässt sich folgender Mittelwert ableiten:
\begin{equation*}
  \bar \lambda = \SI{680.63(1066)e-9}{\meter}
\end{equation*}
Zur Bestimmung des Brechungsindex in Luft werden folgende Angaben benötigt:
\begin{align*}
  \text{Normaldruck} \, p_0 &= \SI{1.0132}{\bar} \\
  \text{Nomraltemperatur} \, T_0 &= \SI{273.15}{\kelvin} \\
  \text{Umgebungstemperatur} \, T &= \SI{296.15}{\kelvin} \\
  \text{Breite der Messzelle} \, b &= \SI{50e-3}{\meter} \\
\end{align*}
Aus den gemessenen Zählraten $z$ und den entsprechenden Drücken $p$ lässt sich nun mit Gleichung \eqref{eqn:gl2} der Brechungsindex $n$ ermitteln.
In Tabelle \ref{tab:n} sind die Messwerte und der dazu berechnete Brechungsindex dargestellt.
\begin{table}
  \centering
  \caption{Messwerte zur Berechnung des Brechungsindex.}
  \label{tab:n}
  \begin{tabular}{c c c}
    \toprule
    $z$ & $p$ [\si{\bar}] & $n$ \\
    \midrule
    33 & 0,8 & 1,0002878 \\
    33 & 0,8 & 1,0002878 \\
    33 & 0,8 & 1,0002878 \\
    33 & 0,8 & 1,0002878 \\
    33 & 0,8 & 1,0002878 \\
    33 & 0,8 & 1,0002878 \\
    32 & 0,8 & 1,0002790 \\
    33 & 0,8 & 1,0002878 \\
    33 & 0,8 & 1,0002878 \\
    35 & 0,8 & 1,0003052 \\
    \bottomrule
  \end{tabular}
\end{table}
\FloatBarrier
Für den Mittelwert ergibt sich folgender Wert:
\begin{equation*}
  \bar n = \num{1.0002886(61)}
\end{equation*}
