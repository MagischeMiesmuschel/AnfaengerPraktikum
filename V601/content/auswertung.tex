\section{Auswertung}
\label{sec:Auswertung}

\subsection{mittlere freie Weglänge}
\label{sec:A1}
Die vom Sättigungsdampfdruck abhängige mittlere freie Weglänge $\bar{w}$ hat maßgeblich auf die Zusammenstöße zwischen Elektronen Hg-Atome Einfluss.
Das Verhältnis von mittlerer freier Weglänge $\bar{w}$ und dem Abstand Kathode-Beschleunigungselektrode $a$ muss zwischen 1000 und 4000 liegen.
In diesem Bereich ist der Sättigungsdampfdruck groß genug, um ausreichend Zusammenstöße für die Beobachtung des Franck-Hertz-Effekts hervorzurufen.
Der Sättigungsdampfdruck ist jedoch noch zu gering für ein vermehrtes Auftreten von elastischen Stößen,
die dafür sorgen, dass die Elektronen ihre Richtung ändern und die Auffängerelektrode nciht mehr erreichen.
Für die mittlere freie Weglänge und den Sättigungsdampfdruck $p_\text{sät}$ gelten folgende Beziehungen.
\begin{align}
  \bar{w} \, [\text{cm}] &= \frac{0,0029}{p_\text{sät}} \\
  p_\text{sät}(T) \, [\text{mbar}] &= 5,5 \cdot 10^7 e^{-6876/T}
\end{align}
In der anschließenden Tabelle \ref{tab:weg} werden die mittlere freie Weglänge $\bar{w}$ und der Abstand Kathode-Beschleunigungselektrode $a$ nun verglichen.
Für die hier verwendete Röhre beträgt $a = \SI{1}{\centi\meter}$.
\begin{table}
  \centering
  \caption{Verhältnis von mittlerer freier Weglänge $\bar{w}$ und der Abstand Kathode-Beschleunigungselektrode $a$.}
  \label{tab:weg}
  \begin{tabular}{c c c}
    \toprule
    Temperatur $T$ in K & Weglänge $\bar{w}$ in m & Verhältnis $a/\bar{w}$\\
    \midrule
    314,15 & 1 &  1\\
    423,15 & 1 &  1\\
    448,15 & 1 &  1\\
    378,15 & 1 &  1\\
    \bottomrule
  \end{tabular}
\end{table}

\subsection{Energieverteilung der Elektronen}
\label{sec:A2}

Die integrale Energieverteilung kann durch die Überführung der differntiellen in die integrale Energieverteilung bestimmt werden.
Zunächst muss also die differntielle Energieverteilung aus den ersten beiden Graphen in Kapitel \ref{sec:Anhang} ermittelt werden, die die Bremsspannung $U_\text{A}$ in Abhängigkeit des Auffängerstroms $I_\text{A}$ abbilden.
Der Auffängerstrom ist proportional zur Anzahl der Elektronen und die Bremsspannung proportional zur Energie der Elektronen.
Die differntielle Energieverteilung kann also über in die Graphen eingezeichnete Steigungsdreicke berechnet werden.
Dafür ist allerdings eine Skalierung der x-Achse der Graphen notwendig, die im Folgenden durchgeführt wird (siehe Tabelle \ref{tab:skal1}).
\begin{table}
  \centering
  \caption{Skalierung der x-Achse der ersten beiden Graphen aus Kapitel \ref{sec:Anhang}.}
  \label{tab:skal1}
  \begin{tabular}{c c c}
    \toprule
    Bremsspannung & Graph 1 (\SI{314,15}{\kelvin}) & Graph 2 (\SI{423,15}{\kelvin})\\
    in V & Anzahl mm-Kästchen & Anzahl mm-Kästchen \\
    \midrule
    0-2  & 48 &  49\\
    2-4  & 47 &  49\\
    4-6  & 52 &  51\\
    6-8  & 53 &  50\\
    \bottomrule
    Skalierung $S$ & $S_1 = \num{4.010(12)e-2}$ V/mm & $S_2 = \num{4.021(4)e-2}$ V/mm \\
  \end{tabular}
\end{table}
Die aus den Steigungsdreicken berechneten Steigungen sind in Tabelle \ref{tab:steig} aufgeführt und werden in Abbildung \ref{fig:plot1} und \ref{fig:plot2} gegen die Bremsspannung aufgetragen.
Die Ordinaten sind dabei proportional zur Anzahl, die Abszissen proportional zur Energie der Elektronen.
\begin{table}
  \centering
  \caption{ Bremsspannung und Steigung der Graphen.}
  \label{tab:steig}
  \begin{tabular}{c c c | c c c}
    \toprule
    \multicolumn{3}{c}{Graph 1 (\SI{314,15}{\kelvin})} & \multicolumn{3}{c}{Graph 2 (\SI{423,15}{\kelvin})} \\
    Stelle $x$ & Bremsspannung $x \cdot S_1$ & Steigung & Stelle $x$ & Bremsspannung $x \cdot S_2$ & Steigung  \\
    in mm & in V & $\increment y / \increment x$ & in mm & in V & $\increment y / \increment x$  \\
    \midrule
     6  & 0,240  & 0,1 & 3,5   & 0,140 & 3,0   \\
    16  & 0,641  & 0,1 & 8.5   & 0,341 & 2,6 \\
    26  & 1,042  & 0,1 & 13,5  & 0,542 & 2,6 \\
    36  & 1,443  & 0,1 & 18,5  & 0,743 & 4,4 \\
    46  & 1,844  & 0,1 & 23,5  & 0,944 & 2,6 \\
    56  & 2,245  & 0,2 & 28,5  & 1,146 & 1,0 \\
    66  & 2,646  & 0,1 & 33,5  & 1,347 & 1,4 \\
    76  & 3,047  & 0,2 & 38,5  & 1,548 & 0,8 \\
    86  & 3,448  & 0,3 & 43,5  & 1,749 & 0,2 \\
    96  & 3,850  & 0,2 & 48,5  & 1,950 & 0,4 \\
    106 & 4,251  & 0,3 & 53,5  & 2,151 & 0,08 \\
    116 & 4,652  & 0,2 & 58,5  & 2,352 & 0,08 \\
    126 & 5,053  & 0,2 & 63,5  & 2,553 & 0,08 \\
    136 & 5,454  & 0,3 & 68,5  & 2,754 & 0,08 \\
    146 & 5,855  & 0,4 & 73,5  & 2,955 & 0,08 \\
    156 & 6,256  & 0,4 & 78,5  & 3,156 & 0,4 \\
    166 & 6,657  & 0,5 & 83,5  & 3,357 & 0,4 \\
    176 & 7,058  & 0,6 & 88,5  & 3,558 & 0,4 \\
    186 & 7,459  & 0,7 & 93,5  & 3.759 & 0,4 \\
    193 & 7,760  & 1,0 & 98,5  & 3.960 & 0,4 \\
    198 & 7,960  & 1,0 & 103,5 & 4,161 & 0,6 \\
    203 & 8,161  & 1,2 & 108,5 & 4,363 & 0,6 \\
    208 & 8,361  & 1,2 & 113,5 & 4,564 & 0,6 \\
    213 & 8,562  & 1,6 & 118,5 & 4,765 & 0,6 \\
    218 & 8,762  & 2,0 & 123,5 & 4,966 & 0,6 \\
    223 & 8,963  & 2,8 & 128,5 & 5,167 & 0,6 \\
    228 & 9,163  & 4,4 & 133,5 & 5,368 & 0,4 \\
    233 & 9,364  & 3,6 & 138,5 & 5,569 & 0,6 \\
    238 & 9,564  & 0,8 & 143,5 & 5,770 & 0,6 \\
    243 & 9,765  & 0,0 & 148,5 & 5,971 & 0,4 \\
        &        &     & 153,5 & 6,172 & 0,4 \\
        &        &     & 158,5 & 6,373 & 0,4 \\
        &        &     & 163,5 & 6,574 & 0,2 \\
        &        &     & 168,5 & 6,775 & 0,2 \\
        &        &     & 173,5 & 6,976 & 0,0 \\
        &        &     & 178,5 & 7,177 & 0,0 \\

    \bottomrule
  \end{tabular}
\end{table}
\begin{figure}
  \centering
  \includegraphics{plot1.pdf}
  \caption{Differntielle Energieverteilung bei 314,15 K.}
  \label{fig:plot1}
\end{figure}
\begin{figure}
  \centering
  \includegraphics{plot2.pdf}
  \caption{Differntielle Energieverteilung bei 423,15 K.}
  \label{fig:plot2}
\end{figure}
\FloatBarrier

\subsection{Franck-Hertz-Kurve}
\label{sec:A3}

Die aufgenommene Franck-Hertz-Kurve ist als dritter Graph in Kapitel \ref{sec:Anhang} zu sehen.
Genau wie zuvor in Kapitel \ref{sec:A2} muss die x-Achse skaliert werden (siehe Tabelle \ref{tab:skal2}).
\begin{table}
  \centering
  \caption{Skalierung der x-Achse des dritten Graphen aus Kapitel \ref{sec:Anhang}.}
  \label{tab:skal2}
  \begin{tabular}{c c}
    \toprule
    Bremsspannung & Graph 3 (\SI{448,15}{\kelvin}) \\
    in V & Anzahl mm-Kästchen  \\
    \midrule
    0-10   &  33\\
    10-20  &  32\\
    20-30  &  32\\
    30-40  &  34\\
    40-50  &  33\\
    \bottomrule
    Skalierung $S$ & $S_3 = \num{3.050(35)e-1}$ V/mm  \\
  \end{tabular}
\end{table}
Aus Abstände der Maxima lässt die erste Anregungsenergie $U_1$ berechnen.
Die Abstände können aus dem dritten Graphen abgelesen werden und sind in Tabble \ref{tab:an} mit den entsprechenden Energien aufgeführt.
\begin{table}
  \centering
  \caption{Abstände der Maxima der Franck-Hertz-Kurve und Anregungsenergie.}
  \label{tab:an}
  \begin{tabular}{c c c}
    \toprule
    Ordnungszahl & Abstand der Maxima $M_{k+1} - M_k$ & Anregungsenergie $U_1$\\
    & Anzahl mm-Kästchen & in eV \\
    \midrule
    1 & 16 & 4,880 \\
    2 & 16 & 4,880 \\
    3 & 16 & 4,880 \\
    4 & 16 & 4,880 \\
    5 & 18 & 5,490 \\
    6 & 19 & 5,795 \\
    7 & 17 & 5,185 \\
    \bottomrule
    Mittelwert $\bar{U_1}$ & & \num{5.142(140)} \\
  \end{tabular}
\end{table}
Aus der mittleren Anregungsenergie kann die dazugehörige Wellenlänge berechnet werden.
\begin{equation*}
  \lambda = \frac{h \cdot c}{U_1} =
\end{equation*}
Aus der Abweichung zwischen erstem Maxima und der Anregungsenergie lässt sich das Kontaktpotential berechnen.
\begin{equation*}
  K_2 = M_1 - U_1 =
\end{equation*}
Der Energieverlust durch die elastischen Stöße führt nur zu einem Abflachen der Kurve.
Da jedoch für die Auswertung der Anregungsenergie nur die Abstände der Maxima von Interesse sind, spielt dieser Effekt keine Rolle.

\subsection{Ionisierungsenergie}
\label{sec:A4}

Die Ionisierungsenergie wird mit Hilfe des letzten Graphen aus Kapitel \ref{sec:Anhang} bestimmt.
Dazu wird eine Asymptote in den Graphen eingezeichnet und aus deren Schnittpunkt $x_0 = \SI{15.45}{\volt}$ mit der x-Achse und den zuvor berechneten Kontaktpotentialen die Ionisierungsenergie berechnet.
\begin{align*}
E_\text{ion,1} &= (x_0 - K_1) \text{e}_0 = \SI{4}{\volt}  \\
E_\text{ion,2} &= (x_0 - K_1) \text{e}_0 = \SI{4}{\volt}  \\
\end{align*}
