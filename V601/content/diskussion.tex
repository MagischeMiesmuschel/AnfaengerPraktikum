\section{Diskussion}
\label{sec:Diskussion}

Die zwei berechneten Ionisierungsenergien weichen deutlich vom Literaturwert $E_\text{Lit} = \SI{10.438}{\eV}$ \cite{Ion} ab:
\begin{align*}
  \frac{E_\text{ion,1} - E_\text{Lit}}{E_\text{Lit}} &= \SI{30.99}{\percent} \\
  \frac{E_\text{ion,2} - E_\text{Lit}}{E_\text{Lit}} &= \SI{32.99}{\percent}
\end{align*}
Die Abweichung der Ionisierungsenergien untereinander liegt jedoch bei \SI{1.51}{\percent}.
Die Kontaktpotentiale sind also wahrscheinlich relativ genau bestimmt worden.
Die deutlich größere Fehlerquelle ist der Schnittpunkt der Asymptote, der genutzt wird, um die Ionisierungsenergie zu berechnen.
Denn die Asymptote wird nicht berechnet, sondern nach Maß eingezeichnet und der Schnittpunkt abgelesen.
Zwischen Literaturwert $U_\text{Lit} = \SI{4.9}{\eV}$ \cite{Queck} und berechnetem Mittelwert der Anregungsenergie liegt nur eine Abweichung von:
\begin{equation*}
  \frac{\bar{U_1} - U_\text{Lit}}{U_\text{Lit}} = \SI{4.94}{\percent}
\end{equation*}
Eine Ursache für die Abweichung stellt die Genauigkeit der Werte dar, da diese nur aus den Graphen aus \ref{sec:Anhang} abgelesen werden.
Außerdem zeichnet der xy-Schreiber die Messwerte auch nur begrenzt genau auf.
