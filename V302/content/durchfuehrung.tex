\section{Durchführung}
\label{sec:Durchführung}

Die Messreihen mit den ersten vier Brückenschaltungen werden mit einem Wechselstrom bei einer Frequenz von 1 kHz durchgeführt.
Die Brückenspannung wird über ein digitales Oszilloskop abgelesen.

\subsection{Wheatstonesche Brücke}

Die Schaltung wird wie in Abbildung \ref{fig:Schaltung2} aufgebaut.
Die Verhältnisse der Widerstände können über ein Potentiometer variiert werden, das das Verhältnis von $R_3/R_4$ reguliert.
Es werden zwei unbekannte Widerstände jeweils dreimal für unterschiedliche $R_2$ vermessen.

\subsection{Kapazitätsmessbrücke}

Der Aufbau der Brücke wird nach Abbildung \ref{fig:Schaltung3} durchgeführt.
Als Stellglieder in diesem Versuchsteil werden $R_2$ und $R_3/R_4$ verwendet, die abwechselnd justiert werden bis die Brückenspannung verschwindet.
Für jedes unbekannte Bauteil werden drei Messreihen mit unterschiedlichem $C_2$ angesetzt.
Es soll die Kapazität zweier Kondensatoren und die Impedanz einer RC-Kombination gemessen werden.

\subsection{Induktivitätsmessbrücke}

Die Induktivität und der Widerstand einer Spule wird mit der in Abbildung \ref{fig:Schaltung4} dargestellten Schaltung vermessen.
Das Verschwinden der Brückenspannung wird über $R_2$ und $R_3/R_4$ eingestellt.

\subsection{Maxwell-Brücke}

Die selbe Spule wie zuvor wird nun mit der Maxwell-Brücke (siehe Abb. \ref{fig:Schaltung5}) ausgemessen.
Varriert werden dieses Mal die Widerstände $R_3$ und $R_4$ einzeln eingestellt.
Die Messung wird für zwei weitere $R_2$ wiederholt.

\subsection{Wien-Robinson-Brücke}

Die Brückenschaltung wird, wie in Abbildung \ref{fig:Schaltung6} zu sehen, aufgebaut.
Die Frequenzabhängigkeit der Brückenspannung soll untersucht werden.
Dafür werden die Brückenspannung und  die Speisespannung im Frequenzbereich von 20-30000 Hz betrachtet.
Besonders im Frequenzbereich, in dem die Brückenspannung minimal wird, sollen viele Messdaten notiert werden.
Das Ergebnis soll mit der Theorie verglichen werden.
