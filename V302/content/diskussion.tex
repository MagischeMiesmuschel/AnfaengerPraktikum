\section{Diskussion}
\label{sec:Diskussion}
Bei der Bestimmung des Widerstandes mit der Wheatonschen Brücke in Abschnitt \ref{sec:Weed} ist die Unsicherheit aus dem Gauß höher als aus dem Mittelwert.
Dies lässt auf eine genaue Messmethode schließen, welche lediglich durch noch genauere Bauteile optimiert werden kann.
Das Gegenteil ist bei der Kapazitätsmessbrücke in Abschnitt \ref{sec:Kapazitaet} der Fall.
Hier dominiert bei den Kapazitäten die Unsicherheit aus dem Mittelwert, dadurch ist zu überlegen, ob die Messmethode zu optimieren ist.
Beim Vergleich der Ergebnisse von Abschnitt \ref{sec:Induktiv} und \ref{sec:Maxwell} fällt auf, dass die Induktivität beider Messmethoden gut übereinstimmt.
Die Wetre liegen innerhalb eines "Unsicherheits-Intervalls" auseinander.
Aber die Widerstände liegen mit 24,5 $\Omega$ deutlich auseinander.
Vom Messwert aus Abschnitt \ref{sec:Maxwell} sind es mit der dominierenden Gauß-Unsicherheit
\begin{align}
    \Delta R_{x,18} &= 373,5 \Omega - 348,5 \Omega = 24,5 \Omega
    I_R &= \frac{24,5 \Omega}{9 \Omega} =2,72
\end{align}
2,72 Unsicherheits-Intervalle Unterschied.
Mit der Gaußunsicherheit des Widerstandes aus Abschnitt \ref{sec:Induktiv}
\begin{align}
    I_R = \frac{24,5 \Omega}{2 \Omega} = 12,25
\end{align}
sind es sogar 12,25 Unsicherheits-Intervalle.
Durch weitere Messungen in Abschnitt \ref{sec:Induktiv} wäre der Widerstand genauer zu bestimmen gewesen.
Auch wurde bei der Induktivitätsmessbrücke vorrausgesetzt, dass die eingebaute Spule $L_2$ eine perfekte sei ohne Innenwiderstand.
Bei der Maxwell-Brücke wurde für $C_4$ ein verlustfreier Kondensator vorrausgesetzt.
Dies ist allerdings nicht in gänze realisierbar.
Die in der Auswertung vermutete Aussage zur Güte der Sinusspannung wurde durch den Klirrfaktor bestätigt.