\section{Auswertung}
\label{sec:Auswertung}

Durch die relative Unsicherheit der Bauteile wird die Unsicherheit für jeden Mittelwert einmal nach Gauß und einmal nach der Gleichung \eqref{eqn:fehler2}.
Für die Werte $R_2$, $C_2$ und $L_2$ wird eine Unsicherheit von \pm 0.2\% angenommen.

\subsection{Wheatsonesche Brücke}
\label{sec:Weed}
Die relative Unsicherheit der Bauteile beträgt \pm 0.5\% für das Verhältnis $R_3/R_4$.
Mit den notierten Werten aus Tabelle 1 im Anhang und mit Gleichung \eqref{eqn:gl4} lässt sich der Widerstand $R_x$ berechnen zu:
\begin{align*}
  R_{x,10} &= (240 \pm 0,7 \textrm{(Gauß)} \pm 0,39 \textrm{(Mittelwert)}) \upOmega \\
  R_{x,13} &= (321 \pm 1 \textrm{(Gauß)} \pm 0,43 \textrm{(Mittelwert)}) \upOmega
\end{align*}

\subsection{Kapazitätsmessbrücke}
\label{sec:Kapazitaet}
Der variable Widerstand $R_2$ hat hier eine Unsicherheit von \pm 3\% und das Verhältnis von $R_3/R_4$ eine von \pm 0.5\%.
Die Berechnung von $C_x$ und $R_x$ erfolgt nach Gleichung \eqref{eqn:gl5} und \eqref{eqn:gl6}, mit den notierten Werten in Tabelle 2 im Anhang.
\begin{align*}
  C_{x,3} &= (407,6 \pm 1,3 \textrm{(Gauß)} \pm 9,9 \textrm{(Mittelwert)}) \textrm{nF} \\
  R_{x,3} &= 0\\
  \\
  C_{x,1} &= (643,3 \pm 2 \textrm{(Gauß)} \pm 1,5 \textrm{(Mittelwert)}) \textrm{nF} \\
  R_{x,1} &= 0\\
  \\
  C_{x,8} &= (287,2 \pm 0,9 \textrm{(Gauß)} \pm 6,7 \textrm{(Mittelwert)}) \textrm{nF} \\
  R_{x,8} &= (577 \pm 10 \textrm{(Gauß)} \pm 1,41 \textrm{(Mittelwert)}) \upOmega
\end{align*}

\subsection{Induktivitätsmessbrücke}
\label{sec:Induktiv}
Der variable Widerstand $R_2$ hat die gleiche Unsicherheit wie im vorherigen Kapitel \ref{sec:Kapazitaet}, sowie das Verhältnis $R_3/R_4$.
Es war nur möglich eine Messung durchzuführen, durch beschädigte Spulen.
Lediglich eine war nutzbar und an der Stelle $L_2$ einzubauen.
Deshalb wird hier kein Mittelwertsfehler angegeben bei der Berechnung von $L_x$ und $R_x$ nach Gleichung \eqref{eqn:gl7} und \eqref{eqn:gl8} mit den Werten aus Tabelle 3 im Anhang.
\begin{align*}
  L_{x,18} &= (49,58 \pm 0,27) \textrm{mH} \\
  R_{x,18} &= (373,5 \pm 2) \upOmega
\end{align*}

\subsection{Maxwell-Brücke}
\label{sec:Maxwell}
Die relativen Unsichereiten der beiden Widerstände $R_3$ und $R_4$ betragen 3\% und für $C_4$ gibt es eine von 0.2\%.
Die Werte für $L_x$ und $R_x$ berechnen sich nach den Gleichungen \eqref{eqn:gl9} und \eqref{eqn:gl10} mit den Werten aus der Tabelle 4 im Anhang.
\begin{align*}
  L_{x,18} &= (49,7 \pm 0,9 \textrm{(Gauß)} \pm 0,11 \textrm{(Mittelwert)}) \textrm{mH} \\
  R_{x,18} &= (348 \pm 9 \textrm{(Gauß)} \pm 1,2 \textrm{(Mittelwert)}) \upOmega
\end{align*}

\subsection{Wien-Robinson-Brücke}
\label{sec:Wien}
Das Verhältnis der effektiven Brückenspannung $U_{Br,eff}$ zur Speisespannung $U_S$ wird gegen $\upOmega = \frac{\nu}{\nu_0}$ in Abb. \ref{fig:plot} abgetragen.
$U_{Br,eff}$ berechnet sich nach:
\begin{equation}
  U_{Br,eff} = \frac{U_{Br}}{2 \sqrt{2}}
  \label{eqn:eff}
\end{equation}
In die selbe Abbildung wird eine Theoriekurve, die nach Gleichung \eqref{eqn:gl12} berechnet wird, eingezeichnet.
Die Frequenz bei der die Brückenspannung erschwinden sollte ergibt sich zu:
\begin{align*}
  \omega_0 &= \frac{1}{RC} = \frac{1}{332 \upOmega \cdot 993 \cdot 10^{-9} \textrm{F}} = 3033,28 \textrm{Hz} \\
  \nu_0 &= \frac{\omega_0}{2 \pi} = 482,76 \textrm{Hz}
\end{align*}
\newpage
\begin{figure}
  \centering
  \includegraphics{build/plot1.pdf}
  \caption{Vergleich von Messdaten und Theoriekurve.}
  \label{fig:plot}
\end{figure}
\noindent
Der Vergleich der Kurven zeigt einen ähnlichen Verlauf mit guter Übereinstimmung bei $\omega_0$.
Doch die generelle Abweichung der Messwerte oberhalb der Kurve, lässt auf einen größeren Klirrfaktor schließen, welcher im folgenden berechnet wird.
Für diese Berechnung wird zunächt genähert, dass die Summe der Oberwellen nur von dem Term der zweiten Oberwelle bestimmt wird.
Es wird mit der Effektivspannung nach Gleichung \ref{eqn:14} der Wert für $U_2$ mit $\upOmega = 2$ berechnet.
\begin{align*}
  U_2 &= \frac{0.014 V}{\sqrt{\frac{(2^2-1)^2}{9 \cdot [(1-2^2)^2+9 \cdot 2^2]}}} \\
  &= 0,63 V
\end{align*}
$U_1$ sind 3.88 V, von $U_S$ bei $\nu_0$.
Dementsprechend ist der Klirrfaktor:
\begin{align*}
  k = \frac{U_2}{U_1} = 0,1624
\end{align*}
