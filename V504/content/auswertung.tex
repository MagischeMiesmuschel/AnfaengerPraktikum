\section{Auswertung}
\label{sec:Auswertung}

\subsection{Kennlinien und Sättigungsstrom}
\label{sec:str}

Für 5 verschiedene Heizleistungen werden Anodenspannung und -strom gemessen, um entsprechende Kennlinien zu erstellen.
Die erfassten Daten sind in folgender Tabelle dargestellt und können für eine Berechnung des Sättigungsstroms genutzt werden.

Die Berechnung erfolgt über eine Ausgleichsrechnung mit folgender Funktion:
\begin{equation}
  I(U) = I_\text{S} - A e^{B \cdot U}
\end{equation}
Für die Kennlinie mit der maximalen Heizleistung kann kein Sättigungsstrom bestimmt werden, da die Kennlinie noch nicht im Sättigungsstromgebiet verläuft.
Dies ist in Abbildung \ref{fig:plot5} eindeutig an der fehlenden Rechtskrümmung der Werte zu erkennen.
\begin{figure}
  \centering
  \includegraphics{plot1.pdf}
  \caption{Kennlinie 1.}
  \label{fig:plot1}
\end{figure}
\begin{figure}
  \centering
  \includegraphics{plot2.pdf}
  \caption{Kennlinie 2.}
  \label{fig:plot2}
\end{figure}
\begin{figure}
  \centering
  \includegraphics{plot3.pdf}
  \caption{Kennlinie 3.}
  \label{fig:plot3}
\end{figure}
\begin{figure}
  \centering
  \includegraphics{plot4.pdf}
  \caption{Kennlinie 4.}
  \label{fig:plot4}
\end{figure}
\begin{figure}
  \centering
  \includegraphics{plot5.pdf}
  \caption{Kennlinie 5.}
  \label{fig:plot5}
\end{figure}
\noindent
Die berechneten Sättigungsströme sind in folgender Tabelle aufgeführt.

\subsection{Langmuir-Schottkysches Raumladungsgesetz}
\label{sec:expo}

Kennlinie 5 (siehe Abb. \ref{fig:plot5}) verläuft, wie an der Linkskrümmung zu erkennen ist, im Raumladungsgebiet.
Mit einer Ausgleichsrechnung kann diese Kennlinie also betrachtet werden, um den Exponenten des Langmuir-Schottkysches Raumladungsgesetzes zu überprüfen.
Die Regression wird mit folgender Gleichung durchgeführt
\begin{equation}
  I(U) = A \cdot U^B
\end{equation}
und liefert für den Exponenten $B = \num{1.358(15)}$.

\subsection{Anlaufstrom und Kathodentemperatur}
\label{sec:anlauf}

Die gemessenen Anlaufströme in Abhängigkeit der Gegenspannung sind in folgender Tabelle abgebildet.

Die gemessenen Spannungen entsprechen jedoch nicht der Spannung zwischen Anode und Kathode, da der Innenwiderstand des Nanoamperemeters (1 M$\Omega$) einen Spannungsabfall hervorruft.
Eine wie folgend aussehende Korrektur ist notwendig:
\begin{equation*}
  U_\text{korr} = U - R_\text{Innen} I_\text{A}
\end{equation*}
Anschließend kann über eine Ausgleichsrechnung (siehe Abb. \ref{fig:plot6}) die Kathodentemperatur bestimmt werden.
Dafür wird diese Gleichung benutzt:
\begin{equation}
  I(U) = A e^{B \cdot U}
\end{equation}
Für die Kathodentemperatur berechnet sich so $T = \SI{2790(60)}{\kelvin}$.
\begin{figure}
  \centering
  \includegraphics{plot6.pdf}
  \caption{Anlaufstrom.}
  \label{fig:plot6}
\end{figure}

\subsection{Leistungsbilanz und Kathodentemperatur}
\label{sec:temp}

Aus den eingestellten Heizspannungen und -strömen lässt sich mit die Kathodentemperatur berechnen.
Alle Konstanten werden der Anleitung entnommen.
Die Temperaturen sind in der anschließenden Tabelle dargestellt.

\subsection{Austrittsarbeit}

Die Austrittsarbeit kann nun über die Richardson-Gleichung bestimmt werden.
Dafür werden die Temperaturen aus Kapitel \ref{sec:temp} und die Sättigungsströme aus Kapitel \ref{sec:str} verwendet.
Folgende Werte für die Austrittsarbeit werden berechnet.
