\section{Auswertung}
\label{sec:Auswertung}

\subsection{Kennlinien und Sättigungsstrom}
\label{sec:str}

Für 5 verschiedene Heizleistungen werden Anodenspannung und -strom gemessen, um entsprechende Kennlinien zu erstellen.
Die erfassten Daten sind in folgender Tabelle dargestellt und können für eine Berechnung des Sättigungsstroms genutzt werden.
\begin{table}
\centering
\caption{Anodenspannungen und -ströme.}
\label{tab:messwerte}
\begin{tabular}{c c c c c c}
\toprule
$U_\text{A}$ in V & $I_1$ in mA & $I_2$ in mA & $I_3$ in mA & $I_4$ in mA & $I_5$ in mA \\
\midrule
10  & 0.020 & 0.026 & 0.028 & 0.029 & 0.050 \\
20  & 0.044 & 0.060 & 0.069 & 0.073 & 0.077 \\
30  & 0.072 & 0.100 & 0.119 & 0.128 & 0.141 \\
40  & 0.097 & 0.139 & 0.177 & 0.195 & 0.216 \\
50  & 0.111 & 0.180 & 0.232 & 0.266 & 0.307 \\
60  & 0.120 & 0.206 & 0.289 & 0.351 & 0.411 \\
70  & 0.130 & 0.234 & 0.355 & 0.436 & 0.519 \\
80  & 0.134 & 0.255 & 0.412 & 0.526 & 0.631 \\
90  & 0.136 & 0.267 & 0.460 & 0.609 & 0.754 \\
100 & 0.136 & 0.274 & 0.499 & 0.686 & 0.874 \\
110 & 0.140 & 0.279 & 0.534 & 0.764 & 1.006 \\
120 & 0.142 & 0.283 & 0.560 & 0.832 & 1.137 \\
130 & 0.143 & 0.286 & 0.579 & 0.901 & 1.260 \\
140 & 0.144 & 0.289 & 0.594 & 0.962 & 1.381 \\
150 & 0.145 & 0.291 & 0.606 & 1.021 & 1.522 \\
160 & 0.146 & 0.293 & 0.618 & 1.071 & 1.652 \\
170 & 0.147 & 0.295 & 0.622 & 1.117 & 1.784 \\
180 & 0.148 & 0.297 & 0.626 & 1.156 & 1.919 \\
190 & 0.149 & 0.299 & 0.632 & 1.187 & 2.050 \\
200 & 0.150 & 0.301 & 0.637 & 1.212 & 2.190 \\
\bottomrule
\end{tabular}
\end{table}
Die Berechnung erfolgt über eine Ausgleichsrechnung mit folgender Funktion:
\begin{equation}
  I(U) = I_\text{S} - A e^{B \cdot U}
\end{equation}
Für die Kennlinie mit der maximalen Heizleistung kann kein Sättigungsstrom bestimmt werden, da die Kennlinie noch nicht im Sättigungsstromgebiet verläuft.
Dies ist in Abbildung \ref{fig:plot5} eindeutig an der fehlenden Rechtskrümmung der Werte zu erkennen.
Die berechneten Sättigungsströme sind in folgender Tabelle aufgeführt.
\begin{table}
\centering
\caption{Sättigungsströme.}
\label{tab:a}
\begin{tabular}{c c}
\toprule
Kennlinie & $I_\text{S}$ \\
\midrule
1 & \SI{0.148(1)}{\milli\ampere} \\
2 & \SI{0.308(4)}{\milli\ampere} \\
3 & \SI{0.750(31)}{\milli\ampere} \\
4 & \SI{2.773(445)}{\milli\ampere} \\
\bottomrule
\end{tabular}
\end{table}
\begin{figure}
  \centering
  \includegraphics{plot1.pdf}
  \caption{Kennlinie 1.}
  \label{fig:plot1}
\end{figure}
\begin{figure}
  \centering
  \includegraphics{plot2.pdf}
  \caption{Kennlinie 2.}
  \label{fig:plot2}
\end{figure}
\begin{figure}
  \centering
  \includegraphics{plot3.pdf}
  \caption{Kennlinie 3.}
  \label{fig:plot3}
\end{figure}
\begin{figure}
  \centering
  \includegraphics{plot4.pdf}
  \caption{Kennlinie 4.}
  \label{fig:plot4}
\end{figure}
\begin{figure}
  \centering
  \includegraphics{plot5.pdf}
  \caption{Kennlinie 5.}
  \label{fig:plot5}
\end{figure}

\FloatBarrier
\subsection{Langmuir-Schottkysches Raumladungsgesetz}
\label{sec:expo}

Kennlinie 5 (siehe Abb. \ref{fig:plot5}) verläuft, wie an der Linkskrümmung zu erkennen ist, im Raumladungsgebiet.
Mit einer Ausgleichsrechnung kann diese Kennlinie also betrachtet werden, um den Exponenten des Langmuir-Schottkysches Raumladungsgesetzes zu überprüfen.
Die Regression wird mit folgender Gleichung durchgeführt
\begin{equation}
  I(U) = A \cdot U^B
\end{equation}
und liefert für den Exponenten $B = \num{1.358(15)}$.

\FloatBarrier
\subsection{Anlaufstrom und Kathodentemperatur}
\label{sec:anlauf}

Die gemessenen Anlaufströme in Abhängigkeit der Gegenspannung sind in folgender Tabelle abgebildet.
\begin{table}
\centering
\caption{Anlaufströme und Gegenspannungen.}
\label{tab:c}
\begin{tabular}{c c}
\toprule
Gegenspannung $U$ in V & Anlaufströme $I_\text{A}$ in nA \\
\midrule
0.05 & 100 \\
0.1  & 80  \\
0.15 & 58  \\
0.2  & 47  \\
0.25 & 37  \\
0.3  & 28  \\
0.35 & 20  \\
0.4  & 15  \\
0.45 & 14  \\
0.5  & 9.5 \\
0.55 & 7   \\
0.6  & 6   \\
0.65 & 4.5 \\
0.7  & 3   \\
0.75 & 2.1 \\
0.8  & 1.5 \\
0.85 & 1   \\
0.9  & 0.8 \\
0.95 & 0.5 \\
1    & 0.4 \\
\bottomrule
\end{tabular}
\end{table}
Die gemessenen Spannungen entsprechen jedoch nicht der Spannung zwischen Anode und Kathode, da der Innenwiderstand des Nanoamperemeters (1 M$\Omega$) einen Spannungsabfall hervorruft.
Eine wie folgend aussehende Korrektur ist notwendig:
\begin{equation*}
  U_\text{korr} = U - R_\text{Innen} I_\text{A}
\end{equation*}
Anschließend kann über eine Ausgleichsrechnung (siehe Abb. \ref{fig:plot6}) die Kathodentemperatur bestimmt werden.
Dafür wird diese Gleichung benutzt:
\begin{equation}
  I(U) = A e^{B \cdot U}
\end{equation}
Für die Kathodentemperatur berechnet sich so $T = \SI{2790(60)}{\kelvin}$.
\begin{figure}
  \centering
  \includegraphics{plot6.pdf}
  \caption{Anlaufstrom.}
  \label{fig:plot6}
\end{figure}

\FloatBarrier
\subsection{Leistungsbilanz und Kathodentemperatur}
\label{sec:temp}

Aus den eingestellten Heizspannungen und -strömen lässt sich mit die Kathodentemperatur berechnen.
Alle Konstanten werden der Anleitung entnommen.
Die Temperaturen sind in der anschließenden Tabelle dargestellt.
\begin{table}
\centering
\caption{Kathodentemperatur.}
\label{tab:d}
\begin{tabular}{c c c c}
\toprule
Kennlinie & Heizspannung $U_\text{H}$ & Heizstrom $I_\text{H}$ & Temperatur $T$ \\
\midrule
1 & \SI{4.3}{\volt} & \SI{2}{\ampere}   & \SI{1964}{\kelvin} \\
2 & \SI{4.6}{\volt} & \SI{2.1}{\ampere} & \SI{2029}{\kelvin} \\
3 & \SI{5}{\volt}   & \SI{2.2}{\ampere} & \SI{2103}{\kelvin} \\
4 & \SI{5.3}{\volt} & \SI{2.3}{\ampere} & \SI{2163}{\kelvin} \\
5 & \SI{6.2}{\volt} & \SI{2.5}{\ampere} & \SI{2308}{\kelvin} \\
\bottomrule
\end{tabular}
\end{table}

\FloatBarrier
\subsection{Austrittsarbeit}

Die Austrittsarbeit kann nun über die Richardson-Gleichung bestimmt werden.
Dafür werden die Temperaturen aus Kapitel \ref{sec:temp} und die Sättigungsströme aus Kapitel \ref{sec:str} verwendet.
Folgende Werte für die Austrittsarbeit werden berechnet.
\begin{table}
\centering
\caption{Austrittsarbeit.}
\label{tab:e}
\begin{tabular}{c c c c}
\toprule
Kennlinie & Temperatur $T$ & Sättigungsstrom $I_\text{S}$ & Austrittsarbeit $e_0 \phi$ \\
\midrule
1 & \SI{1964}{\kelvin} & \SI{0.148(1)}{\milli\ampere}   & \SI{4.6769(12)}{\eV} \\
2 & \SI{2029}{\kelvin} & \SI{0.308(4)}{\milli\ampere}   & \SI{4.7152(25)}{\eV} \\
3 & \SI{2103}{\kelvin} & \SI{0.750(31)}{\milli\ampere}  & \SI{4.740(8)}{\eV} \\
4 & \SI{2163}{\kelvin} & \SI{2.773(445)}{\milli\ampere} & \SI{4.642(30)}{\eV} \\
\bottomrule
\end{tabular}
\end{table}
