\section{Durchführung}
\label{sec:Durchführung}
Durch Varriation der Heizleistung wird eine Kennlinienschar einer Hochvakuumdiode mit mindestens 5 Kennlinien erstellt.
Darauf wird der jeweilige Sättigungsstrom $\text{I}_s$ abgelesen.
Danach wird für die maximal mögliche Heizleistung versucht den ungefähren Gültigkeitsbereich des Langmuir-Schottkyschen Raumladungsgesetzes zu finden.
Dort wird aus den gemessenen Wertepaaren der Exponent der Strom-Spannungs-Beziehung berechnet.
Als nächstes wird das Anlaufstromgebiet der Diode für die maximal mögliche Heizleistung untersucht und daraus die Kathodentemperatur bestimmt.
Aus einer Leistungsbilanz des Heizstromkreises wird dann die Kathodentemperatur, bei den am Anfang verwendeten Heizleistungen, geschätzt.
Zuletzt wird aus den verschiedenen T- und zugehörigen $\text{I}_s$-Werten die Austrittsarbeit errechnet.