\section{Diskussion}
\label{sec:Diskussion}

Der berechnete Exponent des Langmuir-Schottkysches-Gesetzes (siehe \ref{sec:expo}) und der theoretische Wert von 1.5 weichen um $\SI{9.5}{\percent}$ ab.
Die Kathodentemperatur bei maximaler Heizleistung variiert je nach Methode.
Der Literaturwert der Austrittsarbeit des Wolframs ($4,5 \, eV$) weicht nur minimal von den berechneten Werten ab.
Zwischen Mittelwert und Literaturwert liegen nur $\SI{4.3}{\percent}$.
Ungenauigkeiten können auf die nicht berücksichtigten Innenwiderstände der Leitungen und Netzgeräte oder die beschränkte Genauigkeit der Messgeräte zurückgeführt werden.
Aber auch Vereinfachungen in der Theorie wie z.B. die Annahme, dass Anode und Kathode unendlich ausgedehnte ebene Oberflächen darstellen, spielen eine Rolle.
Bei der Messung des Anlaufstroms ist außerdem noch die hohe Empfindlichkeit des Nanoamperemeters anzumerken.
Selbst bei kleinsten Änderungen des Versuchaufbaus änderte sich der gemessene Strom.
Dieser Umstand könnte die abweichenden Kathodentemperaturen erklären.
