\section{Theorie}
\label{sec:Theorie}

In einem Körper, welcher sich nicht im Temperaturgleichgewicht befindet, kommt es zu einem Wärmetransport entlang des Temperaturgefälles.
Es wird sich hierbei auf die Wärmeleitung beschrenkt..
In festen Körpern geschieht der Wärmetransport durch Phononen und über frei bewegliche Elektronen.
Der Gitterbeitrag ist in Metallen vernachlässigbar. \\
\noindent
Zu Grunde liegt ein Stab der Länge L, mit dem Querschnitt A, aus einem Material der Dichte \rho und mit spezifischer Wärme c.
Durch die Querschnittsfläche A fließt in der Zeit dt die Wärmemenge
\begin{equation}
  dQ = -\kappa \text{A} \frac{\delta T}{\delta x} dt
  \label{eqn:eqn1}
\end{equation}
wenn das eine Ende wärmer ist als das andere.
\kappa ist die Wärmeleitfähigkeit.
Der Wärmestrom fließt nach immer in Richtung abnehmender Temperatur, dies wird durch das Minuszeichen in der Gleichung gezeigt.
Für die Wärmestromdichte $j_{\text{w}}$ gilt entsprechend
\begin{equation}
  j_{\text{w}} = -\kappa \frac{\delta T}{\delta x}
  \label{eqn:eqn2}
\end{equation}
Durch die Kontinuitätsgleichung wird hieraus die (eindimensionale) Wärmeleitungsgleichung
\begin{equation}
  \frac{\delta T}{\delta t} = \frac{\kappa}{\rho c} \frac{\delta^2 T}{\delta x^2}
  \label{eqn:eqn3}
\end{equation}
abgeleitet, welche die räumliche und zeitliche Entwicklung der Temperaturverteilung beschreibt.
Die Größe $\sigma T = \frac{\kappa}{\rho c}$ wird als Temperaturleitfähigkeit bezeichnet.
Durch sie wird die Schnelligkeit des Ausgleiches der Temperaturunterschiede angegeben.\\
\noindent
Durch abwechselndes erwärmen und abkühlen eines sehr langen Stabes mit der Periode T, pflanzt sich eine räumliche und zeitliche Temperaturwelle der Form
\begin{equation}
  T(x,t) = T_{\text{max}}e^{\sqrt{\frac{\omega \rho c}{2 \kappa}}x} \text{cos} \left ( \omega t - \sqrt{\frac{\omega \rho c}{2 \kappa}}x \right)
  \label{eqn:eqn4}
\end{equation}
durch den Stab fort. 
Die Phasengeschwindigkeit der Welle lautet:
\begin{equation}
  v = \frac{\omega}{k} = \frac{\omega}{sqrt{\frac{\omega \rho c}{2 \kappa}}} = \sqrt{\frac{2 \kappa \omega}{\rho c}}
  \label{eqn:eqn5}
\end{equation}
Aus dem Amplitudeverhältnis $A_{\text{nah}}$ und $A_{\text{fern}}$ der Welle an zwei Stellen $x_{\text{nah}}$ und $x_{\text{fern}}$.
Wird nun genutzt, dass $\omega = \frac{2 \pi}{T*}$ und $\Phi = \frac{2 \pi \delta t}{T*}$ (mit der Phase \Phi und Periodendauer T*),
dann ergibt sich für die Wärmeleitfähigkeit:
\begin{equation}
  \kappa = \frac{\rho c (\Delta x)^2}{2 \Delta t \text{ln}(A_{\text{nah}} / A_{\text{fern}})}
  \label{eqn:eqn6}
\end{equation}
$\Delta x$ ist der Abstand der beiden Messstellen und $\Delta t$ die Phasendifferenz der Temperaturwellen zwischen den beiden Messstellen.