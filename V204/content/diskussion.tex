\section{Diskussion}
\label{sec:Diskussion}

Die Abweichung der berechneten Wärmeleitfähigkeit und der theoretischen liegt innerhalb des ersten Fehlerintervalls für Edelstahl und für Aluminium und im zweiten für Messing.
\begin{align*}
  \text{Messing: }   \kappa_\text{Literatur} - \kappa = \SI{26}{\watt\per\meter\per\kelvin} &< 2\increment \kappa = \SI{40}{\watt\per\meter\per\kelvin} \\
  \text{Aluminium: } \kappa_\text{Literatur} - \kappa = \SI{37}{\watt\per\meter\per\kelvin} &< \increment \kappa = \SI{60}{\watt\per\meter\per\kelvin} \\
  \text{Edelstahl: } \kappa_\text{Literatur} - \kappa = \SI{0.9}{\watt\per\meter\per\kelvin} &< \increment \kappa = \SI{2.4}{\watt\per\meter\per\kelvin} \\
\end{align*}
Jedoch fallen die Fehler relativ betrachtet auffällig groß auf.
Messing hat einen reltiven Fehler von $\pm \SI{21.28}{\percent}$ und Aluminium von $\pm \SI{30.00}{\percent}$.
Nur der relative Fehler von Edelstahl mit $\pm \SI{17.02}{\percent}$ ist geringfügig kleiner.
Mögliche Ursachen könnten im Versuchsaufbau liegen.
Die Metallproben können in der Praxis nicht optimal isoliert werden, es treten immer Wärmeverluste an die Umgebung auf.
Der Generator, der das Peltierelement betreibt liefert keine perfekt konstante Spannung, es treten leichte Schwankungen auf.
Außerdem sind die Thermoelement nur begrenzt dazu in der Lage, die Temperatur exakt zu messen.
Besonders auffällig ist dies bei der Temperaturmessung des Edelstahls.
Da die Temperatur bei dieser Materialprobe nur langsam steigen im Vergleich zu den Messschwankungen, können keine eindeutigen Extrempunkte bestimmt werden.
Systematische Fehler können auch nicht ausgeschlossen werden, wie zum Beispiel das manuelle Wechseln zwischen Aufwärmen und Abkühlen.
