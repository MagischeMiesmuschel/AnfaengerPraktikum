\section{Diskussion}
\label{sec:Diskussion}

Alle drei Messungen der Leerlaufspannung der Monozelle liegen nah beieinander.
Die Differnz zwischen direkter Messung und Messung ohne Gegenspannng liegen im ersten Fehlerintervall.
Die direkte Messung und die Messung mit Gegenspannng weichen auch nur um 3 Fehlerintervalle ab.
Auch für die Berechnung des Innenwiderstands der Monozelle zeichnet sich ein akzetables Bild ab.
Bezogen auf die erste Messung beträgt die relative Abweichung zwischen Messung mit und ohne Gegenspannng etwa $\SI{8.9}{\percent}$.
Abweichungen können auftreten durch die begrenzte Mess- und Ablesegenauigkeit der Messgeräte.
Auch die Widerstände der Kabel  und der Messgeräte beeinflussen die Genauigkeit, wie in Kapitel \ref{sec:Auswertung} für das Voltmeter gezeigt wurde.
Die dadurch auftretenden Ungenauigkeiten liegen jedoch im Toleranzbereich, was an der Qualität der Ergebnisse zu sehen ist.

\noindent
Für die Sinus- und Rechteckspannung gibt es keine Refernzwerte, die Ergebnisse scheinen sich aber in einer realistischen Größenordnung zu befinden.
Bei diesen Messungen ist noch ein weiterer Faktor aufgefallen, der für eine größere Ungenauigkeit sorgt.
Der RC-Generator liefert keine konstante Spannung, was zur Folge hat, dass der Zeiger des Voltmeters schwankte und nur beschränkt genau abgelesen werden konnte.

\noindent
Zwischen Theoriekurve und Messwerten der Leistung in Abhängigkeit vom Belastungswiderstand treten keine signifikanten Unterschiede auf.
Es kann also vermutet werden das kein systematischer Fehler vorliegt und Abweichungen auf statistische Schwankungen zurückzuführen sind.
