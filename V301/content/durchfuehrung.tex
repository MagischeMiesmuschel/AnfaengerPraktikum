\section{Durchführung}
\label{sec:Durchführung}
Zuerst wird die Leerlaufspannung der Monozelle mit einem geeigneten Spannungsmesser ermittelt und der Eigenwiderstand $R_v$ notiert.
Für die erste Messreihe wird in Abb. \ref{fig:abb2} die Spannung $U_k$ in Abhängigkeit von I aufgetragen, für einen Belatungswiderstand von 0 bis 50 \Omega .
Danach wird an die Monozelle eine Gegenspannung, die ca. 2V größer als $U_0$ ist angelegt (siehe Abb. \ref{fig:abb3}). 
Es wird ein Strom in umgekehrter Richtung fließen und die Klemmenspannung beträgt
\begin{align}
  U_k = U_0 + \symbf{I} R_i
\end{align}
Es wird wieder $U_k$ in Abhängigkeit von $\symbf{I}$ gemessen.
Zuletzt wird die erste Messreihe wiederholt, nur nicht mit einer Monozelle, sondern dem Sinus- und Rechteckausgang eines RC-Generators.
Für die 1V-Rechteckspannung wird ein Variationsbereich von $R_a$: 20 - 250 \Omega benutzt. \\

Für die 1V-Sinusspannung wird ein Variationsbereich von $R_a$: 0.1 - 5 k\Omega benutzt.