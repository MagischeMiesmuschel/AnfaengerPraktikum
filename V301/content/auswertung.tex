\section{Auswertung}
\label{sec:Auswertung}

Das für die Messungen verwendete Voltmeter hat einen Innenwiderstand von $R_v \geq 10 \, \text{M}\Omega $.
Für die ertsen beiden Messreihen wird eine Monozelle verwendet, deren Leerlaufspannung $U_0$ direkt gemessen 1,6 Volt beträgt.

Zur Berechnung des Innenwiderstands $R_i$ und der Leerlaufspannung $U_0$ werden $U_k$ und $\symbf{I}$ gegeneinader aufgetragen.
Anschließend wird mit diesen Werten und Gleichung \eqref{eqn:eq3} eine lineare Ausgleichsrechnung durchgeführt.
Der y-Achsenabschnitt $b$ und die Steigung $a$ der resultierenden Geraden (siehe Abb. \ref{fig:plot1}) sind die Leerlaufspannung $U_0$ und der Innenwiderstand $R_i$ der Monozelle:
\begin{align*}
  -a = R_i = (16,9 \pm 0,2) \Omega \\
  b = U_0 = (1,60 \pm 0,01) \text{V}
\end{align*}
\begin{figure}
  \centering
  \includegraphics{plot1.pdf}
  \caption{Klemmenspannung und Stromstärke der Monozelle ohne Gegenspannung.}
  \label{fig:plot1}
\end{figure}
Im zweiten Versuchsteil mit der Monozelle wird eine Gegenspannung von 3,6 Volt angelegt.
Zur Bestimmung des Innenwiderstands und der Leerlaufspannung werden die gleichen Schritte wie zuvor durchgeführt.
Lediglich die Vorzeichen der Stromstärke tauschen sich für die Gleichung der Klemmenspannung (siehe Gl. \eqref{eqn:eq5}), da die Stromrichtung sich umkehrt.
Für die Ausgleichsgerade (siehe Abb. \ref{fig:plot1}) ergeben sich dann folgende Werte:
\begin{align*}
  a = R_i = (15,4 \pm 0,6) \Omega \\
  b = U_0 = (1,68 \pm 0,03) \text{V}
\end{align*}
\begin{figure}
  \centering
  \includegraphics{plot2.pdf}
  \caption{Klemmenspannung und Stromstärke der Monozelle mit Gegenspannung.}
  \label{fig:plot2}
\end{figure}
Innenwiderstand und Leerlaufspannung der Rechteckspannungsquelle bzw. der Sinusspannungsquelle lassen sich analog über eine lineare Regression mit Gleichung \eqref{eqn:eq3} bestimmen.
Es werden die in Abbildung \ref{fig:plot3} und \ref{fig:plot4} dargestellten Ausgleichsgeraden mit folgenden Werten berechnet:
\begin{align*}
  \intertext{Rechteckspannung:}
  -a = R_i = (55,3 \pm 0,4) \Omega \\
  b = U_0 = (0,632 \pm 0,001) \text{V}
  \intertext{Sinusspannung:}
  -a = R_i = (665 \pm 5) \Omega \\
  b = U_0 = (1,080 \pm 0,002) \text{V}
\end{align*}
\begin{figure}
  \centering
  \includegraphics{plot3.pdf}
  \caption{Klemmenspannung und Stromstärke der Rechteckspannung.}
  \label{fig:plot3}
\end{figure}
\begin{figure}
  \centering
  \includegraphics{plot4.pdf}
  \caption{Klemmenspannung und Stromstärke der Sinusspannung.}
  \label{fig:plot4}
\end{figure}

Da der Eingangswiderstand $R_v$ des Voltmeters nicht unendlich ist, tritt bei der direkten Messung der Leerlaufspannung ein systematischer Fehler auf.
Dieser soll im Folgenden aus den zu Anfang notierten $U_0 \approx U_k = 1,6 \text{V}$ und $R_v = 10 \, \text{M}\Omega $ und dem berechneten $R_i = 16,9 \, \Omega$ ermittelt werden.
Mit $\symbf{I} = U_k / R_a$ und $R_a = R_v$ lässt sich Gleichung \eqref{eqn:eq3} schreiben als:
\begin{equation*}
  U_0 = U_k \frac{R_i}{R_v} + U_k
\end{equation*}
Mit dieser Gleichung lässt sich nun die systematische Abweichung von $U_0$ ohne Berücksichtigung und mit Berücksichtigung des Eingangswiderstands berechnen.
\begin{align*}
  \increment U_0 = U_0 - U_k = 2,704 \cdot 10^{-6} \text{V} \\
  \frac{\increment U_0}{U_k} = 1,69 \cdot 10^{-6}
\end{align*}
Diese relative Abweichung ist so klein, dass der systematische Fehler vernachlässigt werden kann.
Ein weiterer systematischer Fehler tritt auf, wenn das Voltmeter hinter Punkt H in Abbildung \ref{fig:abb2} angeschlossen wird.
In diesem Fall wird am Voltmeter nicht nur die Spannug über der Spannungsquelle gemessen, sondern auch über dem Amperemeter, das auch einen Widerstand besitzt.

Als Letztes soll die Leistung, die über dem Belastungswiderstand anfällt und von diesem abhängt, betrachtet werden.
Dieser Versuchsteil bezieht sich nur auf die Monozelle ohne Gegenspannng.
Dazu wird die aus den Messwerten $U_k$ und $\symbf{I}$ mit einer Theoriekurve (siehe Abb. \ref{fig:plot5}), die auf dem zuvor berechneten Innenwiderstand und der Leerlaufspannung der Monozelle beruht.
Für die Leistung gilt:
\begin{align*}
  N = U_k \cdot \symbf{I}
  \intertext{und durch Umformen von Gleichung \eqref{eqn:eq2} nach $\symbf{I}$:}
  N = \symbf{I}^2 \cdot R_a = \frac{U_0^2}{(R_a + R_i)^2} R_a
\end{align*}
Wie in der Theorie \ref{sec:Theorie} beschrieben nimmt die Leistung für ein bestimmtes $N$ einen Maximalwert an (Leistungsanpassung).
\begin{figure}
  \centering
  \includegraphics{plot5.pdf}
  \caption{Am Belastungswiderstand umgesetzte Leistung.}
  \label{fig:plot5}
\end{figure}
