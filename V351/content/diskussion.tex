\section{Diskussion}
\label{sec:Diskussion}

Die Genauigkeit der Fourier-Analyse ist extrem hoch, alle Abweichungen liegen unter einem Prozent.
Der Unterschied von Theorie- und Praxiswert für alle drei Spanungen liegt innerhalb des ersten Intervalls der Unsicherheit.
\begin{align*}
  \intertext{Rechteckspannung}
  \increment m_{\text{Messung}} = 0,0141 > 0,0023 = \lvert m_{\text{Theorie}} - m_{\text{Messung}} \rvert
  \intertext{Sägezahnspannung}
  \increment m_{\text{Messung}} = 0,0286 > 0,0085 = \lvert m_{\text{Theorie}} - m_{\text{Messung}} \rvert
  \intertext{Dreieckspannung}
  \increment m_{\text{Messung}} = 0,325 > 0,0126 = \lvert m_{\text{Theorie}} - m_{\text{Messung}} \rvert
\end{align*}
Das kann auf den den simplen Aufbau und die einfache Messung zurückgeführt werden.
Der Funktionsgenerator ist direkt an das Oszilloskop angeschlossen, welches auch die Fourier-Transformation übernimmt.
Der Verusch besteht also aus wenigen Bauteilen, es müssen kaum Einstellungen vorgenommen werden und Messwerte werden nicht analog bestimmt.
Diese Faktoren tragen alle zu einer geringen Fehleranflälligkeit und Messunsicherheit bei.

Die Ergebnisse der Fourier-Synthese sind relativ gut, wenn bedacht wird, dass sich das beobachtete Signal nur aus 9 Oberwellen zusammensetzt.
Deutliche Abweichungen treten nur an den Unstetigkeitsstellen auf (zu sehen in Abblidung \ref{fig:abb4} und \ref{fig:abb5}), was als Gibbsches Phänomen bezeichnet wird.
Unter der Bedingung, dass die Fourier-Reihe gleichmäßig konvergent sein muss, kann diese eine Sprungstelle nicht approximieren.
Ein weiterer Grund für Ungenauigkeiten ist die Einstellung der Phase und der Amplitude, die nur begrenzt per Hand vorgenommen werden kann.
Am besten kann die Dreieckspannung moduliert werden, wie man auch an Abbildung \ref{fig:abb6} erkennen.
Als Erklärung dafür kann das Fehlen von Unstetigkeiten genannt werden.
Das Abfallen der Amplituden mit $\frac{1}{n^2}$ ist ein weiterer Faktor, da Oberwellen einer höheren Nummer nur einen geringen Beitrag lesiten.
