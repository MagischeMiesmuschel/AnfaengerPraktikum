\section{Auswertung}
\label{sec:Auswertung}

\subsection{Empfindlichkeit der Braunschen Röhre}
\label{sec:Empfindlichkeit}
Unter der Empfindlichkeit der Braunschen Röhre wird das Verhältnis zwischen der Ablenkung $D$ und der angelegten Spannung $d_U$ verstanden.
Mit den Werten in Tabelle \ref{tab:messwerte1} wird $D$ gegen $U_\text{d}$ in Abbildung \ref{fig:plot1} aufgetragen und durch lineare Regression die Steigung der Ausgleichsgeraden durch die Messpunkte ermittelt.

\FloatBarrier
\begin{table}
  \centering
  \caption{Messwerte Ablenkung}
  \label{tab:messwerte1}
  \begin{tabular}{c c c c c c}
  \toprule
  $U_\text{d} 1$ in V & $U_\text{d} 2$ in V & $U_\text{d} 3$ in V & $U_\text{d} 4$ in V & $U_\text{d} 5$ in V & $D$ in cm \\
  \midrule
  20 & 23 & 25 & 28.6 & 30.3 & 4.8 \\
  16.4 & 19 & 20.7 & 23.4 & 25.4 & 4.2 \\
  12.7 & 15 & 16.3 & 18.4 & 19.8 & 3.6 \\
  9.1 & 10.6 & 11.8 & 13.3 & 14.3 & 3 \\
  5.5 & 6.5 & 7.3 & 8.2 & 8.9 & 2.4 \\
  2.1 & 2.2 & 2.5 & 3.3 & 3.6 & 1.8 \\
  -1.9 & -2.1 & -2.3 & -2.2 & -2.0 & 1.2 \\
  -5.6 & -6.7 & -7.0 & -7.5 & -7.7 & 0.6 \\
  -9.5 & -10.8 & -11.8 & -12.5 & -13 & 0 \\
  \bottomrule
\end{tabular}
\end{table}

\FloatBarrier
\begin{figure}
  \centering
  \includegraphics{plot1.pdf}
  \caption{Ablenkung bei verschiedenen Spannungen $U_\text{d}$}
  \label{fig:plot1}
\end{figure}

\noindent
Die daraus folgenden Steigungen $\frac{D}{U_\text{d}}$ sind in der nachfolgenden Tabelle \ref{tab:messwerte2} aufgelistet.

\FloatBarrier
\begin{table}
  \centering
  \caption{Empfindlichkeiten}
  \label{tab:messwerte2}
  \begin{tabular}{c c c c c c}
  \toprule
   & $U_B 1$ in V & $U_B 2$ in V & $U_B 3$ in V & $U_B 4$ in V & $U_B 5$ in V \\
  \midrule
   & 200 & 230 & 250 & 280 & 300 \\
  \bottomrule
  a & 0.00163 & 0.00141 & 0.00130 & 0.00117 & 0.00110 \\
  \bottomrule
\end{tabular}
\end{table}

\noindent

\subsection{Bestimmung der Apparaturkonstante}
\label{sec:Apparaturkonstante}

Die Apparaturkonstante K
\begin{align*}
  \text{K} = \frac{\text{pL}}{\text{2d}}
\end{align*}
soll berechnet werden.
Dazu werden die zuvor ermittelten Steigungen gegen die reziproke Beschleunigungsspannug $U_\text{B}$ in Abbildung \ref{fig:plot2} aufgetragen und eine lineare Ausgleichsrechnung durchgeführt.

\FloatBarrier
\begin{figure}
  \centering
  \includegraphics{plot2.pdf}
  \caption{Empfindlichkeit gegen reziproke Beschleunigungsspannug}
  \label{fig:plot2}
\end{figure}

\noindent
Die daraus berechnete Steigung beläuft sich auf $a$ = 0.3190.
Mit Hilfe der Gleichung
\begin{align*}
  D = \frac{\text{p}}{\text{2d}}\text{L}\frac{U_\text{d}}{U_\text{B}} \Leftrightarrow \frac{D}{U_\text{d}} = \frac{\text{Lp}}{2d} \frac{1}{U_\text{B}} = a \cdot \frac{1}{U_\text{B}}
\end{align*}
entspricht die Steigung der Ausgleichsgeraden der gesuchten Apparaturkonstanten.
Durch die angegebenen Abmessungen des Versuchsaufbaus kann diese auch theoretisch noch berechnet werden.

\begin{align*}
  \text{p} = 1.9 \text{cm}, & \text{L} = 14.3 \text{cm}, \text{d} = 0.38 \text{cm} \\
  & a = 35.75 \text{cm}	\widehat{=} 0.3575 \text{m}
\end{align*}

Die Abweichung zwischen beiden Werten für die Apparaturkonstante liegt bei 10,77\%.

\subsection{Synchronisationsfrequenz}
\label{synchro}

Bei den folgenden Frequenzen in der Tabelle \ref{tab:messwerte4} sind bei einer Beschleunigungsspannug $U_B = 422$V stehende Bilder entstanden.
Die dazugehörige Zahl n kann anhand der Wellenbäuche der stehenden Wellen gut erkannt werden.

\FloatBarrier
\begin{table}
  \centering
  \caption{Sägezahnspannungsfrequenzen}
  \label{tab:messwerte4}
  \begin{tabular}{c c}
  \toprule
   $\nu$ in Hz & n \\
  \midrule
  22.63 & 3 \\
  39.98 & 2 \\
  79.99  & 1 \\
  159.69 & $\frac{1}{2}$ \\
  \bottomrule
  \bottomrule
\end{tabular}
\end{table}

Daraus ergibt sich die mittlere Frequenz von $\nu_{sinus} = (79.92 \pm 0.07) \text{Hz}$, diese entspricht der Sinusfrequenz des angeschlossenen Sinusgenerators.
Mit der beobachteten Auslenkung $D = 13$mm und der Formel
\begin{align*}
  D = K \frac{U_d}{U_B} \Leftrightarrow U_d = \frac{D U_B}{K}
\end{align*}
wird der Scheitelwert der angelegten Sinusspannung berechnet.
Es ergibt sich $U_d = 16.25$V.


\subsection{Spezifische Ladung der Elektronen}
\label{spezLadung}

Um die Ablenkung durch das Magnetfeld berechnen zu können, werden in der Messreihe
der Spulenstrom $I$, die Stärke des an der Spule herrschenden Magnetfelds $B$ sowie die Ablenkung $D$ der Elektronen, die auf dem Leuchtschirm sichtbar ist, aufgenommen.

\FloatBarrier
\begin{table}
  \centering
  \caption{Messwerte für die Spule mit 20 Windungen bei $U_B = 250$V}
  \label{tab:messwerte3}
  \begin{tabular}{c c c}
  \toprule
   $I$ in A & $B$ in mT & $D$ in cm \\
  \midrule
  0    & 0 & 0 \\
  0.28 & 0.018 & 0.6 \\
  0.6  & 0.038 & 1.2 \\
  0.94 & 0.060 & 1.8 \\
  1.25 & 0.080 & 2.4 \\
  1.56 & 0.099 & 3.0 \\
  1.9  & 0.121 & 3.6 \\
  2.25 & 0.143 & 4.2 \\
  2.59 & 0.165 & 4.8 \\
  \bottomrule
  \bottomrule
\end{tabular}
\end{table}

\FloatBarrier
\begin{table}
  \centering
  \caption{Messwerte für die Spule mit 20 Windungen bei $U_B = 350$V}
  \label{tab:messwerte3}
  \begin{tabular}{c c c}
  \toprule
   $I$ in A & $B$ in mT & $D$ in cm \\
  \midrule
  0    & 0 & 0 \\
  0.34 & 0.022 & 0.6 \\
  0.74  & 0.047 & 1.2 \\
  1.1 & 0.070 & 1.8 \\
  1.46 & 0.093 & 2.4 \\
  1.85 & 0.118 & 3.0 \\
  2.24  & 0.143 & 3.6 \\
  2.65 & 0.169 & 4.2 \\
  3.05 & 0.195 & 4.8 \\
  \bottomrule
  \bottomrule
\end{tabular}
\end{table}

\noindent

Es wird die $\frac{D}{L^2 + D^2}$ gegen das Magnetfeld $B$ in die Abbildung \ref{fig:plot3} aufgetragen und eine lineare Ausgleichsrechnung durchgeführt.
\begin{align*}
  \frac{D}{L^2 +D^2} = -0.0717 \cdot B_1 + 0.1667 \\
  \frac{D}{L^2 +D^2} = -0.0916 \cdot B_2 + 0.1963 \\
  b_1 = -0.0717 \\
  b_2 = -0.0916
\end{align*}

\FloatBarrier
\begin{figure}
  \centering
  \includegraphics{plot2.pdf}
  \caption{D/(L²+D²) gegen B mit N=20}
  \label{fig:plot3}
\end{figure}

Aus der Steigung der Ausgleichsgeraden wird mit folgender Formel die spezifische Ladung der Elektronen berechnet.
\begin{align*}
  \frac{D}{L^2+D^2} = \frac{1}{\sqrt{8 U_B}} \sqrt{\frac{e_0}{m_0}} B \\
  \Leftrightarrow b = \frac{1}{\sqrt{8 U_B}}\sqrt{\frac{e_0}{m_0}}
  \Leftrightarrow \frac{e_0}{m_0} = b^2 \cdot 8 U_B
\end{align*}

Mit den berechneten Steigungen $b_1 \text{und} b_2$ ergeben sich für die speziefische Ladung die Werte:

\begin{align*}
  \frac{e_0}{m_0} = 1.217 \cdot 10^7 \\
  \frac{e_0}{m_0} = 2.347 \cdot 10^7
\end{align*}

Als Mittelwert erhält man somit für die spezifische Ladung $\frac{e_0}{m_0} = 1.782 \cdot 10^7 \pm 1.130 \cdot 10^7 $.
Dieser weicht vom Literaturwert ($\frac{e_0}{m_0} = 1.759 \cdot 10^11$) um 9864 \% ab.

\subsection{Inklinationswinkel}
\label{inklination}

Bei einer Beschleunigungsspannug $U_b = 200V$ kann mit einer Stromstärke von $I = 0.198 A$ ein Gegenfeld zum Erdmagnetfeld mit der Stärke von $B = 0.0126 \text{mT}$ erzeugt werden, um den Leuchtfleck wieder in Ausgangslage zu bringen.
Mit Hilfe des Deklinatoriums wird der Inklinationswinkel auf $\phi = 71 \circ$ bestimmt.
Durch geometrische Zusammenhänge kann die Totalintensität des Erdmagnetfeldes berechnet werden mit:
\begin{align*}
  B_{total} = \frac{B}{\text{cos}(phi)}
\end{align*}
Somit ergibt sich die Totalintensität des Magnetfeldes zu $B_{total} = 0.0286 mT$.
Verglichen mit dem Literaturwert der Totalintensität des Magnetfeldes in Mitteleuropa ($B_{total} = 0.048 mT$) ergibt sich eine Abweichung von 40.42 \%.
