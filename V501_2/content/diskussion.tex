\section{Diskussion}
\label{sec:Diskussion}

Durch den nicht unendlich dünn auflösbaren Leuchtpunkt sind exakte Messungen der Ablenkung nicht möglich.
Diese Ungenauigkeit konnte lediglich durch Fokusierung des Punktes verkleinert, aber nicht eliminiert werden. \\
\noindent 
Bei der Messung der Ausrichtung des Ermagnetfeldes und des Inklinationswinkels gab es Ungenauigkeit durch die Schwerfälligkeit des Deklinatoriums.
Die Drehachsen waren zu schwergängig um jedes Mal zuverlässige Ergebnisse zu liefern.
Nur mit sehr viel Feingefühl sind akzeptable Werte zu vermessen. \\
\noindent 
Die Abweichungen der Ergebisse liegen noch im akzeptablen Bereich , auch auf die spezifische Ladung von Elektronen liegt in der richtigen Größenordnung, totz 88 \% Abweichung. \\
\noindent 
Die Frequenz der Sinusspannung ist als gut zu schätzen, weil am Sinusgenerator zum Vergleich ein ungefähres Frequenzspektrum von 80-90 Hz angegeben war.
Die errechneten 79.92 Hz liegen nah genug am Spektrum um realistisch zu sein.