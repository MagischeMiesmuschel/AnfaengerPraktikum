\section{Theorie}
\label{sec:Theorie}

Der Frequenzbereich von ca. 20 kHz bis ca 1 GHz wird Ultraschall genannt.
Schall ist eine longitudinale Welle, welche sich saufgrund von Druckschwnakungen fortbewegt.
Die akustische Impedanz ist $Z = c * \rho$, die durch die Dichte $\rho$ des durchstrahlten Materials und der Schallgeschwindigkeit in diesem Material bestimmt wird.
Es treten, ähnlich wie bei elektromagnetischen Wellen, Reflexion, Brechung und ähnliche Effekte auf.
Aber die Phasengeschwindigkeit der Schallwelle ist materialabhängig. \\

In Gasen und Flüssigkeiten breitet sich der Schall immer als Longitudinalwelle aus.
Die Schallgeschwindigkeit hängt z.B. bei einer Flüssigkeit von ihrer Kompressibilität $\kappa$ und ihrer Dichte $\rho$ ab.
\begin{equation}
    c_{Fl} = \sqrt{\frac{1}{\kappa \rho}}
    \label{eqn:gl1}
\end{equation}
Bei einem Festkörper ist die Schallausbreitung komplizierter, da infolge Schubspannungen nicht nur Longitudinalwellen sondern auch Transversalwellen möglich sind.
Hier ersetzt bei der Berechnung der Schallgeschwindigkeit in einem Festkörper das Elastizitätsmodul $E$ die Kompressibilität $\kappa^{-1}$.
\begin{equation}
    c_{Fl} = \sqrt{\frac{E}{\rho}}
    \label{eqn:gl2}
\end{equation}
In Festkörpern sind Schallwellen Richtungsabhängig. \\

Bei der Schallausbreitung geht ein Teil der Energie durch Absorption verloren.
Die Intensität $I_0$ nimmt exponentiell nach der Strecke $x$ ab.
\begin{equation}
    I(x) = I_0 \cdot e^{-\alpha x}
    \label{eqn:gl3}
\end{equation}
$\alpha$ ist der Absorptionskoeffizient der Schallamplitude. \\

Beim Treffen auf eine Grenzfläche wird ein Teil der Schallwelle reflektiert.
Der Reflexionskoeffizient $R$, das Verhältnis von reflektierten zu einfallender Schallintensität, setzt sich dabei aus der akustischen Impedanz der beiden angrenzenden Materialien zusammen.
\begin{equation}
    R = \left(\frac{Z_1-Z_2}{Z_1+Z_2}\right)^2
\end{equation}
Der Transmittierte Anteil $T$ läßt sich aus $T = 1 − R$ berechnen.\\

Ultraschall kann auf verschiedene Arten erzeugt werden.
Die Anwendung des reziproken piezo-elektrischen Effekts ist eine.
Bringt man einen piezoelektrischen
Kristall in ein elektrisches Wechselfeld, so kann man diesen zu Schwingungen anregen, wenn eine polare Achse des Kristalls in Richtung des elektrischen Feldes zeigt.
Der Piezokristall strahlt beim Schwingen Ultraschallwellen ab.
Für Resonanz der Anregerfrequenz und Eigenfrequenz können große Amplituden erzeugt werden.
Der Piezokristall kann auch umgekehrt als Schallempfänger genutzt werden, hierbei treffen die Schallwellen auf den Kristall und regen diesen zu Schwingungen an.
Quarze sind dabei die meist benutzten piezoelektrischen Kristalle, da sie gleichbleibende physikalische Eigenschaften haben. Jedoch haben sie einen relativ schwachen piezoelektrischen Effekt.\\

Durch Laufzeitmessungen erhält man Informationen über durchstrahlte Körper.
Es werden die beiden Methoden, das Durchschallungs-Verfahren und das Impuls-Echo-Verfahren verwendet.\\

Beim Durchschallungs-Verfahren wird mit einem Ultraschallsender ein kurzzeitiger Schallimpuls ausgesendet und am anderen Ende
des Probenstücks mit einem Ultraschallempfänger aufgefangen.
Befindet sich eine Fehlstelle in der durchstrahlten Probe, so wird eine abgeschwächte Intensität am Ultraschallempfänger gemessen.
Eine Aussage darüber, wo sich die Fehlstelle in der Probe befindet, ist nicht möglich. \\

Beim Impuls-Echo-Verfahren wird der Ultraschallsender auch als Empfänger verwendet.
Der ausgesendete Ultraschallpuls wird hierbei an einer Grenzfläche reflektiert und nach seiner Rückkehr von der Empfänger aufgenommen.
Bei Fehlstellen kann die Höhe des Echos Aufschluß über die Größe der Fehlstelle geben.
Bei bekannter Schallgeschwindigkeit kann aus der Laufzeit t die Lage der Fehlstelle über
\begin{equation}
    s = \frac{1}{2} c t
    \label{eqn:gl4}
\end{equation}
bestimmt werden.
Die Laufzeitdiagramme können in einem A-Scan, B-Scan oder einem TM-Scan dargestellt werden. \\

Der A-Scan (Amplitude Scan) ist ein eindimensionales Verfahren zum Abtasten von Strukturen,
bei dem die Echoamplituden als Funktion der Zeit abgebildet werden.
Mit Hilfe des B-Scans (Brightness Scan) kann durch das Bewegen der Sonde ein zweidimensionales Bild aufgezeichnet werden.
Dabei werden die Echoamplituden in verschiedenen Helligkeitsstufen dargestellt.
Der TM-Scan (Time-Motion Scan) kann die Bewegung eines Objekts aufgenommen werden,
indem durch schnelles Abtasten eine zeitliche Bildfolge erstellt wird. \cite{US2}
