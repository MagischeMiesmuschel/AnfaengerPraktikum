\section{Diskussion}
\label{sec:Diskussion}
Im Auswertungsteil \ref{sec:fehlstelle} ist gut zu erkennen, dass der Fehler beim Durchmesser der Fehlstellen bei größeren deutlich kleiner ist.
Wobei es bei kleinen Fehlstellen einen größeren Fehler gibt.
Dies ist mit den Ergebnissen aus dem Auswertungsteil \ref{sec:auf} zu erklären.
Die Messungen mit den beiden Sonden zeigt eine Frequenzabhängigkeit für das Auflösungsvermögen.
Je höher die Frequenz der Sonde gewählt wird, desto besser löst sie auf.
Denn bei höherer Frequenz wird die Wellenlänge kleiner und kann besser an kleinen Stellen verwendet werden.
Jedoch nimmt auch die Amplitude des Signals mit steigender Frequenz ab und deswegen muss eine mittlere Frequenz gewählt werden,
bei der die Ausschläge noch groß genug sind, aber das Auflösungsvermögen maximal ist.
Deswegen wurden mit der 2 MHz Sonde im Auswertungsteil \ref{sec:fehlstelle} die kleineren Stellen ungenauer bestimmt, 
aber es konnte auch nach einem langem Weg im Acryl eine gute Amplitude zum Auswerten vermerkt werden.
Mit der gleichen Erklärung kann in den Abbildungen im Auswertungsteil \ref{sec:bscan} auch das "Verschmieren" der kleineren Fehlstellen erklärt werden.