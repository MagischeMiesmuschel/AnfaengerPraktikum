\section{Theorie}
\label{sec:Theorie}

Ein elektrischer Schwingkreis setzt sich zusammen aus einer Kapazität $C$ in Form eines Kondensators und einer Induktivität $L$ in Form einer Spule.
Eine eingespeißte Energiemenge pendelt zwischen diesen beiden Energiespeichern, der Strom wechselt periodisch seine Richtung.
Das System ist also dazu in der Lage, periodische Schwinungen durchzuführen, bei denen die Energie erhalten bleibt.
In der Realität jedoch haben die Bauteile wie Spule, Kondensatoren und Kabel einen ohmschen Widerstand $R$, der dem System Energie in Form von Wärme entzieht.
Für den Schwingkreis mit ohmschen Widerstand $R$ ergibt sich also eine gedämpfte Schwingung, Stromstärke und Spannung nehmen mit der Zeit ab.
\begin{figure}
  \centering
  \includegraphics{}
  \caption{Schematischer Aufbau RLC-Kreis}
  \label{fig:abb1}
\end{figure}
Mit Hilfe der 2. Kirchhoffschen Regel lässt sich eine Differntialgleichung zur Beschreibung des Problems finden.
Die in Abbildung \ref{fig:abb1} dargestellten Spannungen addieren sich zu Null:
\begin{equation}
  U_R(t) + U_C(t) + U_L(t)  = 0
  \label{eqn:gl1}
\end{equation}
