\section{Theorie}
\label{sec:Theorie}

Ein elektrischer Schwingkreis setzt sich zusammen aus einer Kapazität $C$ in Form eines Kondensators und einer Induktivität $L$ in Form einer Spule.
Eine eingespeißte Energiemenge pendelt zwischen diesen beiden Energiespeichern, der Strom wechselt periodisch seine Richtung.
Das System ist also dazu in der Lage, periodische Schwinungen durchzuführen, bei denen die Energie erhalten bleibt.
In der Realität jedoch haben die Bauteile wie Spule, Kondensatoren und Kabel einen ohmschen Widerstand $R$, der dem System Energie in Form von Wärme entzieht.
Für den Schwingkreis mit ohmschen Widerstand $R$ ergibt sich also eine gedämpfte Schwingung, Stromstärke und Spannung nehmen mit der Zeit ab.
\begin{figure}
  \centering
  \includegraphics{}
  \caption{Schematischer Aufbau RLC-Kreis}
  \label{fig:abb1}
\end{figure}
Mit Hilfe der 2. Kirchhoffschen Regel lässt sich eine Differntialgleichung zur Beschreibung des Problems finden.
Die in Abbildung \ref{fig:abb1} dargestellten Spannungen addieren sich zu Null:
\begin{equation}
  U_R(t) + U_C(t) + U_L(t)  = 0
  \label{eqn:gl1}
\end{equation}
Die Spannungen können mit dem Ohmschen Gesetz \eqref{eqn:gl2}, dem Induktionsgesetz \eqref{eqn:gl3} und der Defintion der Kapazität \eqref{eqn:gl4} geschrieben werden als:
\begin{equation}
  U_R(t) = RI(t)
  \label{eqn:gl2}
\end{equation}
\begin{equation}
  U_L(t) = L \frac{dI}{dt}
  \label{eqn:gl3}
\end{equation}
\begin{equation}
  U_C(t) = \frac{Q(t)}{C}
  \label{eqn:gl4}
\end{equation}
Daraus folgt die Gleichung:
\begin{equation}
  RI(t) + \frac{Q(t)}{C} + L \frac{dI}{dt} = 0
  \label{eqn:gl5}
\end{equation}
Wird diese nach der Zeit abgeleitet, erhält man die gewünschte Differntialgleichung, die an den harmonischen Oszillator aus der Mechanik erinnert.
\begin{equation}
  \ddot{I(t)} + \frac{R}{L} \dot{I(t)} + \frac{1}{LC}I(t) = 0
  \label{eqn:gl6}
\end{equation}
Ein Lösungsansatz der Gleichung ist:
\begin{equation}
  I(t) = e^{-2\pi\mu t}(A_1 e^{\text{i}2\pi\nu t} + A_2 e^{-\text{i}2\pi\nu t})
  \label{eqn:gl7}
\end{equation}
mit den Abkürzungen:
\begin{equation}
  \mu := \frac{R}{4\pi L}
  \label{eqn:gl8}
\end{equation}
\begin{equation}
  \nu := \frac{1}{2\pi} \sqrt{\frac{1}{LC} - \frac{R^2}{4L^2}}
  \label{eqn:gl9}
\end{equation}
Die Form der Lösung ist im Besonderen von \nu abhängig, das je nachdem ob die Diskriminante positiv oder negativ ist, reel oder komplex wird.
Deshalb wird für die weitere Betrachtung eine Fallunterscheidung vorgenommen.
