\section{Diskussion}
\label{sec:Diskussion}
Wie in der Auswertung bereits festgestellt, weicht die berechnete Güteziffer deutlich von der idealen Güteziffer ab.
Die größte Abweichung beträgt $95,00 \, \%$ und die kleinste $76,54 \, \%$ (siehe Tabelle \ref{tab:guete}).
Auch der Wirkunsgrad der Wärmepumpe ist relativ niedrig und liegt nur zwischen $24,8 \, \%$ und $21 \, \%$ (siehe Tabelle \ref{tab:Wirkungsgrad}).
Als Ursache für dieses Ergebnis kommt zum einen der Aufbau in Betracht.
Die Leitungen sind nicht optimal isoliert, insbesondere zwischen den Reservoiren und Umgebung kann ein freier Wärmeaustausch stattfinden.
Die Eimer werden durch die Deckel nicht geschlossen und die Umgebungsluft kann in die Eimer eindrigen.
Der Wärmeaustausch zwischen Transportgas und Wasser funktioniert nur bedingt.
Ein Rührmotor funktionierte während der Messung nicht.
Weitere Gründe für den Unterschied zwischen Theorie und Realität sind die vereinfachenden Annahmen, die in der Theorie getroffen werden.
Zur Berechnung der Güteziffer wird angenommen, dass der Prozess reversibel sei und der Kompressor adiabatisch arbeite.
In der Praxis ist dies nicht realisierbar, da es zum Beispiel durch Reibung zu Engergieverlusten kommt.
Außerdem wird bei der Berechnung der mechanischen Kompressorleistung die ideale Gasgleichung herangezogen, obwohl das Transportgas kein ideales Gas ist.
