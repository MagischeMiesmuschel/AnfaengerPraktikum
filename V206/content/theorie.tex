\section{Zielsetzung}
\label{sec:Ziel}
Es soll der Transport von Wärmeenergie entgegen der Richtung des Wärmeflusses untersucht werden.
Zur Untersuchung der Qualität werden Merkmale wie Güteziffer und Massendurchsatz bestimmt.

\section{Theorie}
\label{sec:Theorie}
Zur Umkehr des Wärmeflusses vom kälteren in ein wärmeres Reservoir benötigt es weitere Energie, zum Beispiel in Form von mechanischer Arbeit.
Diese wird von der Wärmepumpe geleistet.

\subsection{Güteziffer}
\label{sec:Güteziffer}
Das Verhältnis zwischen tranportierter Wärmeenergie $Q_{trans}$ und zu verrichtender Arbeit A:
\begin{equation}
  \nu = \frac{Q_{trans}}{A} \stackrel{\ref{eqn:eqn2}}\Rightarrow \nu_{id} = \frac{T_1}{T_1-T_2}
  \label{eqn:eqn1}
\end{equation}
wird durch die Güteziffer \nu bezeichnet.
Dessen Legitimation ergibt sich aus dem zweiten Hauptsatz der Thermodynamik.
Für realitätsgebundene Berechnungen folgt aus der Irreversibilität der ablaufenden Prozesse,
dem zweiten Hauptsatz der Thermodynamik und der Annahme, dass sich bei der Wärmeübertragung die Temperaturen der Reservoirs nicht ändern:
\begin{equation}
  \frac{Q_1}{T_1} - \frac{Q_2}{T_2} > 0
  \label{eqn:eqn2}
\end{equation}
Somit ist der Arbeitsaufwand der Pumpe geringer für kleinere Temperaturdifferenzen der beiden Reservoirs.
Die reale Güteziffer \nu lässt sich aus dem Differentialquotienten $\frac{dT_1}{dt}$ für ein Zeitintervall $dt$ bestimmen.
Daraus berechnet sich die Wärmemenge $\frac{dQ_1}{dt}$ zu:
\begin{equation}
  \frac{dQ_1}{dt} = (m_1c_w + m_kc_k) \frac{dT_1}{dt}
  \label{eqn:eqn3}
\end{equation}
Mit $m_1c_w$ für die Wärmekapazität des Wasser in Reservoir 1 und $m_kc_k$ die Wärmekapazität der Kupferschlange und des Eimers.
Somit folgt für die Güteziffer \nu :
\begin{equation}
  \nu = \frac{dQ_1}{dtN} \stackrel{\ref{eqn:eqn3}}\Rightarrow \nu = (m_1c_w + m_kc_k) \frac{dT_1}{dt} \cdot \frac{1}{N}
  \label{eqn:eqn4}
\end{equation}
mit N := gemittelte Leistungsaufnahme des Kompressors.

\subsection{Massendurchsatz}
\label{sec:Massendurchsatz}
Der Massendurchsatz berechnet sich nach \cite{AnleitungV206} über den Differentialquotienten über:
\begin{equation}
  \frac{dQ_2}{dt} = (m_2c_w + m_kc_k) \frac{dT_2}{dt}
  \label{eqn:eqn5}
\end{equation}
und
\begin{equation}
  \frac{dQ_2}{dt} = L \frac{dm}{dt}
  \label{eqn:eqn6}
\end{equation}
mit Gl. \ref{eqn:eqn5} und \ref{eqn:eqn6} zu:
\begin{equation}
  (m_2c_w + m_kc_k) \frac{dT_2}{dt} = L \frac{dm}{dt} \Leftrightarrow  \frac{dm}{dt} = (m_2c_w + m_kc_k) \frac{dT_2}{dt \cdot L}
  \label{eqn:eqn7}
\end{equation}
mit bekannter Verdampfungswärme L.

\subsection{Mechanische Kompressorleistung $N_{mech}$}
\label{sec:Kompleistung}
Für die Arbeit $A_m$ gilt bei Verringerung des Gasvolumens von $V_a \textrm{und} V_b$:
\begin{align}
  A_m = - \int_{V_a}^{V_b} pdV
  \label{eqn:eqn8}
\end{align}
Aus der adiabatischen Kompression des Kompressors, also eine Zustandsänderung ohne Wärmeverluste an die Umgebung, folgt mit der Poissonschen Gleichung und $N_{mech} = \frac{dA_m}{dt}$:
\begin{equation}
  N_{mech} = \frac{1}{\kappa - 1} \left( p_b \sqrt[\kappa]{\frac{p_a}{p_b}} - p_a \right) \frac{dV_a}{dt} = \frac{1}{\kappa - 1} \left( p_b \sqrt[\kappa]{\frac{p_a}{p_b}} - p_a \right) \frac{1}{\rho} \frac{dm}{dt}
  \label{eqn:eqn9}
\end{equation}
mit der Dichte $\rho$ des Gases und dem Druck $p_a$.
