\section{Auswertung}
\label{sec:Auswertung}

\subsection{verwendete Software und Fehlerrechnung}
\label{sec:SoftwareFehlerrechnung}

Für die Auswertung werden neben NumPy\cite{numpy} mehrere Python Pakete benutzt.
Plots werden mit Matplotlib\cite{matplotlib} erstellt und Ausgleichsgeraden mit SciPy\cite{scipy}.
Fehlerbehaftete Größen werden mit Uncertainties\cite{uncertainties} berechnet, das die Gaußsche Fehlerfortpflanzung benutzt:
\begin{equation*}
    \increment f = \sqrt{\sum_{i=1}^N \left( \frac{\partial f}{\partial x_i} \right)^{2} \cdot (\increment x_i)^{2}}
\end{equation*}
Alle Mittelwerte werden mit folgender Formel berechnet:
\begin{equation*}
  \bar{x} = \frac{1}{N} \sum_{i = 1}^N x_i
\end{equation*}
Der zugehörige Fehler berechnet sich mit:
\begin{equation*}
  \increment \bar{x} = \frac{1}{\sqrt{N}} \sqrt{\frac{1}{N-1} \sum_{i = 1}^N (x_i - \bar{x})^2}
\end{equation*}

\subsection{Messdaten und Systemgrößen}
\label{sec:Auswertung_Messdaten}

Für die Auswertung werden folgende Messwerte verwendet.
\begin{table}
  \centering
  \caption{Temperaturen der Reservoire, Drücke in den Leitungen, Kompressorleistung.}
  \label{tab:Messdaten}
  \begin{tabular}{c c c c c c}
    \toprule
    $t$ (min) & $T_1$ ($^\circ$C) & $T_2$ ($^\circ$C) & $p_a$ (Bar) & $p_b$ (Bar) & $P$ (w) \\
    \midrule
     1 & 20.9 & 19.7 & 1.4 &  5.8 & 170 \\
     2 & 21.8 & 19.6 & 1.8 &  6.4 & 180 \\
     3 & 23.8 & 18.5 & 1.9 &  7.0 & 187 \\
     4 & 26.4 & 17.0 & 2.1 &  7.5 & 195 \\
     5 & 28.8 & 15.4 & 2.2 &  8.0 & 203 \\
     6 & 31.3 & 13.6 & 2.2 &  8.5 & 205 \\
     7 & 33.6 & 11.8 & 2.2 &  9.0 & 208 \\
     8 & 35.7 & 10.2 & 2.2 &  9.4 & 210 \\
     9 & 37.7 &  8.6 & 2.2 &  9.8 & 210 \\
    10 & 39.7 &  7.0 & 2.2 & 10.2 & 212 \\
    11 & 41.5 &  5.5 & 2.2 & 10.6 & 212 \\
    12 & 43.2 &  4.0 & 2.2 & 11.0 & 210 \\
    13 & 44.9 &  2.7 & 2.2 & 11.5 & 212 \\
    14 & 46.5 &  1.4 & 2.2 & 12.0 & 214 \\
    15 & 48.1 &  0.7 & 2.2 & 12.3 & 215 \\
    16 & 49.5 &  0.1 & 2.3 & 12.7 & 214 \\
    17 & 50.9 & -0.2 & 2.2 & 13.0 & 214 \\
    \bottomrule
  \end{tabular}
\end{table}
Auf die Drücke wird noch 1 Bar addiert.
Das Volumen der Reservoire ist jeweils 3 Liter und die spezifische Wärmekapazität von Wasser ist $ c_w =4.182 \cdot 10^3$ J/(kgK).
Die Kupferleitungen der Wärmepumpe haben einen $c_k m_k$-Wert von 660 J/K.
Außerdem werden in späteren Abschnitten noch einige Stoffeigenschaften des Transportgases Dichloridflourmethan ($\text{Cl}_2\text{F}_2\text{C}$) benötigt.
\begin{align*}
  \rho_0 = 5,51 \, \text{g/l bei } T_0 = 0 \, ^\circ\text{C und } p_0 = 1 \, \text{Bar} \\
  \text{molare Masse: } M_{\text{Gas}} = 0,12091 \, \text{kg/mol}
\end{align*}
\FloatBarrier

\subsection{Bestimmung der Güteziffer}
\label{sec:Auswertung_Gueteziffer}

Für die gemessenen Temperaturen (siehe Tabelle \ref{tab:Messdaten}) wird eine Ausgleichsrechnung (siehe Abb. \ref{fig:plot1}) mit folgender Gleichung durchgeführt.
\begin{equation*}
  T(t) = At^2 + Bt + C
\end{equation*}
$T_1$ wird mit folgenden Parametern approximiert:
\begin{align*}
  A_1 &= \SI{-9(1)e-6}{\kelvin\per\second\squared} \\
  B_1 &= \SI{0.042(2)}{\kelvin\per\second} \\
  C_1 &= \SI{290.3(4)}{\kelvin} \\
\end{align*}
Für $T_2$ liefert die Ausgleichsrechnung:
\begin{align*}
  A_2 &= \SI{6(2)e-6}{\kelvin\per\second\squared} \\
  B_2 &= \SI{-0.030(3)}{\kelvin\per\second} \\
  C_2 &= \SI{296.2(6)}{\kelvin}
\end{align*}
\begin{figure}
  \centering
  \includegraphics{build/plot1.pdf}
  \caption{Temperaturverläufe während der Messung.}
  \label{fig:plot1}
\end{figure}
Mit Hilfe dieser Paramter lassen sich nun Differentialquotienten berechnen, die die Temperaturänderung zu bestimmten Zeiten beschreiben.
\begin{equation*}
  \frac{\symup{d}T(t)}{\symup{d}t} = 2At + B
\end{equation*}
In den Tabellen \ref{tab:diffT1} und \ref{tab:diffT2} sind für vier verschiedene Zeiten bzw. Temperaturen Differentialquotienten berechnet.
\begin{table}
  \centering
  \caption{Differentialquotienten für $T_1$.}
  \label{tab:diffT1}
  \begin{tabular}{c c c c}
    \toprule
    $t$ (s) & $T_1$ (K) & $\symup{d}T_1/\symup{t}$ (K/s) & Fehler (K/s) \\
    \midrule
    180 & 296,95 & 0,039 & 0,002 \\
    360 & 304,45 & 0,036 & 0,002 \\
    540 & 310,85 & 0,032 & 0,002 \\
    720 & 316,35 & 0,029 & 0,003 \\
    \bottomrule
  \end{tabular}
\end{table}
\begin{table}
  \centering
  \caption{Differentialquotienten für $T_2$.}
  \label{tab:diffT2}
  \begin{tabular}{c c c c}
    \toprule
    $t$ (s) & $T_2$ (K) & $\symup{d}T_2/\symup{t}$ (K/s) & Fehler (K/s) \\
    \midrule
    180 & 291,65 & -0,028 & 0,003 \\
    360 & 286,75 & -0,026 & 0,003 \\
    540 & 281,75 & -0,024 & 0,004 \\
    720 & 277,15 & -0,021 & 0,004 \\
    \bottomrule
  \end{tabular}
\end{table}
\FloatBarrier
Mit Gleichung \eqref{eqn:eqn4} kann nun die Güteziffer $\nu$ bestimmt werden, indem die entsprechenden Größen aus Kapitel \ref{sec:Auswertung_Messdaten} und die Differentialquotienten für $T_1$ aus Tabelle \ref{tab:diffT1} eingesetzt werden.
Die Theoriewerte der Gütezifffer $\nu_{\text{ideal}}$ werden mit Gleichung \eqref{eqn:eqn1} berechnet.
\begin{table}
  \centering
  \caption{gemessene und ideale Güteziffer.}
  \label{tab:guete}
  \begin{tabular}{c c c c c c}
    \toprule
    $t$ (s) & $T_1$ (K) & $\nu$ & Fehler & $\nu_{\text{ideal}}$ & relative Abweichung von $\nu_{\text{ideal}}$ \\
    \midrule
    180 & 296,95 & 2,8 & 0,1 & 56,0 & 95,00 $\%$ \\
    360 & 304,45 & 2,3 & 0,1 & 17,2 & 86,63 $\%$ \\
    540 & 310,85 & 2,0 & 0,1 & 10,7 & 81,31 $\%$ \\
    720 & 316,35 & 1,9 & 0,1 &  8,1 & 76,54 $\%$ \\
    \bottomrule
  \end{tabular}
\end{table}
Auf die deutliche Abweichung zwischen Theorie und Praxis wird später in der Diskussion eingegangen.
\FloatBarrier

\subsection{Bestimmung des Massendurchsatzes}
\label{sec:Auswertung_Massendurchsatz}

Zur Bestimmung des Massendurchsatzes muss zunächst über eine lineare Ausgleichsrechnung die Verdampfunswärme berechnet werden.
Dafür werden der Druck $p_b$ und die Temperatur $T_1$ in einer Dampfdruck-Kurve (siehe Abb. \ref{fig:plot2}) mit $p_0$ = 1 Bar und der allgemeinen Gaskonstante $R$ aufgetragen.
\begin{equation*}
  \ln{\frac{p_b}{p_o}} = - \frac{L}{R} \cdot \frac{1}{T_1}
\end{equation*}
Für die Ausgleichsgerade $g(T_1) = m \cdot T_1 + n$ ergeben sich folgende Parameter:
\begin{align*}
  m &= -2111 \pm 54 \\
  n &= 9,2 \pm 0,2 \\
\end{align*}
Die Verdampfunswärme $L$ beträgt dann:
\begin{equation*}
  L = - m \cdot R = \SI{1,76(4)e4}{\kelvin\per\mol}
\end{equation*}
\begin{figure}
  \centering
  \includegraphics{build/plot2.pdf}
  \caption{Dampfdruck-Kurve des Transportgases.}
  \label{fig:plot2}
\end{figure}
\FloatBarrier
Der Massendurchsatz kann nun mit Gleichung \eqref{eqn:eqn8} errechnet werden.
Da die Verdampfunswärme pro Mol berechnet wird, wird zunächst die zeitliche Änderung der Stoffmenge $n_{\text{Gas}}$ bestimmt.
Durch Multiplikation mit der molaren Masse des Transportgases (siehe Kapitel \ref{sec:Auswertung_Messdaten}) wird der Massendurchsatz berechnet.
\begin{table}
  \centering
  \caption{Massendurchsatz.}
  \label{tab:Massendurchsatz}
  \begin{tabular}{c c c c c}
    \toprule
    $t$ (s) & $\symup{d}n_{\text{Gas}}/\symup{d}t$ (mol/s) & Fehler (mol/s) & $\symup{d}m/\symup{d}t$ & Fehler (kg/s)  \\
    \midrule
    180 & -0,021 & 0,002 & -0,0025 & 0,0003 \\
    360 & -0,020 & 0,002 & -0,0023 & 0,0003 \\
    540 & -0,018 & 0,003 & -0,0021 & 0,0003 \\
    720 & -0,016 & 0.003 & -0,0019 & 0,0004 \\
    \bottomrule
  \end{tabular}
\end{table}

\subsection{Bestimmung der mechansichen Kompressorleistung und Wirkungsgrad}
\label{sec:Auswertung_Leistung}

Um die Kompressorleistung zu berechnen, ist es notwendig, die Dichte des Transportgases $\rho$ zu kennen.
Diese kann aus der idealen Gasgleichung bestimmt werden.
\begin{equation*}
 \rho = \frac{\rho_0 T_0 p_a}{T_2 p_0}
\end{equation*}
Aus Kapitel \ref{sec:Auswertung_Messdaten} ist $\rho_0$ bekannt und aus Kapitel \ref{sec:Auswertung_Massendurchsatz} der Massendurchsatz.
Das Verhältnis der Molwärmen $\kappa$ soll 1,14 sein.
Mit diesen Angaben kann nun aus Gleichung \eqref{eqn:eqn9} die Kompressorleistung berechnet werden.
\begin{table}
  \centering
  \caption{mechansiche Kompressorleistung.}
  \label{tab:Leistung}
  \begin{tabular}{c c c}
    \toprule
    $t$ (s) & $N_{\text{mech}}$ (W) & Fehler (W) \\
    \midrule
    180 & 46 & 5 \\
    360 & 45 & 5 \\
    540 & 46 & 7 \\
    720 & 45 & 9 \\
    \bottomrule
  \end{tabular}
\end{table}
\FloatBarrier
Der Wirkungsgrad ist der Quotient aus mechansicher Kompressorleistung $N_\text{mech}$ und der aufgebrachten elektrischen Leistung $P$.
\begin{table} wirk
  \centering
  \caption{Wirkungsgrad.}
  \label{tab:Wirkungsgrad}
  \begin{tabular}{c c c}
    \toprule
    $t$ (s) & $N_{\text{mech}}/P$ & Fehler \\
    \midrule
    180 & 24,8 \% & 2,5 \% \\
    360 & 22,1 \% & 2,7 \% \\
    540 & 22,0 \% & 3,3 \% \\
    720 & 21,0 \% & 4,0 \% \\
    \bottomrule
  \end{tabular}
\end{table}
