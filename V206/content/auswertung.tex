\section{Auswertung}
\label{sec:Auswertung}

Für die Auswertung werden folgende Messwerte verwendet.

\begin{table}
  \centering
  \caption{Eine Tabelle mit Messdaten.}
  \label{tab:some_data}
  \begin{tabular}{c c c c c c}
    \toprule
    $t$ (min) & $T_1$ ($\circ$C) & $T_2$ ($\circ$C) & $p_a$ (bar) & $p_b$ (bar) & $P$ (w) \\
    \midrule
     1 & 20.9 & 19.7 & 1.4 &  5.8 & 170 \\
     2 & 21.8 & 19.6 & 1.8 &  6.4 & 180 \\
     3 & 23.8 & 18.5 & 1.9 &  7.0 & 187 \\
     4 & 26.4 & 17.0 & 2.1 &  7.5 & 195 \\
     5 & 28.8 & 15.4 & 2.2 &  8.0 & 203 \\
     6 & 31.3 & 13.6 & 2.2 &  8.5 & 205 \\
     7 & 33.6 & 11.8 & 2.2 &  9.0 & 208 \\
     8 & 35.7 & 10.2 & 2.2 &  9.4 & 210 \\
     9 & 37.7 &  8.6 & 2.2 &  9.8 & 210 \\
    10 & 39.7 &  7.0 & 2.2 & 10.2 & 212 \\
    11 & 41.5 &  5.5 & 2.2 & 10.6 & 212 \\
    12 & 43.2 &  4.0 & 2.2 & 11.0 & 210 \\
    13 & 44.9 &  2.7 & 2.2 & 11.5 & 212 \\
    14 & 46.5 &  1.4 & 2.2 & 12.0 & 214 \\
    15 & 48.1 &  0.7 & 2.2 & 12.3 & 215 \\
    16 & 49.5 &  0.1 & 2.3 & 12.7 & 214 \\
    17 & 50.9 & -0.2 & 2.2 & 13.0 & 214 \\
    \bottomrule
  \end{tabular}
\end{table}

\subsection{verwendete Software und Fehlerrechnung}
\label{sec:SoftwareFehlerrechnung}

Für die Auswertung werden neben NumPy\cite{numpy} mehrere Python Pakete benutzt.
Plots werden mit Matplotlib\cite{matplotlib} erstellt und Ausgleichsgeraden mit SciPy\cite{scipy}.
Fehlerbehaftete Größen werden mit Uncertainties\cite{uncertainties} berechnet, das die Gaußsche Fehlerfortpflanzung benutzt:
\begin{equation*}
    \increment f = \sqrt{\sum_{i=1}^N \left( \frac{\partial f}{\partial x_i} \right)^{2} \cdot (\increment x_i)^{2}}
\end{equation*}
Alle Mittelwerte werden mit folgender Formel berechnet:
\begin{equation*}
  \bar{x} = \frac{1}{N} \sum_{i = 1}^N x_i
\end{equation*}
Der zugehörige Fehler berechnet sich mit:
\begin{equation*}
  \increment \bar{x} = \frac{1}{\sqrt{N}} \sqrt{\frac{1}{N-1} \sum_{i = 1}^N (x_i - \bar{x})^2}
\end{equation*}

\subsection{Bestimmung der Güteziffer}



\begin{figure}
  \centering
  \includegraphics{build/plot1.pdf}
  \caption{Temperaturverläufe während der Messung.}
  \label{fig:plot1}
\end{figure}

\subsection{Bestimmung des Massendurchsatzes}


\begin{figure}
  \centering
  \includegraphics{build/plot2.pdf}
  \caption{Dampfdruck-Kurve des Transportgases.}
  \label{fig:plot2}
\end{figure}

\subsection{Bestimmung der mechansichen Kompressorleistung}
