\section{Auswertung}
\label{sec:Auswertung}

Der anfangs gemessene Abstand $L$ von dem optischen Element zur Messonde sowie die Wellenlänge $\lambda$ des Lasers lauten:
\begin{align*}
  L &= 1 \text{m} \\
  \lambda &= 635 \text{nm}
\end{align*}

Die Werte zur Bestimmung der Spaltgröße $b$ des Einzelspalts über das Beugungsmuster befinden sich nachfolgend in Tabelle \ref{tab:einzel}.

\begin{table}
  \centering
  \caption{Einzelspalt Messwerte.}
  \label{tab:einzel}
\begin{tabular}{c c | c c}
  \toprule
  $\zeta$ in mm & $I$ in nA & $\zeta$ in mm & $I$ in nA \\
  \midrule
  31.0  &  0.005  &  20.5  &  1.2 \\
  30.5  &  0.0085  &  20.0  &  1.12 \\
  30.0  &  0.016  &  19.5  &  0.97 \\
  29.5  &  0.0245  &  19.0  &  0.79 \\
  29.0  &  0.033  &  18.5  &  0.6 \\
  28.5  &  0.3081  &  18.0  &  0.4 \\
  28.0  &  0.038  &  17.5  &  0.24 \\
  27.5  &  0.0318  &  17.0  &  0.125 \\
  27.0  &  0.0208  &  16.5  &  0.054 \\
  26.5  &  0.0119  &  16.0  &  0.0168 \\
  26.0  &  0.0115  &  15.5  &  0.01 \\
  25.5  &  0.0317  &  15.0  &  0.0185 \\
  25.0  &  0.081  &  14.5  &  0.031 \\
  24.5  &  0.17  &  14.0  &  0.04 \\
  24.0  &  0.288  &  13.5  &  0.0422 \\
  23.5  &  0.442  &  13.0  &  0.0378 \\
  23.0  &  0.622  &  12.5  &  0.028 \\
  22.5  &  0.799  &  12.0  &  0.0179 \\
  22.0  &  0.94  &  11.5  &  0.0092 \\
  21.5  &  1.1  &  11.0  &  0.0043 \\
  21.0  &  1.24  &  10.5  &  0.0031 \\
  \bottomrule
\end{tabular}
\end{table}
\FloatBarrier

Die Stromstärken in der Tabelle sind schon vom vorher gemessenen Dunkelstrom $I_{dunkel} = 0.85$nA bereinigt.
Mit Hilfe der Messwerte und der Gleichung \eqref{???} wird eine Ausgleichsrechnung durchgeführt. \\

Es wird für den Winkel $\phi = \text{\phi} = \frac{\zeta - \zeta_0}{L}$ angenommen.
Dabei ist $\zeta$ die Position des Detektors und $\zeta_0$ die Position des Hauptmaximums. \\

Dies wird auf Grund der Kleinwinkelnäherung angenommen, welche genutz wird, weil hier die Frauenhofergleichung genutzt wird.
Das Licht kommt aus dem "unendlichen" und tritt parallel durch den Spalt und trifft im unendlichenerst auf den Schirm.
Durch diese großen Distanzen kann die Kleinwinkelnäherung genutzt werden. \\

Die Regression des Einzelspalts befindet sich in Abbildung \ref{fig:plot1}

\begin{figure}
  \centering
  \includegraphics{plot1.pdf}
  \caption{Beugungsmuster des Einzelspalts.}
  \label{fig:plot1}
\end{figure}
\FloatBarrier

Es ergeben sich folgende Werte:

\begin{align*}
  \zeta_0 &= (20.78 \pm 0.04) \text{mm} \\
  b &= (0.1224 \pm 0.0021) \text{mm} \\
  A_0 &= (8.92 \pm 0.13) \text{nA} \\
\end{align*}

Die Hersteller Angaben hatten folgende Werte:
\begin{align*}
  b_{Her} &= 0.15 \text{mm} \\
  \Rightarrow \Delta b &= \frac{b - b_{Her}}{b_{Her}} = 18.4 \%
\end{align*}

Die gemessenen Daten des ersten Doppelspalts befinden sich in Tabelle \ref{tab:doppel1}, wie zurvor vom Dunkelstrom bereinigt.

\begin{table}
  \centering
  \caption{Doppelspalt 1 Messwerte.}
  \label{tab:doppel1}
\begin{tabular}{c c | c c}
  \toprule
  $\zeta$ in mm & $I$ in nA & $\zeta$ in mm & $I$ in nA \\
  \midrule
  30.3  &  0.0165  &  23.0  &  1.02 \\
  29.8  &  0.0225  &  22.8  &  1.01 \\
  29.3  &  0.027  &  22.6  &  0.89 \\
  28.8  &  0.031  &  22.4  &  0.7 \\
  28.3  &  0.034  &  22.2  &  0.46 \\
  27.8  &  0.028  &  22.0  &  0.28 \\
  27.3  &  0.023  &  21.8  &  0.155 \\
  26.8  &  0.028  &  21.6  &  0.11 \\
  26.3  &  0.046  &  21.4  &  0.185 \\
  25.8  &  0.13  &  21.2  &  0.2 \\
  25.7  &  0.15  &  21.0  &  0.26 \\
  25.6  &  0.16  &  20.8  &  0.3 \\
  25.5  &  0.205  &  20.6  &  0.3 \\
  25.4  &  0.225  &  20.4  &  0.26 \\
  25.2  &  0.265  &  20.2  &  0.2 \\
  25.1  &  0.27  &  20.0  &  0.13 \\
  25.0  &  0.28  &  19.5  &  0.028 \\
  24.8  &  0.24  &  19.0  &  0.0125 \\
  24.7  &  0.22  &  18.5  &  0.016 \\
  24.6  &  0.2  &  18.0  &  0.025 \\
  24.5  &  0.157  &  18.0  &  0.034 \\
  24.4  &  0.155  &  17.8  &  0.043 \\
  24.3  &  0.15  &  17.6  &  0.048 \\
  24.2  &  0.145  &  17.4  &  0.047 \\
  24.1  &  0.16  &  17.2  &  0.041 \\
  24.0  &  0.2  &  17.0  &  0.031 \\
  23.8  &  0.32  &  16.8  &  0.021 \\
  23.6  &  0.52  &  16.6  &  0.014 \\
  23.4  &  0.72  &  16.4  &  0.0106 \\
  23.2  &  0.9  &  16.2  &  0.01 \\
  \bottomrule
\end{tabular}
\end{table}
\FloatBarrier

Mit diesen Werten und der Gleichung \eqref{??} wird die Regression durchgeführt.
Das Ergebnis ist in Abbildung \ref{fig:plot2} zu sehen.

\begin{figure}
  \centering
  \includegraphics{plot2.pdf}
  \caption{Beugungsmuster des ersten Doppelspalts.}
  \label{fig:plot2}
\end{figure}
\FloatBarrier

Daraus ergeben sich folgende Werte:
\begin{align*}
  \zeta_0 &= (22.91 \pm 0.02) \text{mm} \\
  b &= (0.155 \pm 0.004) \text{mm} \\
  d &= (0.245 \pm 0.004) \text{mm} \\
  A_0 &= (3.36 \pm 0.08) \text{nA}
\end{align*}

Die Hersteller Angaben hatten folgende Werte:
\begin{align*}
  b_{Her} &= 0.15 \text{mm} \\
  d_{Her} &= 0.25 \text{mm} \\
  \Rightarrow \Delta b &= \frac{b - b_{Her}}{b_{Her}} = 3.33\% \\
  \Rightarrow \Delta d &= \frac{d - d_{Her}}{d_{Her}} = 2\%
\end{align*}

Die gemessenen Daten des zweiten Doppelspalts befinden sich in Tabelle \ref{tab:doppel2}, wie zurvor vom Dunkelstrom bereinigt.

\begin{table}
  \centering
  \caption{Doppelspalt 2 Messwerte.}
  \label{tab:doppel2}
\begin{tabular}{c c | c c}
  \toprule
  $\zeta$ in mm & $I$ in nA & $\zeta$ in mm & $I$ in nA \\
  \midrule
  19.25  &  0.024  &  22.7  &  0.56 \\
  19.4  &  0.035  &  22.85  &  0.518 \\
  19.55  &  0.054  &  23.0  &  0.407 \\
  19.7  &  0.085  &  23.15  &  0.28 \\
  19.85  &  0.12  &  23.3  &  0.2 \\
  20.0  &  0.15  &  23.45  &  0.191 \\
  20.15  &  0.16  &  23.6  &  0.245 \\
  20.3  &  0.149  &  23.75  &  0.318 \\
  20.45  &  0.125  &  23.9  &  0.363 \\
  20.6  &  0.107  &  24.05  &  0.359 \\
  20.75  &  0.123  &  24.2  &  0.298 \\
  20.9  &  0.19  &  24.35  &  0.212 \\
  21.05  &  0.29  &  24.5  &  0.13 \\
  21.2  &  0.35  &  24.65  &  0.085 \\
  21.35  &  0.421  &  24.8  &  0.075 \\
  21.5  &  0.402  &  24.95  &  0.095 \\
  21.65  &  0.318  &  25.1  &  0.115 \\
  21.8  &  0.23  &  25.25  &  0.121 \\
  21.95  &  0.183  &  25.4  &  0.108 \\
  22.1  &  0.22  &  25.55  &  0.082 \\
  22.25  &  0.32  &  25.7  &  0.07 \\
  22.4  &  0.442  &  25.85  &  0.055 \\
  22.55  &  0.515  &  26.0  &  0.035 \\
  \bottomrule
\end{tabular}
\end{table}
\FloatBarrier

Mit diesen Werten und der Gleichung \eqref{??} wird die Regression durchgeführt.
Das Ergebnis ist in Abbildung \ref{fig:plot3} zu sehen.

\begin{figure}
  \centering
  \includegraphics{plot3.pdf}
  \caption{Beugungsmuster des zweiten Doppelspalts.}
  \label{fig:plot3}
\end{figure}
\FloatBarrier

Daraus ergeben sich folgende Werte:
\begin{align*}
  \zeta_0 &= (22.68 \pm 0.02) \text{mm} \\
  b &= (0.146 \pm 0.010) \text{mm} \\
  d &= (0.465 \pm 0.008) \text{mm} \\
  A_0 &= (2.69 \pm 0.15) \text{nA}
\end{align*}

Die Hersteller Angaben hatten folgende Werte:
\begin{align*}
  b_{Her} &= 0.15 \text{mm} \\
  d_{Her} &= 0.35 \text{mm} \\
  \Rightarrow \Delta b &= \frac{b - b_{Her}}{b_{Her}} = 2.67\% \\
  \Rightarrow \Delta d &= \frac{d - d_{Her}}{d_{Her}} = 32.86\%
\end{align*}

In Abbildung \ref{plot4} wird jetzt noch der Einzelspalt mit dem ersten Doppelspalt gezeigt.
Dazu sind die Hauptmaxima auf den selben x-Wert geschoben und die Amplitude angepasst worden, damit die Abbildungen vergleichbar werden.

\begin{figure}
  \centering
  \includegraphics{plot4.pdf}
  \caption{Doppel- und Einzelspalt im Vergleich.}
  \label{fig:plot4}
\end{figure}
\FloatBarrier