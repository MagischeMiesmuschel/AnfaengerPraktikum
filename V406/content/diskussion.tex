\section{Diskussion}
\label{sec:Diskussion}

Zur Bewertung der berechneten Werte werden die die Fit-Parameter mit den Herstellerangaben über die Fehler verglichen.
Beim Einzelspalt weicht die Spaltbreite um 18.4\% vom Herstellerwert ab und befindet sich im vierzehnten Fehlerintervall.
Aber der Fehler des Fit-Parameters ist vergleichsweise klein mit nur $\pm 0.0021$mm greift er nur auf die vom Hersteller nicht mehr gegebene Stelle nach dem Komma.
Beim ersten Doppelspalt ist der Fehler ca. doppelt so groß mit $\pm 0.004$, aber der Herstellerwert befindet sich noch innerhalb des zweiten Fehlerintervalls und auch dieser bezieht sich auf die nicht mehr vom Hersteller gegebene Stele nach dem Komma.
Das Gleiche gilt für den Spaltabstand mit dem selben Fehler. 
Auch hier befindet sich der Herstellerwert innerhalb des zweiten Fehlerintervalls.
Dazu passend sind auch die relativen Abweichungen nur sehr gering mit $\text{\Delta}b = 3.33\%$ und $\text{\Delta}d = 2\%$.
Der zweite Doppelspalt weißt bei der Spaltbreite einen größeren Fehler ($\pm 0.01$) auf, welcher sich schon auf die zweite signifikante Stelle bezieht.
Jedoch ist hier der Herstellerwert sogar im ersten Fehlerintervall. 
Aber weil dieser Fehler mehr als doppelt so groß ist wie der vom vorherigen Doppelspalt sind die Abweichungen sehr ähnlich.
Dies bestätigt auch der relative Fehler von $\text{\Delta}b = 2.67\%$.
Bei dem Spaltabstand ist eine wesentlich höhere Abweichung festzustellen.
Somit befindet sich der Herstellerwert nur noch im fünfzehnten Fehlerintervall mit einem relativen Fehler von $\text{\Delta}d = 32.86\%$.\\

Es ist auffällig, dass die Bestimmung der Parameter mit geometrisch zunehmender Größer immer ungenauer wird.
Gründe dafür sind, dass das Beugungsmuster immer schwächer wird und das Auflösungsvermögen des Detektors. 
Durch seinen nur endlich breiten Spalt kann er die Maxima nur bedingt gut aufnehmen, wenn sie bei breiteren Beugungsspalten enger zusammen liegen.\\

Bei einem Vergleich des Einzelspalts zu einem der Doppelspalte ist zu sehen, dass der Einzelspalt grob wie die Einhüllende des Doppelspalts wirkt.
Doch leichte Unterschiede sind festzu stellen, auch schon, weil beim bestimmen der Spaltbreite leicht unterschiedliche Größen ermittelt wurden.
Besonders ist dies in den Nebenmaxima zu sehen, wobei das Hauptmaxima noch sehr gut als Einhüllende fungiert.