\section{Durchführung}
\label{sec:Durchführung}

Bevor Messungen für die Halbwertszeiten von In und Ag durchgeführt werden, wird eine Messung ohne Probe vorgenommen.
Diese dient der Bestimmung des Nulleffekts, der durch die Höhenstrahlung und natürliche Zerfälle, wie zum Beispiel von $\ce{^{40}_19K}$ erzeugt wird.
Um den statistischen Fehler so gering wie möglich zu halten, wird die Zählrate über 900 Sekunden gemessen.
Für Indium wird die Zählrate alle 60 Sekunden gemessen und das über den Zeitraum 1 Stunde.
Die Messung für Silber dauert 7 Minuten und die Zählrate alle 10 Sekunden gemessen.
