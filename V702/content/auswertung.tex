\section{Auswertung}
\label{sec:Auswertung}

\subsection{Nullrate}
\label{sec:null}
Vor den Messungen muss die Nullrate bestimmt werden.
Es wird über einen relativ großen Zeitraum gemessen, um den statistische Fehler gering zu halten.
Bei einer Messzeit von $t = 900s$ wird eine Zählrate von $N = 178$ gemessen.
Dies macht eine Nullrate von $N_u \approx 2$ pro 10 Sekunden.
Diese wird vor jeder folgenden Rechnung von der Zählrate abgezogen.

\subsection{Indium}
\label{sec:indium}

Durch die Messung über eine Zeit von 60 Minuten mit Zeitintervallen von $\Delta t = 240s$, bei konstanter Spannung, werden folgende Messwerte für die Zählrate von Indium abzüglich der Nullrate ermittelt:

\begin{table}
  \centering
  \caption{Messwerte Indium.}
  \label{tab:N1}
\begin{tabular}{c c c}
  \toprule
  Messzeit $t$ in s & Zählrate $N$ & $\Omega_N = \sqrt{N}$-Fehler\\
  \midrule
  240.0 & 2731.53 & 52.26  \\
  480.0 & 2439.53 & 49.39  \\
  720.0 & 2346.53 & 48.44  \\
  960.0 & 2281.53 & 47.76  \\
  1200.0 & 2201.53 & 46.92  \\
  1440.0 & 1967.53 & 44.35  \\
  1680.0 & 1938.53 & 44.02  \\
  1920.0 & 1826.53 & 42.73  \\
  2160.0 & 1664.53 & 40.79  \\
  2400.0 & 1596.53 & 39.95  \\
  2640.0 & 1582.53 & 39.78  \\
  2880.0 & 1471.53 & 38.36  \\
  3120.0 & 1391.53 & 37.30  \\
  3360.0 & 1342.53 & 36.64  \\
  3600.0 & 1324.53 & 36.39  \\
  \bottomrule
\end{tabular}
\end{table}
\FloatBarrier

Für die Abbildung \ref{fig:plot1} werden die logarithmischen Werte der Zählrate, sowie deren Fehler benötigt.
In Tabelle \ref{tab:lnN1} sind diese Werte zu sehen.

\begin{table}
  \centering
  \caption{Logarithmische Messwerte Indium.}
  \label{tab:lnN1}
\begin{tabular}{c c c c}
  \toprule
  Messzeit $t$ in s & Zählrate ln($N$) & ln($N + \Omega_N$)- ln($N$) & ln($N$) - ln($N - \Omega_N$)\\
  \midrule
  240.0 & 7.91 & 0.0189 & 0.0193 \\
  480.0 & 7.79 & 0.0200 & 0.0204 \\
  720.0 & 7.76 & 0.0204 & 0.0208 \\
  960.0 & 7.73 & 0.0207 & 0.0211 \\
  1200.0 & 7.60 & 0.0210 & 0.0215 \\
  1440.0 & 7.58 & 0.0222 & 0.0228 \\
  1680.0 & 7.56 & 0.0224 & 0.0229 \\
  1920.0 & 7.51 & 0.0231 & 0.0236 \\
  2160.0 & 7.41 & 0.0242 & 0.0248 \\
  2400.0 & 7.37 & 0.0247 & 0.0253 \\
  2640.0 & 7.36 & 0.0248 & 0.0254 \\
  2880.0 & 7.29 & 0.0257 & 0.0264 \\
  3120.0 & 7.23 & 0.0264 & 0.0271 \\
  3360.0 & 7.20 & 0.0269 & 0.0276 \\
  3600.0 & 7.18 & 0.0271 & 0.0278 \\
  \bottomrule
\end{tabular}
\end{table}
\FloatBarrier

\begin{figure}
  \centering
  \includegraphics[width=4cm]{plot1.pdf}
  \caption{Halblogarithmische Zerfallskurve für Indium.}
  \label{fig:plot1}
\end{figure}
\FloatBarrier

Die Ausgleichsgerade wurde mit der Funktion
\begin{equation}
  f(x) = ax +b
  \label{eqn:glFit}
\end{equation}
berechnet.
Die Parameter $a$ und $b$ wurden berechnet zu 
\begin{align*}
  a &= 0,000217 \pm (8,2 \cdot 10^{-8}) \\
  b &= 7,926 \pm (2 \cdot 10^{-4})
\end{align*}
Hier ist $a$ die Zerfallskonstante $\lambda$ und $b$ ist $\text{ln}(N_0(1-e^{-\lambda \Delta t}))$. \\

\subsubsection{Halbwertszeit}
\label{sec:IH}

Mit $\lambda$ kann nun die Halbwertszeit berechnet werden.
Durch einsetzen in die Formel 
\begin{equation}
  T = \frac{\text{ln}2}{\lambda}
  \label{eqn:glTot}
\end{equation}
ergibt sich für die Halbwertszeit von Indium 
\begin{align*}
  T = 3195,28 \pm 1,20 \text{s}
\end{align*}
Dies entspricht einer Zeit von $t = 53$ Minuten und 15 Sekunden, mit gleicher Unsicherheit.
In der Theorie wird ein Wert von $T = 3257,4 s$ \cite{Periode} angegeben.
Daraus ergibt sich eine prozentuale Abweichung von ca 1,9 \% vom Theoriewert.
Wobei der Fehler $\Delta T = 1,20$ zu $T$ mit folgender Formel berechnet wurde:
\begin{align*}
  \Delta T &= \sqrt{\left(\frac{\delta T}{\delta \lambda}\right)^2 \cdot \left(\Delta \lambda\right)^2} \\
  \Delta T &= \frac{\text{ln(2)}}{\lambda^2} \cdot \Delta \lambda \\
  \Delta T &= 1,20
\end{align*}

\subsection{Silber}
\label{sec:silber}

Durch die Messung über eine Zeit von 7 Minuten mit Zeitintervallen von $\Delta t = 10s$, bei konstanter Spannung, erhält man folgende Messwerte für die Zählrate von Silber abzüglich der Nullrate:

\begin{table}
  \centering
  \caption{Messwerte Silber.}
  \label{tab:N2}
\begin{tabular}{c c c}
  \toprule
  Messzeit $t$ in s & Zählrate $N$ & $\Omega_N = \sqrt{N}$-Fehler\\
  \midrule
  10.0 & 236.02 & 15.36 \\
  20.0 & 168.02 & 12.96 \\
  30.0 & 121.02 & 11.00 \\
  40.0 & 104.02 & 10.19 \\
  50.0 & 88.02 & 9.38  \\
  60.0 & 65.02 & 8.06  \\
  70.0 & 55.02 & 7.41  \\
  80.0 & 32.02 & 5.65  \\
  90.0 & 36.02 & 6.00  \\
  100.0 & 35.02 & 5.91  \\
  110.0 & 25.02 & 5.00  \\
  120.0 & 32.02 & 5.65  \\
  130.0 & 23.02 & 4.79  \\
  140.0 & 24.02 & 4.90  \\
  150.0 & 16.02 & 4.00 \\
  160.0 & 19.02 & 4.36  \\
  170.0 & 18.02 & 4.24  \\
  180.0 & 10.02 & 3.16  \\
  190.0 & 14.02 & 3.74  \\
  200.0 & 14.02 & 3.74  \\
  210.0 & 15.02 & 3.87  \\
  220.0 & 9.02 & 3.00  \\
  230.0 & 18.02 & 4.24  \\
  240.0 & 11.02 & 3.31  \\
  250.0 & 11.02 & 3.31  \\
  260.0 & 11.02 & 3.31  \\
  270.0 & 4.02 & 2.00  \\
  280.0 & 9.02 & 3.00  \\
  290.0 & 12.02 & 3.00  \\
  300.0 & 10.02 & 3.16  \\
  310.0 & 8.02 & 2.83  \\
  320.0 & 7.02 & 2.64  \\
  330.0 & 4.02 & 2.00  \\
  340.0 & 9.02 & 3.00  \\
  350.0 & 4.02 & 2.00  \\
  360.0 & 5.02 & 2.24  \\
  370.0 & 5.02 & 2.24  \\
  380.0 & 7.02 & 2.64  \\
  390.0 & 6.02 & 2.45  \\
  400.0 & 4.02 & 2.00  \\
  410.0 & 8.02 & 2.83  \\
  420.0 & 3.02 & 1.73  \\
  \bottomrule
\end{tabular}
\end{table}
\FloatBarrier

Für die Abbildung \ref{fig:plot2} wurden die logarithmischen Werte der Zählrate, sowie deren Fehler benötigt.
In Tabelle \ref{tab:lnN2} sind diese Werte zu sehen.

\begin{table}
  \centering
  \caption{Logarithmische Messwerte Silber.}
  \label{tab:lnN2}
\begin{tabular}{c c c c}
  \toprule
  Messzeit $t$ in s & Zählrate ln($N$) & ln($N + \Omega_N$)- ln($N$) & ln($N$) - ln($N - \Omega_N$)\\
  \midrule
  10.0 & 5.46 & 0.06 & 0.06 \\
  20.0 & 5.12 & 0.07 & 0.08 \\
  30.0 & 4.79 & 0.08 & 0.09 \\
  40.0 & 4.64 & 0.09 & 0.10 \\
  50.0 & 4.47 & 0.10 & 0.11 \\
  60.0 & 4.17 & 0.11 & 0.13 \\
  70.0 & 4.00 & 0.12 & 0.14 \\
  80.0 & 3.46 & 0.16 & 0.19 \\
  90.0 & 3.58 & 0.15 & 0.18 \\
  100.0 & 3.55 & 0.15 & 0.18 \\
  110.0 & 3.21 & 0.18 & 0.22 \\
  120.0 & 3.46 & 0.16 & 0.19 \\
  130.0 & 3.13& 0.18 & 0.23 \\
  140.0 & 3.17 & 0.18 & 0.22 \\
  150.0 & 2.77 & 0.22 & 0.28 \\
  160.0 & 2.94 & 0.20 & 0.26 \\
  170.0 & 2.89 & 0.21 & 0.26 \\
  180.0 & 2.30 & 0.27 & 0.37 \\
  190.0 & 2.64 & 0.23 & 0.31 \\
  200.0 & 2.64 & 0.23 & 0.31 \\
  210.0 & 2.70 & 0.22 & 0.29 \\
  220.0 & 2.19 & 0.28 & 0.40 \\
  230.0 & 2.89 & 0.21 & 0.26 \\
  240.0 & 2.39 & 0.26 & 0.35 \\
  250.0 & 2.39 & 0.26 & 0.35 \\
  260.0 & 2.39 & 0.26 & 0.35 \\
  270.0 & 1.39 & 0.40 & 0.69 \\
  280.0 & 2.19 & 0.28 & 0.40 \\
  290.0 & 2.48 & 0.25 & 0.34 \\
  300.0 & 2.30 & 0.27 & 0.37 \\
  310.0 & 2.08 & 0.30 & 0.43 \\
  320.0 & 1.94 & 0.32 & 0.47 \\
  330.0 & 1.39 & 0.40 & 0.69 \\
  340.0 & 2.19 & 0.28 & 0.40 \\
  350.0 & 1.39 & 0.40 & 0.69 \\
  360.0 & 1.61 & 0.36 & 0.59 \\
  370.0 & 1.61 & 0.36 & 0.59 \\
  380.0 & 1.94 & 0.32 & 0.47 \\
  390.0 & 1.79 & 0.34 & 0.52 \\
  400.0 & 1.39 & 0.40 & 0.69 \\
  410.0 & 2.08 & 0.30 & 0.43 \\
  420.0 & 1.10 & 0.45 & 0.85 \\
  \bottomrule
\end{tabular}
\end{table}
\FloatBarrier

\begin{figure}
  \centering
  \includegraphics{plot3.pdf}
  \caption{Halblogarithmische Zerfallskurve für Silber.}
  \label{fig:plot2}
\end{figure}
\FloatBarrier

\subsubsection{$^{107}$Ag}

Weil Silber aus den zwei Isotopen $^{107}\text{Ag}$ und $^{109}\text{Ag}$, mit verschiedenen Halbwertszeiten besteht, muss vor der Ausgleichrechnung eine Trennung der beiden Bereiche vorgenommen werden.
Links vom Punkt $t^*$ sind noch beide Zerfälle in signifikanter Größe vorhanden und rechts nur noch der Zerfall des langlebigen Isotops $^{107}\text{Ag}$ mit einer Halbwertszeit von $T = 142,2 s$ in der Theorie \cite{Periode}.
$t^*$ wird hier durch beobachten des Kurvenverlaufs auf 100s gesetzt.
Zur weiteren Auswertung muss erst nur der rechte Bereich mit etwas Abstand zum Punkt $t^*$ mit einer Ausgleichsgeraden in Abbildung \ref{fig:plot3} berechnet werden.
Dazu wird die Funktion \ref{eqn:glFit} genutzt.
In dem Zeitbereich ab 120 s ergeben sich die Werte
\begin{align*}
  a &=  0.005681 \pm (1,08 \cdot 10^{-4}) \\
  b &= 3.790 \pm 0.032
\end{align*}
Auch hier ist $a$ gleich $\lambda$ und $b$ ist $\text{ln}(N_0(1-e^{-\lambda \Delta t}))$.

\subsubsection{Halbwertszeit $^{107}$Ag}

Mit $\lambda$ kann nun die Halbwertszeit berechnet werden.
Durch einsetzen in die Formel \ref{eqn:glTot} ergibt sich für die Halbwertszeit von $^{107}\text{Ag}$ 
\begin{align*}
  T = 122,02 \pm 2,30 \text{s}
\end{align*}
Dies entspricht ungefähr einer Zeit von $t = 2$ Minuten und 2 Sekunden, mit gleicher Unsicherheit.
In der Theorie wird ein Wert von $T = 142,2 s$ \cite{Periode} angegeben.
Daraus ergibt sich eine prozentuale Abweichung von ca 14,2 \% vom Theoriewert.

\begin{figure}
  \centering
  \includegraphics{plot5.pdf}
  \caption{Langlebiger Anteil des Silbers $^{108}\text{Ag}$.}
  \label{fig:plot3}
\end{figure}
\FloatBarrier

\subsubsection{$^{109}$Ag}

Zur Auswertung des kurzlebigen Isotops $^{109}\text{Ag}$, mit einer Halbwertszeit von $T = 24,6 s$ in der Theorie \cite{Periode}, müssen die Werte 

\begin{align*}
  N_{\Delta t} := N_0(1-e^{\lambda \Delta t})e^{-\lambda t}
\end{align*}

von den Zählraten im Bereich $t < t^*$ abgezogen werden.
Es wird der Bereich bis 70 s gewählt.
Die neuen Zählraten des kurzlebigen Isotops sind in Tabelle \ref{tab:Nk} zu sehen. 

\begin{table}
  \centering
  \caption{Messwerte $^{109}$Ag.}
  \label{tab:Nk}
\begin{tabular}{c c c}
  \toprule
  Messzeit $t$ in s & Zählrate $N$ & $\Omega_N = \sqrt{N}$-Fehler\\
  \midrule
  10.0 & 194.2 & 13.9\\
  20.0 & 128.5 & 11.3 \\
  30.0 & 83.7 & 9.1 \\
  40.0 & 68.7 & 8.2 \\
  50.0 & 54.7 & 7.3 \\
  60.0 & 33.5 & 5.7 \\
  70.0 & 25.2 & 5.0 \\
  \bottomrule
\end{tabular}
\end{table}
\FloatBarrier

Für die Abbildung \ref{fig:plot4} wurden die logarithmischen Werte der Zählrate, sowie deren Fehler benötigt.
In Tabelle \ref{tab:lnN2k} sind diese Werte zu sehen.

\begin{table}
  \centering
  \caption{Logarithmische Messwerte Silber.}
  \label{tab:lnN2k}
\begin{tabular}{c c c c}
  \toprule
  Messzeit $t$ in s & Zählrate ln($N$) & ln($N + \Omega_N$)- ln($N$) & ln($N$) - ln($N - \Omega_N$)\\
  \midrule
  10.0 & 5.4 & 0.06 & 0.07 \\
  20.0 & 5.1 & 0.08 & 0.09 \\
  30.0 & 4.7 & 0.10 & 0.11 \\
  40.0 & 4.6 & 0.11 & 0.12 \\
  50.0 & 4.4 & 0.12 & 0.14 \\
  60.0 & 4.1 & 0.15 & 0.18 \\
  70.0 & 4 & 0.18 & 0.22 \\
  \bottomrule
\end{tabular}
\end{table}
\FloatBarrier

\begin{figure}
  \centering
  \includegraphics{plot4.pdf}
  \caption{Kurzlebiger Anteil des Silbers $^{110}\text{Ag}$.}
  \label{fig:plot4}
\end{figure}
\FloatBarrier

Hier wird für die Ausgleichsgerade wieder die Formel \ref{eqn:glFit} verwendet, mit den Parametern 
\begin{align*}
  a &= 0,032949 \pm (9,7 \cdot 10^{-5}) \\
  b &= 5,54 \pm (4 \cdot 10^{-3})
\end{align*}
Auch hier ist $a$ gleich $\lambda$ und $b$ ist $\text{ln}(N_0(1-e^{-\lambda \Delta t}))$.

\subsubsection{Halbwertszeit $^{109}$Ag}

Mit $\lambda$ kann nun die Halbwertszeit berechnet werden.
Durch einsetzen in die Formel \ref{eqn:glTot} ergibt sich für die Halbwertszeit von $^{109}\text{Ag}$ zu 
\begin{align*}
  T = 21,04 \pm 0,06 \text{s}
\end{align*}
In der Theorie wird ein Wert von $T = 24,6 s$ \cite{Periode} angegeben.
Daraus ergibt sich eine prozentuale Abweichung von ca 14,5 \% vom Theoriewert.

Mit den nun beiden bekannten Halbwertszeiten und Faktoren, der kurz- und langlebigen Isotope, kann die Funktion in Abbildung \ref{fig:plot2} für die kombinierten Bereiche dargestellt werden.
Sie hat die Gestallt:
\begin{equation}
  e^{b_{kurz}} \cdot e^{-\lambda_{kurz} t} + e^{b_{lang}} \cdot e^{-\lambda_{lang} t}
\end{equation}
Für die Abbildung \ref{fig:plot2} wurde nun nur noch der Logarithmus gebildet.