\section{Diskussion}
\label{sec:Diskussion}
Durch die hohe Messzeit am Anfang, bei der Bestimmung der Nullrate ist der statistische Fehler sehr klein gehalten.
Somit ist dieses Ergebnis mit einer hohen Wahrscheinlichkeit als annähernd genau anzunehmen.
In erster Linie folgen alle Messungen dem Zerfallsgesetz und lassen sich durch logarithmische Gerade ausdrücken.
Dies bestätigt die Funktionalität des Zählrohrs und der Präparate.
Im ersten Versuchsteil mit Indium (Abschnitt \ref{sec:indium}) ist eine Halbwertszeit mit lediglich 1,9\% Abweichung vom Theoriewert berechnet worden.
Diese Abweichung ist klein genug, um von einem guten Messergebnis zu sprechen.
Grund dafür ist die lange Halbwertszeit des Indiums und die entsprechend lange Messreihe.
Im Versuchsteil mit Silber sind höhere Abweichungen aufgetreten, mit jeweils ca 14\% vom Theoriewert.
Die berechneten Werte sind jedoch immernoch in der selben Größenordnung und nah genug am Theoriewert, um von einem funktioniereden Versuch zu reden.
Grund für die Abweichungen sind die kurze Halbwertszeit vom kurzlebigen $^{109}$A.
Bis die Probe in der Haltrung zur Messung angebracht war, sind schon 5-10 Sekunden vergangen.
Dies sind 25-50\% der Halbwertszeit.
Somit wurde das genaue Bestimmen der Werte erschwert.