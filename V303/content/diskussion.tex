\section{Diskussion}
\label{sec:Diskussion}
Die Abweichung der Signalspannung ist durch Eigeninduktion der verbauten Spulen, oder Ungenauigkeit der mechanischen Drehknöpfe am Lock-In-Verstärker zu vermuten.
Die leichte Abweichung, beim überprüfen der Rauschunterdrückung über eine Photodiode,
lässt sich durch einen komplett abgedunkelten Raum optimieren.
Durch die ständig ändernden Lichteinfälle im Raum lassen sich Schwankungen durch einen Lichteinfall in die Photodiode erklären.
Jedoch sind die Abweichung so gering, dass eine Verifizierung der korekten Funktionsweise dieses Lock-In-Verstärkers nur sinnvoll ist.

\newpage
%Berichtigung
Der für Abb. 1 genutzte Fit wurde mit der Funktion 
\begin{align}
    f(x) = a \cdot cos(x + b) +c
\end{align}
geplottet. Die Berechneten Parameter a, b und c lauten:
\begin{align}
    a &= 2,096 \pm 0.040 \\
    b &= 0,168 \pm 0.020 \\
    c &= 0.093 \pm 0.029
\end{align}
