\section{Theorie}
\label{sec:Theorie}
In einer evakuierten Röhre werden aus einer Glühkatode emittierte Elektronen auf eine Anode hin beschleunigt.
Beim Auftreffen auf die Anode entsthet Röntgenstrahlung , die sich aus dem kontinuierlichen Bremsspektrum
und der charakteristischen Röntgenstrahlung des Anodenmaterials zusammensetzt.
Durch die Abbremsung eines Elektrons im Coulombfeld des Atomkerns wird ein Photon,
dessen Energie dem Energieverlust des abgebremsten Elektrons entspricht,
ausgesendet.
Das Bremsspektrum ist kontnuierlich, weil das Elektron sowhl einen Teil, als auch seine gesamte kinetische Energie abgeben kann.
Die maximale Energie bzw. minimale Wellenlänge
\begin{equation}
    \lambda = \frac{h \cdot c}{e_0 U}
    \label{eqn:gl1}
\end{equation}
ergibt sich bei der vollständigen Abbremsung des Elektrons.
Hierbei wird die gesamte kinetische Energie $E_{\text{kin}} = e_0 U$ in Strahlungsenergie $E = h \cdot \nu$ umgewandelt.\\
\noindent
Durch das ioniesieren des Anodenmaterials, so dass eine Leerstelle in einer inneren Schale entsteht, kann ein Elektron einer äußeren Schale unter Aussendung eines Röntgenquants in die innere Schale zurückfallen.
Das charakteristische Spektrum besteht aus scharfen Linien, weil beim zurückfallen genau die Energiedifferenz zwischen den beiden Energieniveaus der beiden Schalen $h \nu = E_{\text{m}} - E_{\text{n}}$ freigesetzt wird.
Diese Diferenz ist charakteristisch für bestimmte Anodenmaterialien.