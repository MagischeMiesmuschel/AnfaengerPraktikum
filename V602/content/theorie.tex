\section{Theorie}
\label{sec:Theorie}
In einer evakuierten Röhre werden aus einer Glühkatode emittierte Elektronen auf eine Anode hin beschleunigt.
Beim Auftreffen auf die Anode entsthet Röntgenstrahlung , die sich aus dem kontinuierlichen Bremsspektrum
und der charakteristischen Röntgenstrahlung des Anodenmaterials zusammensetzt.
Durch die Abbremsung eines Elektrons im Coulombfeld des Atomkerns wird ein Photon,
dessen Energie dem Energieverlust des abgebremsten Elektrons entspricht,
ausgesendet.
Das Bremsspektrum ist kontnuierlich, weil das Elektron sowhl einen Teil, als auch seine gesamte kinetische Energie abgeben kann.
Die maximale Energie bzw. minimale Wellenlänge
\begin{equation}
    \lambda = \frac{h \cdot c}{e_0 U}
    \label{eqn:gl1}
\end{equation}
ergibt sich bei der vollständigen Abbremsung des Elektrons.
\begin{figure}
    \centering
    \includegraphics[height=4.0cm]{data/abb1.jpg}
    \caption{Bremsspektrum \cite{V602}}
    \label{fig:abb1}
\end{figure}
Hierbei wird die gesamte kinetische Energie $E_{\text{kin}} = e_0 U$ in Strahlungsenergie $E = h \cdot \nu$ umgewandelt.\\
\noindent
Durch das ioniesieren des Anodenmaterials, so dass eine Leerstelle in einer inneren Schale entsteht, kann ein Elektron einer äußeren Schale unter Aussendung eines Röntgenquants in die innere Schale zurückfallen.
Das charakteristische Spektrum besteht aus scharfen Linien, weil beim zurückfallen genau die Energiedifferenz zwischen den beiden Energieniveaus der beiden Schalen $h \nu = E_{\text{m}} - E_{\text{n}}$ freigesetzt wird.
Diese Diferenz ist charakteristisch für verschiedene Anodenmaterialien.
Die einzelnen Linien werden mit $\text{K}_{\alpha}, \text{K}_{\beta}, \text{L}_{\alpha},...$ bezeichnet, wobei die Buchstaben K,L,M,... die Schale bezeichnen, auf der die Übergänge enden.
Dem griechischem Buchstaben kann man entnehmen, woher das beteiligte äußere Elektron stammt.
In einem Mehrelektronenatom schirmen die Hüllenelektronen und die Wechselwirkung der Elektronen untereinander die Kernladung ab.
Dies führt zu einer Verringerung der Coulomb Anziehung auf das äußere Elektron, sodass für die Bindungsenergie $E_\text{n}$ eines Elektrons auf der n-ten Schale gilt:
\begin{equation}
    E_\text{n} = -R_{\infty}z_{\text{eff}}^2 \cdot \frac{1}{n^2}
    \label{eqn:gl2}
\end{equation}
Der Abschirmeffekt wird durch die effektive Kernladung $z_{\text{eff}} = z - \sigma$ berücksichtigt. 
\sigma ist die Abschirmkonstante und $R_{\infty} = 13,6 \text{eV}$ ist die Rydbergenergie.
Da die äußeren Elektronen aufgrund des Bahndrehimpulses und des Elektronenspins nicht alle dieselbe Bindungsenergie besitzen, ist in der Regel jede charakteristische Linie in eine Reihe von eng beieinander liegenden Linien aufgelöst (Feinstruktur).
Diese können in dem vorliegenden Versuchsaufbau nicht aufgespalten werden.
Bei der im Versuch verwendeten Kupferanode können die $\text{Cu-K}_{\alpha}\text{- und die Cu-K}_{\beta}\text{-Linien}$ beobachtet werden, die der Bremsstrahlung überlagert sind.\\
\noindent
Bei der Absorption von Röntgenstrahlung unter 1 MeV sind der Photoeffekt und der Comptoneffekt die dominanten Prozesse.
Mit zunehmender Energie nimmt der Absorptionskoeffizient ab und steigt sprunghaft an, wenn die Photoenergie gerade größer ist als die Bindungsenergie eines Elektrons aus der nächst inneren Schale.
Die Lage der Absorptionskanten $h \nu_{\text{abs}} = E_{\text{n}} - E_{\infty}$ ist nahezu identisch mit der Bindungsenergie.
Je nachdem aus welches Schale das Elektron stammt, wird die zugehörige Absorptionskante K-, L-, M-,... Absorptionskante bezeichnet.
Aufgrund der Feinstruckturen werden drei L-Kanten beobachtet, aber nur eine K-Kante.
\begin{figure}
    \centering
    \includegraphics[height=4.0cm]{data/abb2.jpg}
    \caption{Bremsspektrum \cite{V602}}
    \label{fig:abb1}
\end{figure}
Die Bindungsenergie $E_{n,j}$ eines Elektrons muss aufgrund der Feinstruckturen mit der Sommerfeldschen Feinstruckturformel berechnet werden.
\begin{equation}
    E_{n,j} = -R_{\infty}\left(z_{\text{eff}}^2 \cdot \frac{1}{n^2} + \alpha^2 z_{\text{eff,2}}^4 \cdot \frac{1}{n^3} \left( \frac{1}{j + \frac{1}{2}} - \frac{3}{4n}\right)\right)
    \label{eqn:gl3}
\end{equation}
n ist die Hauptquantenzahl, \alpha die Sommerfeldsche Feinstruckturkonstante und j der Gesamtdrehimpuls.
Bei der Bestimmung der Abschirmkonstante ${\sigma}_L$ aus der L-Kante müssen die Abschirmzahlen jedes beteiligten Elektrons berücksichtigt werden.
Die Rechnung wird vereinfacht durch bestimmung der Energiedifferenz $\Delta E_{\text{L}}$ zweier L-Kanten.
Im Vorliegenden Versuch können die $\text{L}_1- \text{und L}_2-$Kante nicht aufgelöst werden, so dass sich ${\sigma}_L$
\begin{equation}
    {\sigma}_L = Z - \left(\frac{4}{\alpha}\sqrt{\frac{\Delta E_{\text{L}}}{R_{\infty}}} - \frac{5 \Delta E_{\text{L}}}{R_{\infty}}\right)^{1/2}\left(1 + \frac{19}{32}\alpha^2\frac{\Delta E_{\text{L}}}{R_{\infty}}\right)^{1/2}
    \label{eqn:gl3}
\end{equation}
aus der Energiedifferenz $\Delta E_{\text{L}} = E_{\text{$\text{L}_2$}} - E_{\text{$\text{L}_3$}}$ und der Ordnungszahl $Z$ bestimmen lässt.
Durch die Braggsche Reflexion kann die Energie E bzw. Wellenlänge \lambda experimentall analysiert werden.
\begin{figure}
    \centering
    \includegraphics[height=4.0cm]{data/abb3.jpg}
    \caption{Bremsspektrum \cite{V602}}
    \label{fig:abb1}
\end{figure}
Die Photonen werden an jeden Atom des Gitters gebeugt.
Die Röntgenstrahlen interferieren miteinander und beim Glanzwinkel \Theta erhält man konstruktive Interferenz.
Bei bekannter Gitterkonstante d kann mit Hilfe der Braggschen Bedingung
\begin{equation}
    2 d \sin{\Theta} = n \lambda
\end{equation}
aus dem Winkel \Theta die gebeugt Wellenlänge \lambda bestimmt werden, wobei n die Beugungsordnung ist.