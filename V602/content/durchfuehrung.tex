\section{Durchführung}
\label{sec:Durchführung}
\subsection{Überprüfung der Braggbedingung}
Es wird die Braggbedingung überprüft, indem der LiF-Kristall im Programm auf einen festen Kristallwinkel von $\Theta = 14^\circ$ gestellt wird.
Das Geiger- Müler Zählrohr misst in einem Winkelbereich von ${\alpha}_{\text{GM}} = 26^\circ \text{ bis } {\alpha}_{\text{GM}} = 30^\circ$ mit einem Winkelzuwachs von $\Delta \alpha = 0,1^\circ$ die Intensität der Röntgenstrahlung.
Aus der gemessenen Daten wird das Maximum der Kurve bestimmt und mit dem Sollwinkel verglichen.
\subsection{Das Emissionsspektrum einer Cu-Röntgenröhre}
Es wird im Programm 2:1 Koppelmodus das Röntgenspektrum im Winkelbereich $4^\circ <= \Theta >= 26^\circ$ in $0,2^\circ$ Schritten mit einer Integrationszeit von $\Delta t = 5 \text{s}$ gemessen.
Das ergebnis wird graphisch dargestellt mit den Bezeichnungen für die $\text{K}_{\alpha},\text{K}_{\beta}$ Linie und mit dem Bremsberg.
Aus dem Grenzwinkel wird die minimale Wellenlänge des Bremsspektrums bestimmt und verglichen mit dem zu erwartenden Wert.
Danach wird die Halbwertsbreite der $\text{K}_{\alpha},\text{K}_{\beta}$ Linien berechnet und somit das Auflösungsvermögen der Apparatur bestimmt.
Zuletzt wird die Abschirmkonstante ${\sigma}_\text{K}$ aus der Energiedifferenz der $\text{K}_{\alpha},\text{K}_{\beta}$ Linien berechnet.
\subsection{Das Absoptionsspektrum}
Mit einem Germaniumabsorber vor dem Geiger- Müller Zählrohr wird das Absorbtionsspektrum in $0,1^\circ$ Schritten, bei einer Messzeit von $\Delta t = 20 \text{s}$ pro Winkel in einem geeigneten Winkelbereich gemessen.
Das Ergebnis wird graphisch dargestellt und die Absoptionsenergie der gemessenen K-Kant bestimmt.
Daraus wird die Abschirmzahl ${\sigma}_{\text{K}}$ von Germanium berechnet.
Diese Messung wird für vier witere Absorber mit Kernladungszahlen im Bereich $30 <= Z <= 50$.
Die Energieübergänge und entsprechenden Abschirmzahlen werden bestimmt.
Nach Moseley ist die Absorptionsenergie $E_{\text{K}}$ proportional zu $Z^2$.
Zeichnen Sie aus den erhaltenen Daten ein $\sqrt{E_{\text{K}}} - Z$ Diagramm und bestimmen Sie aus der Steigung die Rydbergkonstante.
Zuletzt wird für einen Absorber $Z >= 70 $ das Absoptionsspektrum in $0,1^\circ$-Schritten mit einer Messzeit von $\Delta t = 20 \text{s}$ pro Winkel gemessen.
Das Ergebnis wird graphisch dargestellt und die Abschirmkonstante ${\sigma}_{\text{L}}$ aus den L-Kanten.