\section{Auswertung}
\label{sec:Auswertung}

Für die Auswertung werden mehrere Materialkonstanten benötigt, die in Tabelle \ref{tab:material} aufgeführt sind.

\begin{table}
  \centering
  \caption{Materialkonstanten der verwendeten Elemente.}
  \label{tab:material}
  \begin{tabular}{c c c c c}
    \toprule
    & $Z$ & $E^\text{Lit}_K$ $[\si{\kilo\eV}]$ & $\Theta^\text{Lit}_K$ $[^\circ]$ & $\sigma^\text{Lit}_K$ \\
    \midrule
    Zn & 30 & 9,65 & 18,6 & 3,56 \\
    Br & 35 & 13,48 & 13,2 & 3,52 \\
    Sr & 38 & 16,12 & 11,0 & 3,58 \\
    Zr & 40 & 18,01 & 9,8 & 3,61 \\
    \bottomrule
  \end{tabular}
\end{table}

\subsection{Braggbedingung}

Die Messung ergibt die in Abbildung \ref{fig:plot1} dargesellte Kurve, aus der sich ein Winkel für das Maximum von $14,2^\circ$ ablesen lässt.
Dieser weicht um weniger als $1^\circ$ vom Sollwinkel ($14^\circ$) ab.
\begin{figure}
  \centering
  \includegraphics{plot1.pdf}
  \caption{Emmissionsspektrum bei festem Kristallwinkel.}
  \label{fig:plot1}
\end{figure}
\FloatBarrier

\subsection{Maximale Energie des Bremsspektrums von Kupfer}

Das kontinuierliche und das charakteristische Spektrum der Cu-Röntgenröhre ist in Abbildung \ref{fig:plot2} zu sehen.
\begin{figure}
  \centering
  \includegraphics{plot2.pdf}
  \caption{Emmissionsspektrum der Cu-Röntgenröhre.}
  \label{fig:plot2}
\end{figure}
Aus dieser Kurve kann der Grenzwinkel abgelesen werden, bei dem keine Strahlung mehr gemessen werden kann.
Aus diesem wird mit folgender Gleichung die maximale Energie berechnet.
\begin{equation}
  E = \frac{h c}{2d \sin(\Theta)}
  \label{eqn:E}
\end{equation}
Für einen Winkel von
\begin{equation*}
  \Theta_\text{min} = \SI{5.4}{\degree}
\end{equation*}
ergibt sich eine maximale Energie von:
\begin{equation*}
  E_\text{max} = \SI{32.646}{\kilo\eV}
\end{equation*}
Dieser Wert liegt leicht unter dem Erwartungswert:
\begin{equation*}
  e_0 U = \SI{35}{\kilo\eV}
\end{equation*}
\FloatBarrier

\subsection{Auflösung der Röntgenröhre}

Aus Abbildung \ref{fig:plot2} lässt sich auch die Auflösung der Röntgenröhre bestimmen.
Dafür werden die Winkel aus der Kurve abgelesen, bei denen die maximale Zählrate auf die Hälfte gefallen ist (Halbwertsbreite):
\begin{align*}
  \Theta_1 &= \SI{19.9}{\degree} \\
  \Theta_2 &= \SI{20.2}{\degree}
\end{align*}
Mit Gleichung \eqref{eqn:E} berechnen sich die entsprechenden Energien:
\begin{align*}
  E_1 = \SI{9.026}{\kilo\eV} \\
  E_2 = \SI{8.897}{\kilo\eV}
\end{align*}
Aus der Differnz dieser beiden Werte ergibt sich die Energieauflösung:
\begin{equation*}
 \increment E = \SI{0.129}{\kilo\eV}
\end{equation*}
Die Genauigkeit dieser Angabe muss als gering eingeschätzt werden, da die Halbwertsbreite nur sehr ungenau aus Abbildung \ref{fig:plot2} und den dazugehörigen Werten abgelesen werden kann.

\subsection{Abschimrkonstanten des Kupfers}

Aus Abbildung \ref{fig:plot2} werden folgende Winkel für die $K_\alpha$- und die $K_\beta$-Linie abgelesen:
\begin{align*}
  \Theta_\alpha &= \SI{22.2}{\degree} \\
  \Theta_\beta &= \SI{20.0}{\degree}
\end{align*}
Aus Gleichung \eqref{eqn:E} folgen die dazugehörigen Energien:
\begin{align*}
  E_\alpha = \SI{8.13}{\kilo\eV} \\
  E_\beta = \SI{8.98}{\kilo\eV}
\end{align*}
Da die Lage der Absorptionskanten $h\nu = E_n - E_\infty$ fast identisch zu der Bindungsenergie des Elektrons ist,
kann die Abschirmunskonstante der $K_\beta$-Linie folgendermaßen berechnet werden:
\begin{equation*}
  \sigma_\beta = z_\text{Cu} - \sqrt{\frac{E_\beta}{R_\infty}} = 3,30
\end{equation*}
Mit der nun bekannten Abschimrkonstanten $\sigma_\beta$ kann die Abschirmkonstante der $K_\alpha$-Linie bestimmt werden:
\begin{equation*}
  \sigma_\alpha = z_\text{Cu} - 2 \cdot \sqrt{\frac{R_\infty (z_\text{Cu}-\sigma_\beta)^2 - E_\alpha}{R_\infty}} = 13,16
\end{equation*}

\subsection{K-Linine des Absorptionsspektrums verschiedener Elemente}

In den Abbildungen \ref{fig:plot3} bis \ref{fig:plot6} sind die Absorptionsspektren vier verschiedener Elemente dargestellt,
aus denen die Winkel abgelesen werden, bei denen die K-Kanten auftreten.
Mit diesen Winkeln können die Absoprtionsenergien bestimmt werden, die in folgende Gleichung eingesetzt werden, um die Abschirmzahlen zu berechnen.
\begin{align*}
   E_\text{abs} &= h\nu = E_1 - E_\infty \\
   E_\text{abs} &= R_\infty (z-\sigma_1)^2 \frac{1}{1^2} - 0 \\
   \sigma_1 &= z - \sqrt{\frac{E_\text{abs}}{R_\infty}}
\end{align*}
Die Tabelle \ref{tab:abs} umfasst die entsprechenden Werte.
\begin{table}
  \centering
  \caption{Aus Kristallwinkel bzw. Absoprtionsenergie berechnete Abschirmzahlen.}
  \label{tab:abs}
  \begin{tabular}{c c c c c}
    \toprule
    & Abbildung & $\Theta_K$ $[^\circ]$ & $E_\text{abs}$ $[\si{\kilo\eV}]$ & $\sigma_K$\\
    \midrule
    Zn & \ref{fig:plot3} & 20,0 & 8,98 & 4,30 \\
    Br & \ref{fig:plot4} & 13,4 & 13,26 & 3,78 \\
    Sr & \ref{fig:plot5} & 11,1 & 15,96 & 3,74 \\
    Zr & \ref{fig:plot6} & 10,1 & 17,52 & 4,11 \\
    \bottomrule
  \end{tabular}
\end{table}
\FloatBarrier
\begin{figure}
  \centering
  \includegraphics{plot3.pdf}
  \caption{Absorptionsspektrum von Zink.}
  \label{fig:plot3}
\end{figure}
\begin{figure}
  \centering
  \includegraphics{plot4.pdf}
  \caption{Absorptionsspektrum von Brom.}
  \label{fig:plot4}
\end{figure}
\begin{figure}
  \centering
  \includegraphics{plot5.pdf}
  \caption{Absorptionsspektrum von Strontium.}
  \label{fig:plot5}
\end{figure}
\begin{figure}
  \centering
  \includegraphics{plot6.pdf}
  \caption{Absorptionsspektrum von Zirconium.}
  \label{fig:plot6}
\end{figure}
\FloatBarrier

\subsection{Das Moseleysche-Gesetz}

Das Moseleysche-Gesetz besagt, dass die Absoprtionsenergie $E_\text{abs}$ proportional zu $Z_\text{eff}^2$ ist.
\begin{equation}
  E_\text{abs} = R_\infty (Z-\sigma)^2
\end{equation}
Aus dem Proportionalitätsfaktor kann die Rydbergkonstante bestimmt werden.
In Abbildung \ref{fig:plot7} ist $\sqrt{E_\text{abs}}$ gegen $Z$ aufgetragen, um über einen linearen Fit die Rydbergenergie zu besitmmen.
Für die Steigung und ergibt sich so ein Wert von:
\begin{equation*}
  m = \SI{0.1143}{\sqrt{\kilo\eV}}
\end{equation*}
Aus diesem berechnet sich dann die Rydbergenergie:
\begin{equation*}
  R_\infty = \SI{13.06}{\eV}
\end{equation*}
\begin{figure}
  \centering
  \includegraphics{plot7.pdf}
  \caption{proportionaler Zusammenhang zwischen Absoprtionsenergie und dem Quadrat der Ordnungszahl.}
  \label{fig:plot7}
\end{figure}
\FloatBarrier

\subsection{L-Linine des Absorptionsspektrums von Gold}

Gold, mit einer Ordnungszahl von 79, hat eine L-Kante bei \SI{14.35}{\kilo\eV} und \SI{13.73}{\kilo\eV}.
\begin{figure}
  \centering
  \includegraphics{plot8.pdf}
  \caption{Absorptionsspektrum von Gold.}
  \label{fig:plot8}
\end{figure}
Aus Abbildung \ref{fig:plot8} lassen sich folgende Werte für die Kanten ablesen:
\begin{align*}
  \sigma_{L,2} &= \SI{12.8}{\degree}\\
  \sigma_{L,3} &= \SI{15.0}{\degree}
\end{align*}
Dies entspricht Energien von:
\begin{align*}
  E_{L,2} = \SI{13.87}{\kilo\eV} \\
  E_{L,3} = \SI{11,87}{\kilo\eV}
\end{align*}
Unter Verwendung von Gleichung \eqref{eqn:gl4} berechnet sich dann für die Abschirmkonstante:
\begin{equation*}
  \sigma_L = 1,9
\end{equation*}
