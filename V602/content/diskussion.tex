\section{Diskussion}
\label{sec:Diskussion}

Der Vergleich zwischen Literatur und berechneten Werten liefert, die in Tabelle \ref{tab:abw} aufgeführten Abweichungen.
\begin{table}
  \centering
  \caption{Vergleich von Literaturwerten und gemessenen Werten.}
  \label{tab:abw}
  \begin{tabular}{c c c}
    \toprule
    & $\increment E_K = \frac{E_K - E^\text{Lit}_K}{E^\text{Lit}_K}$ & $\increment \sigma_K = \frac{\sigma_K - \sigma^\text{Lit}_K}{\sigma^\text{Lit}_K}$ \\
    \midrule
    Zn & \SI{6.94}{\percent} & \SI{20.79}{\percent} \\
    Br & \SI{1.63}{\percent} & \SI{7.39}{\percent} \\
    Sr & \SI{0.99}{\percent} & \SI{4.45}{\percent} \\
    Zr & \SI{2.72}{\percent} & \SI{13.85}{\percent} \\
    \bottomrule
  \end{tabular}
\end{table}
Außerdem weicht die berechnete Rydbergkonstante um \SI{3.97}{\percent} vom Literaturwert ab.
Die Abweichungen generell können einerseits auf Fehlerquellen im Aufbau zurückgeführt werden,
wie das begrenzte Auflösungsvermögen der Messinstrumente oder die minimale Abweichung vom Sollwinkel.
Auf der anderen Seite ist es nicht möglich, die gesuchten Winkel exakt aus den gewonnenen Daten abzulesen,
was zu weiteren Ungenauigkeiten führt.
