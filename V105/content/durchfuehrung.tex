\section{Durchführung}
\label{sec:Durchführung}
\subsection{Bestimmung durch Gravitation}
Die Aluminiumstange mit der verstellbaren Masse wird in die Kugel gesteckt,
die auf dem Messingzylinder platziert wird.
Am Steuergerät wird das Luftkissen eingeschaltet und die Feldrichtung auf \enquote{up} und
der Feldgradient auf \enquote{off} gestellt.
Für einen vorher festgelegten Abstand $r$ der Hebelmasse wird das $\symbf{B}$-Feld über
die Stromstärke $I$ so reguliert, dass sich die Kugel im Gleichgewicht befindet.
Die Werte für $r$ und $I$ werden notiert und die Messung für verschiedene Abstände $r$ wiederholt. 

\subsection{Bestimmung durch Schwingungsdauer}


\subsection{Bestimmung durch Präzession}
