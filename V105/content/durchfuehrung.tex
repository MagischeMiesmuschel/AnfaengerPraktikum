\section{Durchführung}
\label{sec:Durchführung}
\subsection{Bestimmung über Gravitation}
Die Aluminiumstange mit der verstellbaren Masse wird in die Kugel gesteckt,
die auf dem Messingzylinder platziert wird.
Am Steuergerät wird das Luftkissen eingeschaltet und die Feldrichtung auf \enquote{up} und
der Feldgradient auf \enquote{off} gestellt.
Für einen vorher festgelegten Abstand $r$ der Hebelmasse wird das $\symbf{B}$-Feld über
die Stromstärke $I$ so reguliert, dass sich die Kugel im Gleichgewicht befindet.
Die Werte für $r$ und $I$ werden notiert und die Messung für verschiedene Abstände $r$ wiederholt.

\subsection{Bestimmung über Schwingungsdauer}

Die vorherigen Einstellungen am Steuergerät werden beibehalten.
Es werden 10 Messungen für unterschiedliche Magentfeldstärken $\symbf{B}$ durchgeführt.
Die Kugel wird um einen kleinen Winkel ausgelenkt, sodass sie wie ein harmonischer Oszillator schwinkt.
Anschließend werden 10 Periodendauern $T$ am Stück gemessen und das Ergebnis gemittlet.

\subsection{Bestimmung über Präzession}

Am Steuergerät wird jetzt noch zusätlich das Stroboskop mit einer Frequenz von 5,1 Hertz eingeschaltet.
Die Kugel wird in eine möglichst stabile Rotation versetzt, um später eine ungewollte Nutation zu vermeiden.
Dabei soll die Drehachse nicht vertikal sein.
Mit Hilfe des Stroboskops wird die Frequenz der Kugel eingestellt.
Wenn der weiße Punkt auf der Kugel stationär erscheint, stimmt die Frequenz mit der des Stroboskops überein.
Jetzt wird das Magentfeld eingeschaltet und die Kugel beginnt zu präzedieren.
Es wird die Zeit $T$ von 3 Umläufen gemessen und die mittlere Umlaufzeit gebildet.
Die Messung wird durchgeführt mit 10 verschiedenen Magnetfeldstärken $\symbf{B}$.
