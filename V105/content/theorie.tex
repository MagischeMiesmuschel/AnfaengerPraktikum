\section{Theorie}
\label{sec:Theorie}
In der Natur gibt es, anders als bei Ladungen, kein magnetisches Monopl.
Die einfachste Form des Magnetismus ist der \textit{magnetische Dipol} mit in sich geschlossenen Magnetfeldlinien.
Durch einen Permanentmagneten oder eine stromdurchflossene Leiterschleife kann man einen makroskopischen Dipol realisieren,
die das \textit{magnetische Moment}
\begin{equation}
    \symbf{\mu} = I \cdot \symbf{A}
\end{equation}
besitzt, wobei $I$ der Strom in der Leiterschleife und $A$ die Querschnittsfläche der Schleife ist.
Für einen Permanentmagneten ist $\symbf{\mu}$ nicht so einfach zu berechnen; es kann aber auf verschiedene Weise experimentell bestimmt werden.

In einem homogenen Magnetfeld wirkt auf einen Dipol ein Drehmoment $\symbf{D} = \symbf{\mu} \times \symbf{B} $. Der Dipol erfährt hierbei solange eine Drehung, bis das magnetische Moment $\symbf{\mu}$ und die magnetische Flußdichte $\symbf{B}$ gleichgerichtet sind. 
Zum Aufbau eines \textit{homogenen Magnetfeldes} werden häufig zwei gleichsinnig vom Strom $I$ durchflossene Kreisspulen so angeordnet, daß die Achsen zusammenfallen und dass der gegenseitige Abstand der Spulen dem Spulenradius $R$ entspricht. 
Das Magnetfeld im Inneren des \textit{Helmholtz-Spulenpaares} ist auf der Symmetrieachse homogen und lässt sich aus aus dem \textit{Biot-Savartschen Gesetz}
\begin{equation}
    d\symbf{B} = \frac{\mu_0 I}{4 \pi} \frac{d\symbf{s} \times \symbf{r}}{r^3}
\end{equation}
für eine stromdurchflossene Spule mit einer Windung
\begin{equation}
    \symbf{B}(x) = \frac{\mu_0 I}{2} \frac{R^2}{(R^2 + x^2)^{\frac{3}{2}}} 
\end{equation}
herleiten. Das Feld im Zentrum des Helmholtz-Spulenpaares findet man durch Überlagerung der Einzelfelder, wobei der Ursprung im Idealfall in der Mitte des Spulenpaares gelegt wird. 
In diesem Experiment unterscheidet sich der Spulenradius R geringfügig vom Abstand $d = 2·x$, sodaß der allgemeine Fall berechnet wird. Das Feld in der Mitte der Helmholtz-Spulen ergibt sich dann zu
\begin{equation}
    B(0) = B_1 (x) + B_1 (-x) = \frac{\mu_0 I R^2}{(R^2 + x^2)^{\frac{3}{2}}}
\end{equation}


\subsection{Bestimmung des magnetischen Momentes eines Magnetens unter Ausnutzung der Gravitation}
\label{sec:Gravitation}
Bei dieser statistischen Methode wirkt auf die Masse m die Gravitationskraft $\symbf{F}_g = m \cdot \symbf{g}$, 
die ein Drehmoment $\symbf{D}_g = m \cdot (\symbf{r} \times \symbf{g})$ auf die Billardkugel ausübt.
Die verschiebbare Masse m ist auf einen Aluminiumstab gesteckt, der wiederum in die Kugel gesteckt werden kann.
Der Abstand r für das Drehoment ist der Abstand vom Anfang des Stabes bis zur Masse m.
Die Gravitationskraft wirkt dem Magnetfeld $\symbf{B}$ der Spulen entgegen.
Bei einer gegebenen Magnetfeldstärke liegt ein Gleichgewicht zwischen dem Drehmoment $\symbf{D}_B = \symbf{\mu}_{Dipol} \times \symbf{B}$ und dem Drehmoment $\symbf{D}_g$,
welches die Gravitation verursacht.
\begin{equation}
    \symbf{\mu}_{Dipol} \times \symbf{B} = m \cdot (\symbf{r} \times \symbf{g})
\end{equation}

Das Kreuzprodukt kann man durch $ r m g sin(\theta) = \mu_{Dipol} B sin(\theta)$ ersetzen.
$\theta$ ist der von dem Aluminiumstab und dem Magnetfeld eingeschlossene Winkel.
Da $\symbf{g}$ und $\symbf{B}$ parallel sind, fällt die Winkeabhängigkeit weg und man kann das magnetische Moment durch den Abstand r und das Magnetfeld bestimmen.
\begin{equation}
    \mu_{Dipol} \cdot B = m \cdot r \cdot g
\end{equation}


\subsection{Bestimmung des magnetischen Moments über die Schwingungsdauer eines Magnetens}
\label{sec:Schwingungsdauer}
Die schwingende Billardkugel verhält sich im homogenen magnetischen Feld der Helmholtz-Spulen wie ein harmonischer Oszillator, 
dessen Bewegung sich durch 
\begin{equation}
    -|\symbf{\mu}_{Dipol} \times \symbf{B}| = J_K \cdot \frac{d^2\theta}{dt^2}
\end{equation}
beschreiben lässt. Die Lösung dieses DGL ist die Schwingungsdauer T der oszilierenden Kugel.
Das magnetische Moment $\mu_{Dipol}$ lässt sich dann quantitativ über das Trägheitsmoment $J_K$ der Kugel,
der Magnetfeldstärke $B$ und der Schwingungsdauer
\begin{equation}
    T^2 = \frac{4 \pi^2 J_K}{\mu_{Dipol}} \frac{1}{B}
\end{equation}
bestimmen.

\subsection{Bestimmung des magnetischen Moments über die Präzession eines Magneten}
\label{sec:Präzession}
Wirkt eine äußere Kraft auf die Drehachse eines rotierenden Körpers, dann führt die Figurenachse eine Präzessionsbewegung aus.
Die Präzessionsbewegung im Magnetfeld der Helmholtzspulen lässt sich durch die Differentialgleichung
\begin{equation}
    \symbf{\mu_{Dipol}} \times \symbf{B} = \frac{d\symbf{L}_K}{dt}
\end{equation}
beschreiben, wobei die Präzessionsfrequenz $\Omega_p$
\begin{equation}
    \Omega_p = \frac{\mu B}{|L_K|}
\end{equation}
eine Lösung der Differentailgleichung ist.
Den Drehimpuls $L_K = J_K \omega$ der Kugel kann man über das Trägheitsmoment $J_K$ der Billiardkugel und deren Kreisfrequenz $\omega = 2\pi \nu$ bestimmen.
Da sich die Präzessionsfrequenz $\Omega_p$ aus der Zeit $T_p$ für einen Umlauf berechnen lässt, lässßt sich das magnetische Moment $\mu_{Dipol}$ der Kugel über
\begin{equation}
    \frac{1}{T_p} = \frac{\mu_{Dipol}}{2 \pi L_K} B
\end{equation}
berechnen.

Durch ein Stroboskop wird die notwendige Konstanz der Rotationsfrequenz $\nu = \omega / 2 \pi$ kontrolliert.
Wenn die weiße Markierung auf der Kugel stationär unter dem Stroboskop erscheint, dann hat die Kugel die am Gerät eigenstellte Frequenz.
Da die Frequenz exponentiell mit der Zeit abnimmt, muss man direkt nach dem Erreichen der eingestellten Frequenz mit dem Messen beginnen und 
eine geeignete Frequenz wählen, zwischen $\nu = \SI{4}{\Hz}$ und $\nu = \SI{6}{\Hz}$, weil hier der Abfall der Exponentialkurve bereits hinreichend langsam geschieht (etwa 2 Hz pro Minute).

