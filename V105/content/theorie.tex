\section{Theorie}
\label{sec:Theorie}
In der Natur gibt es, anders als bei Ladungen, kein magnetisches Monopl.
Die einfachste Form des Magnetismus ist der \textit{magnetische Dipol} mit in sich geschlossenen Magnetfeldlinien.
Durch einen Permanentmagneten oder eine stromdurchflossene Leiterschleife kann man einen makroskopischen Dipol realisieren,
die das \textit{magnetische Moment}
\begin{equation}
    \symbf{\mu} = I * \symbf{A}
\end{equation}
besitzt, wobei $I$ der Strom in der Leiterschleife und $A$ die Querschnittsfläche der Schleife ist.
Für einen Permanentmagneten ist $\symbf{\mu}$ nicht so einfach zu berechnen; es kann aber auf verschiedene Weise experimentell bestimmt werden.

In einem homogenen Magnetfeld wirkt auf einen Dipol ein Drehmoment $\symbf{D} = \symbf{\mu} \times \symbf{B} $. Der Dipol erfährt hierbei solange eine Drehung, bis das magnetische Moment $\symbf{\mu}$ und die magnetische Flußdichte $\symbf{B}$ gleichgerichtet sind. 
Zum Aufbau eines \textit{homogenen Magnetfeldes} werden häufig zwei gleichsinnig vom Strom $I$ durchflossene Kreisspulen so angeordnet, daß die Achsen zusammenfallen und dass der gegenseitige Abstand der Spulen dem Spulenradius $R$ entspricht. 
Das Magnetfeld im Inneren des \textit{Helmholtz-Spulenpaares} ist auf der Symmetrieachse homogen und lässt sich aus aus dem \textit{Biot-Savartschen Gesetz}
\begin{equation}
    d\symbf{B} = \frac{\mu_0 I}{4 \pi} \frac{d\symbf{s} \times \symbf{r}}{r^3}
\end{equation}
für eine stromdurchflossene Spule mit einer Windung
\begin{equation}
    \symbf{B}(x) = \frac{\mu_0 I}{2} \frac{R^2}{(R^2 + x^2)^{\frac{3}{2}}} 
\end{equation}
herleiten. Das Feld im Zentrum des Helmholtz-Spulenpaares findet man durch Überlagerung der Einzelfelder, wobei der Ursprung im Idealfall in der Mitte des Spulenpaares gelegt wird. 
In diesem Experiment unterscheidet sich der Spulenradius R geringfügig vom Abstand $d = 2·x$, sodaß der allgemeine Fall berechnet wird. Das Feld in der Mitte der Helmholtz-Spulen ergibt sich dann zu
\begin{equation}
    B(0) = B_1 (x) + B_1 (-x) = \frac{\mu_0 I R^2}{(R^2 + x^2)^{\frac{3}{2}}}
\end{equation}


\subsection{Bestimmung des magnetischen Momentes eines Magnetens unter Ausnutzung der Gravitation}


\cite{sample}
