\section{Auswertung}
\label{sec:Auswertung}

Für die Auswertung werden neben NumPy\cite{numpy} mehrere Python Pakete benutzt.
Plots werden mit Matplotlib\cite{matplotlib} erstellt und Ausgleichsgeraden mit SciPy\cite{scipy}.
Fehlerbehaftete Größen werden mit Uncertainties\cite{uncertainties} berechnet, das auf der Gaußschen Fehlerfortpflanzung basiert:
\begin{equation}
    \increment f = \sqrt{\sum_{i=1}^N \left( \frac{\partial f}{\partial x_i} \right)^{2} \cdot (\increment x_i)^{2}}
\end{equation}
Alle gemessenen Größen werden als fehlerbehaftet betrachtet.
Die Größe der Fehler wird an der Ungenauigkeit der Messinstrumente orientiert:
\begin{itemize}
  \item Das Magentfeld konnte auf 0,1 Ampere genau eingestellt werden.
  \item Die Schieblehre hatte eine Millimeterskala, also eine Ungenauigkeit von $10^{-3}$ Meter.
  \item Die Stoppuhr maß auf 0,01 Sekunden genau.
\end{itemize}

\subsection{Bestimmung über Gravitation}



\begin{figure}
  \centering
  \includegraphics{plot1.pdf}
  \caption{Gravitation.}
  \label{fig:plot1}
\end{figure}

\subsection{Bestimmung über Schwingungsdauer}

\begin{figure}
  \centering
  \includegraphics{plot2.pdf}
  \caption{Schwingung.}
  \label{fig:plot2}
\end{figure}

\subsection{Bestimmung über Präzession}

\begin{figure}
  \centering
  \includegraphics{plot3.pdf}
  \caption{Präzession.}
  \label{fig:plot3}
\end{figure}
