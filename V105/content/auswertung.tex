\section{Auswertung}
\label{sec:Auswertung}

Für die Auswertung werden neben NumPy\cite{numpy} mehrere Python Pakete benutzt.
Plots werden mit Matplotlib\cite{matplotlib} erstellt und Ausgleichsgeraden mit SciPy\cite{scipy}.
Fehlerbehaftete Größen werden mit Uncertainties\cite{uncertainties} berechnet, das auf der Gaußschen Fehlerfortpflanzung basiert:
\begin{equation}
    \increment f = \sqrt{\sum_{i=1}^N \left( \frac{\partial f}{\partial x_i} \right)^{2} \cdot (\increment x_i)^{2}}
\end{equation}
Alle gemessenen Größen werden als fehlerbehaftet betrachtet.
Die Größe der Fehler wird an der Ungenauigkeit der Messinstrumente orientiert:
\begin{itemize}
  \item Das Magentfeld konnte auf 0,1 Ampere genau eingestellt werden.
  \item Die Schieblehre hatte eine Millimeterskala, also eine Ungenauigkeit von $10^{-3}$ Meter.
  \item Die Stoppuhr maß auf 0,01 Sekunden genau.
\end{itemize}

\subsection{Bestimmung über Gravitation}

Aus den eingestellten Stromstärken $I$ werden mit Gleichung \eqref{eqn:bfeld} die entsprechenden Magnetfeldstärken $\symbf{B}$ berechnet.
Um das magnetische Moment zu bestimmen,
muss zunächst eine Ausgleichsrechnung für die gemessenen Größen durchgeführt werden.
Dafür wird $r$ gegen $\symbf{B}$ aufgetragen (siehe Abbildung \ref{fig:plot1}) und eine Regressionsgerade berechnet mit:
\begin{equation}
  f(x) = ax+b
\end{equation}
\begin{figure}
  \centering
  \includegraphics{plot1.pdf}
  \caption{Gravitation.}
  \label{fig:plot1}
\end{figure}
Es ergeben sich folgende Parameter für die Ausgleichsgerade:
\begin{equation*}
  a = (2,99 \pm 0,05) \cdot 10^{-2} \: \text{T/m}
\end{equation*}
\begin{equation*}
  b = (6,77 \pm 0,48) \cdot 10^{-4} \: \text{T/m}
\end{equation*}
Gleichung \eqref{eqn:gravitation} lässt sich dann zu umstellen und das magnetsiche Moment kann berechnet werden:
\begin{equation*}
  \symbf{\mu}_{Dipol} = (0,459 \pm 0,008) \: \text{A}^{2}/\text{m}
\end{equation*}

\subsection{Bestimmung über Schwingungsdauer}

Genau wie zuvor muss zunächst mit Gleichung \eqref{eqn:bfeld} die Magentfeldstärke berechnet werden.
Im nächsten Schritt wird, um das magentische Moment zu bestimmen, $1/B$ gegen $T^{2}$ aufgetragen und die lineare Regression durchgeführt.
Für die Ausgleichsgerade (siehe Abbildung \ref{fig:plot2}) werden folgende Parameter ermittelt:
\begin{equation*}
  a = (3,30 \pm 0,13) \cdot 10^{-3} \: \text{s}^{2}/\text{T}
\end{equation*}
\begin{equation*}
  b = (3,82 \pm 7,64) \cdot 10^{-2} \: \text{s}^{2}/\text{T}
\end{equation*}
\begin{figure}
  \centering
  \includegraphics{plot2.pdf}
  \caption{Schwingung.}
  \label{fig:plot2}
\end{figure}
Gleichung \eqref{eqn:schwingung} lässt sich dann umstellen zu
\begin{equation}
  \symbf{\mu}_{Dipol} = 4 \pi^2 J_K \frac{1}{a}
\end{equation}
mit $a = T^{2} \cdot B$.
Das magnetische Moment lässt sich dann einfach berechnen:
\begin{equation*}
  \symbf{\mu}_{Dipol} = (0,442 \pm 0,018) \: \text{A}^{2}/\text{m}
\end{equation*}

\subsection{Bestimmung über Präzession}

\begin{figure}
  \centering
  \includegraphics{plot3.pdf}
  \caption{Präzession.}
  \label{fig:plot3}
\end{figure}
