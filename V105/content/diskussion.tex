\section{Diskussion}
\label{sec:Diskussion}
Durch Vergleich der drei Messmethoden lassen sich Aussagen über die Fehlerträchtigkeit und den Aufwand der einzelnen Verfahren machen.
Beginnend mit den negativen Aspekten der Gravitationsmethode:
\begin{itemize}
    \item Mit dem bloßen Augen lässt sich das perfekte Gleichgewicht schwer erkennen, da ein \texttt{\glqq Zittern\dq} oft noch vorhanden ist.
    \item Durch längeres Aufrechthalten des Magnetfeldes schwächt es sich ab.
\end{itemize}
Die Schwingungsdauer:
\begin{itemize}
    \item Die Kugel könnte durch zu große Winkel ausgelenkt werden und die Näherung wäre zu ungenau. 
\end{itemize}
Die Präzession:
\begin{itemize}
    \item Die Frequenz nimmt während langer Umlaufdauer merkbar ab.
    \item Bei hellerer Beleuchtung im Raum ist der stationäre Punkt schwer zu sehen.
    \item Die Rotation ist schwer ohne \texttt{\glqq Zittern\dq} einzustellen.
    \item Durch die lange Einstellzeit ist dieses Verfahren sehr Zeitaufwendig und durch die vergleichsweise komplexe Durchführung am kompliziertesten.
\end{itemize}

Mit Betrachtung der Messergebnisse und der darauf basierenden Plots ist klar zu erkennen, dass es bei der Gravitationsmethode die geringsten Abweichungen gab. 
Dies ist der Fall durch die schnelle Durchführbarkeit der Methode, somit schächt sich das Magnetfeld nur sehr wenig über die kurze Zeit ab.
Auch war es visuell gut einzusehen und in den meisten Fällen sehr eindeutig zu bestimmen. 