\section{Diskussion}
\label{sec:Diskussion}
Durch einen Vergleich der drei Messmethoden und deren Ergebnissen lassen sich Aussagen über deren Genauigkeit und Aufwand treffen.
Den Geringsten Aufwand bei der Durchführung ergibt sich aus dem Messen der Schwingungsdauer. 
Dort wird lediglich um einen kelinen Winkel ausgelenkt und es ist gut observabel wie die Schwingung zu bestimmen ist.
Dagegen liefert die Messung über Präzession den höchsten Aufwand. 
Durch den hohen Zeitaufwand beim einstellen der richtigen Frequenz durch das Strobskop und die darauf noch folgende Messung der Umlaufdauer,
wird hier am meisten Zeit gebraucht um Messergebnisse zu erhalten. 
Desweiteren nimmt die Frequenz der Kugel während des Durchlaufens der Präzession ab, welches besonders bei geringem Magnetfeld und somit langer Umlaufdauer zu Fehlern führt.
Die besten Messergebnisse mit Hinblick auf den Erwartungswert liefert hingegen das Messen über die Gravitation. 
Das einstellen des Gleichgewichts ist aufwändiger als die Messung der Schwingungsdauer, jedoch mit hinreichend ruhiger Hand ist diese Methode nur marginal Aufwändiger.
Mit den erhobenen Messdaten ist zu sehen, dass die Genauigkeit der Methoden des Messens über Schwingung und Präzession gegenüber der Messung über die Gravitation, im Nachteil sind
und unpräzisere Ergebnisse liefern.
Auf Basis dieser Messergebnisse und dem Vergleich aller Methoden ist die Aussage zu treffen,
dass die Methode der Bestimmung des magnetichen Moments über die Gravitation die genaueste und zu bevorzugende Methode ist.