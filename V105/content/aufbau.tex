\section{Aufbau}
\label{sec:Aufbau}

Ein äußeres Magnetfeld wird durch ein Helmholtz-Spulenpaar mit  $(\text{N} \: = \: 195)$ Windungen erzeugt,
deren Radius $R_{\text{Spule}} = 0.109 \: \si{\meter}$ beträgt und die in einem Abstand von $(\text{d} \: = \: 0.138 \: \si{\meter})$ stehen.
In der Mitte der beiden Spulen steht ein zylindrisches Podest, auf dem sich eine Billiardkugel mit einem mittig eingelassenen Permanentmagneten ,
reibungsfrei mittels eines Luftkissens bewegen lässt.
Der Permanentmagnet ist so ausgerichtet, dass sein magnetisches Moment $\mu_{\text{Dipol}}$ in Richtung des Stiels gerichtet ist, der sich auf der Kugel befindet.
Zur Bestimmung der Drehfrequenz befindet sich auf der Kugel ein weißer Punkt, den man mit Hilfe eines ,an der oberen Spule befestigtem, Stroboskops,
zum stehen bringen kann, so dass die Drehfrequenz mit der Frequenz des Stroboskops übereinstimmt.
Das Stroboskop, das Luftkissen und der Spulenstrom und somit auch das externe Magnetfeld, werden über ein Steuergerät angesteuert.
