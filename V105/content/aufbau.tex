\section{Aufbau}
\label{sec:Aufbau}

\textcquote{sample}{Ein kleiner zylindrischer Permanentmagnet befindet sich in der Mitte einer Billiardkugel, dessen magnetisches Moment $\mu_{\text{Dipol}}$
in Richtung des Stiels gerichtet ist, der sich auf der Kugel befindet.
Das äußere Magnetfeld wird durch ein Helmholtz-Spulenpaar $(\text{N} \: = \: 195)$ erzeugt, dessen Spulen einen Abstand von
d = 0.138 m und einen Radius von $R_{\text{Spule}} = 0.109 \: \si{\meter}$ haben.
In der Mitte der Helmholtz-Spulen befindet sich ein Messingzylinder, auf dem sich die Kugel
mit dem Permanentmagneten mittels eines Luftkissens reibungsfrei bewegen kann.
Zur Bestimmung der Drehbewegung befindet sich ein Stroboskop an der oberen Spule des Helmholtz-Spulenpaares.
Der Spulenstrom und somit auch das externe Magnetfeld, das Stoboskop und
das Luftkissen können über ein Steuergerät angesteuert werden.}
