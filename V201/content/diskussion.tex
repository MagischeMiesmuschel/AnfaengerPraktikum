\section{Diskussion}
\label{sec:Diskussion}
Die Abweichungen der Molwärmen von den, durch das Dulong-Petitsche Gesetz erwarteten, 3R zeigen die Inkorektheit dieses Gesetzes.
Auch wenn Fehlerquellen in diesem Veruch an vielen Stellen auftreten, ist deren Wirkung nicht groß genug, um an der Aussage zu zweifeln.
Die Abweichung von den Wärmekapazitäten können durch keine genaue Temperaturbestimmung das Materials verschuldet sein.
Es wurde lediglich die Wassertemperatur in dem das Material erhitzt wurde gemessen.
Auch die Bestimmung der Mischtempemperatur ist nicht exakt möglich.
Durch den zu schwachen Rührfisch, welcher sich oft nicht drehte war ein konstantes Umrühren und Vermischen nicht möglich.
Somit gab es Temperaturdifferenzen zwischen der oberen und unteren Wasserschicht von bis zu $3^\circ$C.
Durch längeres Abwarten hätte sich diese Differenz gelegt, jedoch durch den nicht dichten Verschluss des Kalorimeters entweicht durch den Probendeckel immer mehr Wärme in die Umgebung, je länger gewartet wird.
Dadurch ist bereits die Annahme beim Aufstellen der Gleichungen \eqref{eqn:gl11} und \eqref{eqn:gl12}, dass die beiden Wärmemengen eins zu eins in einander übergehen, fehlerhaft.
Weil dies nur für abgeschlossene Systeme gilt.
