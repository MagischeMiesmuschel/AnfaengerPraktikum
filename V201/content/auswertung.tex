\section{Auswertung}
\label{sec:Auswertung}

\subsection{Wärmekapazität des Kalorimeters}
\label{sec:Kalorie}
Für die spätere Berechnung der spezifischen Wärmekapazitäten verschiedener Stoffe wird unter anderem die Wärmekapazität des Kaloriemeters benötigt.
Die Berechnung dieser erfolgt durch Gleichung \eqref{eqn:gl12} und mit den gemessenen Werten:
\begin{align*}
 T_\text{x} = 293,75 \text{K}\\
 m_\text{x} = 312,42 \text{g}\\
 T_\text{y} = 359,15 \text{K}\\
 m_\text{y} = 300,54 \text{g}\\
 T_\text{m} = 323,15 \text{K}\\
 m_\text{g} = 841,85 \text{g}
\end{align*}
Für $c_\text{w}$ wird der in \cite{V201} angegeben Wert von $\SI{4,18}{\joule\per\gram\per\kelvin}$ verwendet.
Somit berechnet sich ein Wert von $ c_\text{g} = \SI{0,276}{\joule\per\gram\per\kelvin}.$

\subsection{Wärmekapazität verschiedener Stoffe}
\label{sec:verschiedeneStoffe}

Analog zum vorherigen Auswertungsteil \ref{sec:Kalorie} wird die Wärmekapazität $c_\text{k}$ der Stoffe bestimmt durch die Gleichung \eqref{eqn:gl11}.
Bei den drei Messungen für Blei ergeben sich folgende Werte:
\begin{table}
  \centering
  \label{tab:WerteBlei}
  \begin{tabular}{c c c c}
   \toprule
   & Messung 1 & Messung 2 & Messung 3 \\
   \midrule
   $m_\text{w}$/g & 599,956 & 589,26 & 572,89 \\
   $T_\text{w}$/K & 294,45 & 293,95 & 294,35 \\
   $T_\text{Blei}$/K & 366,15 & 360,15 & 363,15 \\
   $T_\text{m}$/K & 296,35 & 295,05 & 295,75 \\
   \bottomrule
  \end{tabular}
\end{table}
\FloatBarrier
Die Werte für die spezifische Wärmekapazität werden mit
\begin{equation}
  \overline{x} = \frac{1}{N} \cdot \sum_{i=1}^N x_i
\end{equation}
gemittelt. Der zugehörige Fehler bestimmt sich durch
\begin{equation}
  \Delta \overline{x} = \frac{1}{\sqrt{N}} \cdot \sqrt{\frac{1}{N-1} \cdot \sum_{i=1}^N \left(x_i - \overline{x}\right)^2}
\end{equation}
Somit ergibt sich die spezifische Wärmekapazität für Blei zu:
\begin{align*}
  c_\text{k} = \left(0,108 \pm 0,016 \right) \si{\joule\per\gram\per\kelvin}
\end{align*}
Dies entspricht einer Abweichung von 16,3\% vom Literaturwert $c_\text{k} = \SI{0,130}{\joule\per\gram\per\kelvin}$.

Für Kupfer wird erneut die Gleichung \eqref{eqn:gl11} benutzt, mit den Daten
\begin{table}
  \centering
  \label{tab:WerteKupfer}
  \begin{tabular}{c c c c}
   \toprule
   & Messung 1 & Messung 2 & Messung 3 \\
   \midrule
   $m_\text{w}$/g & 572,89 & 559,56 & 633,43 \\
   $T_\text{w}$/K & 294,35 & 294,35 & 294,35 \\
   $T_\text{Kupfer}$/K & 361,15 & 363,15 & 367,15 \\
   $T_\text{m}$/K & 294,85 & 298,65 & 297,15 \\
   \bottomrule
  \end{tabular}
\end{table}
\FloatBarrier
ergibt sich die spezifische Wärmekapazität von Kupfer zu:
\begin{align*}
  c_\text{k} = \left(0,444 \pm 0,192 \right) \si{\joule\per\gram\per\kelvin}
\end{align*}
Dies entspricht einer Abweichung von 15,3\% vom Literaturwert $c_\text{k} = \SI{0,385}{\joule\per\gram\per\kelvin}$.

Für Aluminium wird erneut die Gleichung \eqref{eqn:gl11} benutzt, mit den Daten
\begin{table}
  \centering
  \label{tab:WerteAluminium}
  \begin{tabular}{c c c c}
   \toprule
   & Messung 1 & Messung 2 & Messung 3 \\
   \midrule
   $m_\text{w}$/g & 573 & 566,22 & 588,78 \\
   $T_\text{w}$/K & 294,75 & 294,85 & 294,75 \\
   $T_\text{Aluminium}$/K & 361,65 & 363,15 & 365,95 \\
   $T_\text{m}$/K & 299,15 & 298,75 & 298,55 \\
   \bottomrule
  \end{tabular}
\end{table}
\FloatBarrier
ergibt sich die spezifische Wärmekapazität von Aluminium zu:
\begin{align*}
  c_\text{k} = \left(0,105 \pm 0,069 \right) \si{\joule\per\gram\per\kelvin}
\end{align*}
Dies entspricht einer Abweichung von 23,2\% vom Literaturwert $c_\text{k} = \SI{0,897}{\joule\per\gram\per\kelvin}$.

\subsection{Molwärme der Stoffe}
\label{sec:Molwärme}
Die Molwärme bei konstantem Druck ergibt sich nach Gleichung \eqref{eqn:gl4}.
Die Werte von $M$, $\alpha$ und $\kappa$ werden der Versuchsanleitung entnommen \cite{V201}.
Mit der Benutzung der drei berechneten $c_\text{k}$ ergeben sich für die Molwärmen der Stoffe folgende Werte:
\begin{align*}
  C_{p,\text{Blei}} = \SI{22,54 \pm 3.32}{\joule\per\gram\per\kelvin}\\
  C_{p,\text{Kupfer}} = \SI{6,91 \pm 1,02}{\joule\per\gram\per\kelvin}\\
  C_{p,\text{Aluminium}} = \SI{2,94 \pm 0,43}{\joule\per\gram\per\kelvin}
\end{align*}
Dies ergibt für Blei eine Abweichung von 9,48\% von den zu überprüfenden 3R (= $\SI{24,9}{\joule\per\gram\per\kelvin}$) aus dem Dulong-Petitschen Gesetz.
Für Kupfer beträgt die Abweichung somit 72,26\% von 3R und für Aluminium liegt diese bei 88,20\%.
