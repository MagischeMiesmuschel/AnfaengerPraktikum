\section{Auswertung}
\label{sec:Auswertung}

\subsection{verwendete Software und Fehlerrechnung}
\label{sec:SoftwareFehlerrechnung}

Für die Auswertung werden neben NumPy\cite{numpy} mehrere Python Pakete benutzt.
Plots werden mit Matplotlib\cite{matplotlib} erstellt und Ausgleichsgeraden mit SciPy\cite{scipy}.
Fehlerbehaftete Größen werden mit Uncertainties\cite{uncertainties} berechnet, das die Gaußsche Fehlerfortpflanzung benutzt:
\begin{equation*}
    \increment f = \sqrt{\sum_{i=1}^N \left( \frac{\partial f}{\partial x_i} \right)^{2} \cdot (\increment x_i)^{2}}
    \label{eqn:fehler}
\end{equation*}

\subsection{Wärmekapazität des Kalorimeters}
\label{sec:Kalorie}
Für die spätere Berechnung der spezifischen Wärmekapazitäten verschiedener Stoffe wird unter anderem die Wärmekapazität des Kalorimeters benötigt.
Die Berechnung dieser erfolgt durch Gleichung \eqref{eqn:gl12} und mit den gemessenen Werten:
\begin{align*}
 T_\text{x} = 293,75 \text{K}\\
 m_\text{x} = 312,42 \text{g}\\
 T_\text{y} = 359,15 \text{K}\\
 m_\text{y} = 300,54 \text{g}\\
 T_\text{m} = 323,15 \text{K}\\
 m_\text{g} = 841,85 \text{g}
\end{align*}
Für $c_\text{w}$ wird der in \cite{V201} angegeben Wert von $\SI{4,18}{\joule\per\gram\per\kelvin}$ verwendet.
Somit berechnet sich ein Wert von $ c_\text{g} = \SI{0,276}{\joule\per\gram\per\kelvin}.$

\subsection{Wärmekapazität verschiedener Stoffe}
\label{sec:verschiedeneStoffe}

Analog zum vorherigen Auswertungsteil \ref{sec:Kalorie} wird die Wärmekapazität $c_\text{k}$ der Stoffe bestimmt durch die Gleichung \eqref{eqn:gl11}.
Bei den drei Messungen für Blei ergeben sich folgende Werte:
\begin{table}
  \centering
  \caption{Messwerte Blei}
  \label{tab:WerteBlei}
  \begin{tabular}{c c c c}
   \toprule
   & Messung 1 & Messung 2 & Messung 3 \\
   \midrule
   $m_\text{w}$/g & 599,956 & 589,26 & 572,89 \\
   $T_\text{w}$/K & 294,45 & 293,95 & 294,35 \\
   $T_\text{Blei}$/K & 366,15 & 360,15 & 363,15 \\
   $T_\text{m}$/K & 296,35 & 295,05 & 295,75 \\
   \bottomrule
  \end{tabular}
\end{table}
\FloatBarrier
\noindent
Dafür ergeben sich nach Gleichung \eqref{eqn:gl11}:
\begin{table}
  \centering
  \caption{$c_\text{k}$ Blei}
  \label{tab:ckBlei}
  \begin{tabular}{c c c c}
   \toprule
   & Messung 1 & Messung 2 & Messung 3 \\
   \midrule
   $c_\text{k}$ in $\si{\joule\per\gram\per\kelvin}$ & 0.139 & 0.085 & 0.102 \\
   \bottomrule
  \end{tabular}
\end{table}
\FloatBarrier
\noindent
Die Werte für die spezifische Wärmekapazität werden mit
\begin{equation}
  \overline{x} = \frac{1}{N} \cdot \sum_{i=1}^N x_i
\end{equation}
gemittelt. Der zugehörige Fehler bestimmt sich durch
\begin{equation}
  \Delta \overline{x} = \frac{1}{\sqrt{N}} \cdot \sqrt{\frac{1}{N-1} \cdot \sum_{i=1}^N \left(x_i - \overline{x}\right)^2}
  \label{eqn:fehlermittel}
\end{equation}
Somit ergibt sich die spezifische Wärmekapazität für Blei zu:
\begin{align*}
  c_\text{k} = \left(0,108 \pm 0,016 \right) \si{\joule\per\gram\per\kelvin}
\end{align*}
Dies entspricht einer Abweichung von 16,3\% vom Literaturwert $c_\text{k} = \SI{0,130}{\joule\per\gram\per\kelvin}$ \cite{Wkappa}.
\noindent
Für Kupfer wird erneut die Gleichung \eqref{eqn:gl11} benutzt, mit den Daten
\begin{table}
  \centering
  \caption{Messwerte Kupfer}
  \label{tab:WerteKupfer}
  \begin{tabular}{c c c c}
   \toprule
   & Messung 1 & Messung 2 & Messung 3 \\
   \midrule
   $m_\text{w}$/g & 572,89 & 559,56 & 633,43 \\
   $T_\text{w}$/K & 294,35 & 294,35 & 294,35 \\
   $T_\text{Kupfer}$/K & 361,15 & 363,15 & 367,15 \\
   $T_\text{m}$/K & 294,85 & 298,65 & 297,15 \\
   \bottomrule
  \end{tabular}
\end{table}
\FloatBarrier
\noindent
ergeben sich nach Gleichung \eqref{eqn:gl11}:
\begin{table}
  \centering
  \caption{$c_\text{k}$ Kupfer}
  \label{tab:ckKupfer}
  \begin{tabular}{c c c c}
   \toprule
   & Messung 1 & Messung 2 & Messung 3 \\
   \midrule
   $c_\text{k}$ in $\si{\joule\per\gram\per\kelvin}$ & 0.086 & 0.745 & 0.500 \\
   \bottomrule
  \end{tabular}
\end{table}
\FloatBarrier
\noindent
Daraus folgt im Mittel:
\begin{align*}
  c_\text{k} = \left(0,444 \pm 0,192 \right) \si{\joule\per\gram\per\kelvin}
\end{align*}
Dies entspricht einer Abweichung von 15,3\% vom Literaturwert $c_\text{k} = \SI{0,385}{\joule\per\gram\per\kelvin}$ \cite{Wkappa}.
\noindent
Für Aluminium wird erneut die Gleichung \eqref{eqn:gl11} benutzt, mit den Daten
\begin{table}
  \centering
  \caption{Messwerte Aluminium}
  \label{tab:WerteAluminium}
  \begin{tabular}{c c c c}
   \toprule
   & Messung 1 & Messung 2 & Messung 3 \\
   \midrule
   $m_\text{w}$/g & 573 & 566,22 & 588,78 \\
   $T_\text{w}$/K & 294,75 & 294,85 & 294,75 \\
   $T_\text{Aluminium}$/K & 361,65 & 363,15 & 365,95 \\
   $T_\text{m}$/K & 299,15 & 298,75 & 298,55 \\
   \bottomrule
  \end{tabular}
\end{table}
\FloatBarrier
\noindent
ergeben sich nach Gleichung \eqref{eqn:gl11}:
\begin{table}
  \centering
  \caption{$c_\text{k}$ Aluminium}
  \label{tab:ckAlu}
  \begin{tabular}{c c c c}
   \toprule
   & Messung 1 & Messung 2 & Messung 3 \\
   \midrule
   $c_\text{k}$ in $\si{\joule\per\gram\per\kelvin}$ & 1.241 & 1.056 & 1.018 \\
   \bottomrule
  \end{tabular}
\end{table}
\FloatBarrier
\noindent
Daraus folgt im Mittel:
\begin{align*}
  c_\text{k} = \left(0,105 \pm 0,069 \right) \si{\joule\per\gram\per\kelvin}
\end{align*}
Dies entspricht einer Abweichung von 23,2\% vom Literaturwert $c_\text{k} = \SI{0,897}{\joule\per\gram\per\kelvin}$ \cite{Wkappa}.

\subsection{Molwärme der Stoffe}
\label{sec:Molwärme}
Die Molwärme bei konstantem Druck ergibt sich nach Gleichung \eqref{eqn:gl4}.
Die Werte von $M$, $\alpha$ und $\kappa$ werden der Versuchsanleitung entnommen \cite{V201}.
Mit der Benutzung der drei berechneten $c_\text{k}$ Werte ergeben sich für die Molwärmen der Stoffe folgende Werte, mit Fehleberechnung nach Gleichung \eqref{eqn:fehler} und \eqref{eqn:fehlermittel}:
\begin{align*}
  C_{p,\text{Blei}} &= \SI{22,54 \pm 3.32}{\joule\per\gram\per\kelvin}\\
  C_{p,\text{Kupfer}} &= \SI{6,91 \pm 1,02}{\joule\per\gram\per\kelvin}\\
  C_{p,\text{Aluminium}} &= \SI{2,94 \pm 0,43}{\joule\per\gram\per\kelvin}
\end{align*}
Der Fehler nach Gleichung \eqref{eqn:fehler} berechnet sich in diesem Fall mit:
\begin{align*}
  \Delta C_\text{p} &= \sqrt{\frac{\partial (c_\text{k} \cdot M)}{\partial c_\text{k}}^2 \cdot (\Delta c_\text{k})^2} \\
  &= \sqrt{M^2 \cdot (\Delta c_\text{k})^2}\\
  &= M \cdot \Delta c_\text{k}
\end{align*}
Nach Gleichung \eqref{eqn:gl3} ergibt sich für $C_\text{V}$ der verschiedenen Metalle:
\begin{align*}
  C_{V,\text{Blei}} &= \SI{20,43 \pm 3.31}{\joule\per\gram\per\kelvin}\\
  C_{V,\text{Kupfer}} &= \SI{6,02\pm 1,02}{\joule\per\gram\per\kelvin}\\
  C_{V,\text{Aluminium}} &= \SI{1,58 \pm 0,44}{\joule\per\gram\per\kelvin}
\end{align*}
Der Fehler nach Gleichung \eqref{eqn:fehler} berechnet sich in diesem Fall mit:
\begin{align*}
  \Delta C_\text{V} &= \sqrt{\frac{\partial (c_\text{k} M - 9 \alpha \kappa \frac{M}{\rho}T)}{\partial c_\text{k}}^2 \cdot (\Delta c_\text{k})^2} \\
  &= \sqrt{M^2 \cdot (\Delta c_\text{k})^2}\\
  &= M \cdot \Delta c_\text{k}
\end{align*}
Dies ergibt für Blei eine Abweichung von 17,94\% von den zu überprüfenden 3R (= $\SI{24,9}{\joule\per\gram\per\kelvin}$) aus dem Dulong-Petitschen Gesetz.
Für Kupfer beträgt die Abweichung somit 75,84\% von 3R und für Aluminium liegt diese bei 93,65\%.
