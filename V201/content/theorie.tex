\section{Theorie}
\label{sec:Theorie}

\subsection{Molwärme und spezifische Wärmekapazität}
\label{sec:molwärme}

Die Molwärme ist die Wärmemenge $\text{d}Q$, die benötigt wird um einen Stoff um $\text{T}$ zu erwärmen, bezogen auf \SI{1}{\mol}.
Dabei muss zwischen Molwärme bei konstantem Druck $C_p$ und konstantem Volumen $C_V$ unterschieden werden.
Für den Fall, dass die Wärmezufuhr- oder abfuhr bei konstantem Volumen staffindet, ergibt sich für die Molwärme:
\begin{equation}
  C_V = \left(\frac{\text{d}Q}{\text{d}T}\right)_V
  \label{eqn:gl1}
\end{equation}
Da aufgrund des konstanten Volumens keine Arbeit verrichtet wird, ist nach dem ersten Hauptsatz der Thermodynamik die innere Engegie $U$ gleich der Wärmemenge $Q$.
Daraus folgt für die Molwärme:
\begin{equation}
  C_V = \left(\frac{\text{d}U}{\text{d}T}\right)_V
  \label{eqn:gl2}
\end{equation}
Die dazugehörige Dimension ist:
\begin{equation*}
  dim (C_V) = \frac{\text{Energie}}{\text{Temperatur} \cdot \text{Stoffmenge}}
\end{equation*}
Über die spezifische Wärmekapazität eines Festkörpers $c_\text{k}$ lässt sich die Molwärme bestimmen.
Die experimentelle Bestimmung der spezifischen Wärmekapazität  bei konstantem Volumen $c_{\text{k},V}$ lässt sich jedoch nur schwer umsetzten, denn die Ausdehnung des Stoffes lässt sich nur mit sehr großen Drücken verhindern.
Deshalb wird die spezifsche Wärmekapazität bei konstantem Druck $c_{\text{k},p}$ gemessen und zur Umrechnung der Zusammenhang zwischen $C_V$ und $C_p$ verwendet.
\begin{equation}
  C_V = C_p - 9 \alpha^2 \kappa V_0 T
  \label{eqn:gl3}
\end{equation}
Der lineare Ausdehnungskoeffizient $\alpha$ und das Kompressionsmodul $\kappa$ sind materialabhängige Größen und das Molvolumen $V_0$ undd $C_p$ können umgeschrieben werden zu \cite{Molwärme}:
\begin{align*}
  V_0 = \frac{M}{\rho} \\
  C_p = c_{\text{k},p} \cdot M \\
  \intertext{mit der molaren Masse $M$.}
\end{align*}
Die Gleichung \eqref{eqn:gl3} kann dann folgendermaßen ausgedrückt werden:
\begin{equation}
  C_V = c_{\text{k},p} M - 9 \alpha^2 \kappa \frac{M}{\rho} T
  \label{eqn:gl4}
\end{equation}
\subsection{klassische Betrachtung und das Dulong-Petitsche Gesetz}
\label{sec:klassisch}

Das Dulong-Petitsche Gesetz besagt, dass die Molwärme bei konstantem Volumen unabhängig von den chemischen Eigenschaften des Festkörpers ist.
Die entsprechende Erklärung liefert die klassiche Thermodynamik.
Die Atome im Gitter eines Festkörpers besitzen keine Translations- und Rotationsfreiheitsgrade, sondern nur Schwingungsfreiheitsgrade.
Betrachtet man die Schwingung der Atome als harmonische Oszillation, ergibt sich für die mittlere innere Energie eines Atoms $\langle u \rangle$:
\begin{equation}
  \langle u \rangle = 2 \langle E_\text{kin} \rangle
  \label{eqn:gl5}
\end{equation}
Nach dem Äquipartitionstheorem ist die kinetische Energie eines Atoms pro Freiheitsgrad
\begin{equation}
  \langle E_\text{kin} \rangle = \frac{1}{2} k T
  \label{eqn:gl6}
\end{equation}
mit der Boltzmann-Konstante $k$.
Durch Multiplizieren mit der Loschmidtschen Zahl $N_\text{L}$ kann nun die mittlere innere Engergie des Körpers pro \si{\mol} $\langle U \rangle$ bestimmt werden.
\begin{equation}
  \langle U \rangle = 3 N_\text{L} \langle u \rangle = 3 N_\text{L} k T = R T
  \label{eqn:gl7}
\end{equation}
Aus Gleichung \eqref{eqn:gl2} und \eqref{eqn:gl7} lässt sich nun die Aussage des Dulong-Petitsche Gesetz ableiten, dass für die Molwärme $C_V$ gilt
\begin{equation}
  C_V = 3 R
  \label{eqn:gl8}
\end{equation}
mit der allgemeinen Gaskonstante $R$.

\subsection{quantenmechanische Betrachtung}
\label{sec:quantenmechanisch}

Wenn die Temperatur des Körpers niedrig genug ist, kann die Molwärme beliebig klein werden, was dem Dulong-Petitschen Gesetz widerspricht.
Der Grund dafür liegt in der Annahme der klassichen Physik, dass die Oszillation der Atome jeden Betrag annehmen kann.
Diese Annahme steht jedoch im Widerspruch mit der Quantenmechanik, denn diese besagt, dass ein Teilchen, das mit der Freqeunz $\omega$ schwingt, nur bestimmte Energieniveaus annehmen kann.
\begin{equation}
  \increment u = n \hbar \omega
  \label{eqn:gl9}
\end{equation}
Daraus folgt der wesentlich kompliziertere Zusammenhang:
\begin{equation}
  \langle U_\text{qu} \rangle = \frac{3 N_\text{L} \hbar \omega}{\exp(\hbar \omega / k T) - 1}
  \label{eqn:gl10}
\end{equation}
Das Dulong-Petitsche Gesetz ist ein Spezialfall dieser Gleichung \eqref{eqn:gl10}, für große $T$ strebt $\langle U_\text{qu} \rangle$ gegen $3 R$.
Außerdem ist $\omega \propto 1/\sqrt{m}$, deshalb wird für Elemente mit geringem Atomgewicht die klassiche Näherung erst für deutlich höhere Temperaturen erreicht.

\subsection{Messung der spezifische Wärmekapazität mit Kalorimeter}
\label{sec:wärmeaustausch}

Durch den Wärmeaustausch eines erhitzen Metalls mit Wasser in einem Kalorimeter lässt sich mit folgender Gleichung die spezifsche Wärmekapazität des Metalls bestimmen.
\begin{equation}
  c_\text{k} = \frac{(c_\text{w} m_\text{w} + c_\text{g} m_\text{g}) (T_\text{m} - T_\text{w})}{m_\text{k} (T_\text{k} - T_\text{m})}
  \label{eqn:gl11}
\end{equation}
$c_\text{w}$, $m_\text{w}$ und $T_\text{w}$ sind die spezifische Wärmekapazität, die Masse und die Temperatur des Wassers.
$T_\text{k}$ ist die Temperatur des Metalls und $T_\text{m}$ die Mischtemperatur, also die Temperatur nachdem der Wärmeaustausch beendet ist.
Die spezifische Wärmekapazität $c_\text{g}$ und die Masse $m_\text{g}$ des Kalorimeters lassen sich mit einem Wärmeaustausch zwischen Wasser unterschiedlicher Temperaturen $T_\text{y}$ und $T_\text{x}$ bestimmen.
\begin{equation}
  c_\text{g} m_\text{g} = \frac{c_\text{w} m_\text{y} (T_\text{y} - T_\text{m}) - c_\text{w} m_\text{x} (T_\text{m} - T_\text{x})}{(T_\text{m} - T_\text{x})}
  \label{eqn:gl12}
\end{equation}
