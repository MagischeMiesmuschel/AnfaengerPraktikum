\input{header.tex}

\subject{V201}
\title{Das Dulong-Petitsche Gesetz}
\date{%
  Durchführung: 09.01.18
  \hspace{3em}
  Abgabe: 16.01.18
}

\begin{document}

\maketitle
\thispagestyle{empty}
\tableofcontents
\newpage

\section{Ziel}
\label{sec:Ziel}

Dieser Versuch soll die Frage klären,
ob die Oszillation der Atome bzw. der Moleküle eines Festkörpers mit der klassischen Physik beschrieben werden kann oder
ob eine quantenmechansische Betrachtung notwendig ist.
Eine Untersuchung der Molwärme von Festkörpern kann diese Frage beantworten.

\section{Theorie}
\label{sec:Theorie}

\subsection{Kernreaktionen mit Neutronen}
\label{sec:kr}

Da Neutronen nicht die Coulomb-Barriere eines Kerns überwinden müssen, sind sie besonders gut dafür geeignet, instablie Nuklide zu erzeugen.
Im folgenden Kapitel wird deshalb, die Wechselwirkung zwischen Neutronen und Kernen näher erläutert.
Durch die Absorption eines Neutrons entsteht ein neuer Kern $\text{A}^*$, der Zwischenkern,
dessen Energie um die kinetische Energie und die Bindungsenergie des Neutrons zunimmt.
Aufgrund der starken Wechselwirkung verteilt sich die Energie auf alle Nukleonen und der Neutroneneinfang sorgt nicht für das Abstoßen eines Nukleons aus dem Kern.
Stattdessen emittiert der angeregte Zwischenkern ein $\gamma$-Quant und geht in seinen Grundzustand über.
\begin{equation}
  \ce{^{m}_zA + ^{1}_0n -> ^{m+1}_zA^* -> ^{m+1}_zA + \gamma }
\end{equation}
Der neue Kern $\ce{^{m+1}_zA}$ ist meistens instabil, da er im Vergleich zu einem stabilen Kern gleicher Ordnungszahl mehr Neutronen enthält.
Durch einen $\beta$-Zerfall, bei dem infolge des Massendefekts Energie frei wird,
wandelt er sich in einen stabilen Kern um. \cite{V702}
\begin{equation}
  \ce{^{m+1}_zA -> ^{m+1}_{z+1}C + \beta^- + E_{kin} + \bar{\nu}_e}
\end{equation}

\subsection{Wirkungsquerschnitt}

Der Wirkungsquerschnitt $\sigma$ ist ein Maß für die Wahrscheinlichkeit, dass ein Neutron eingefangen wird.
Er beschreibt die fiktive Fläche, die ein Kern bsitzen müsste, um jedes eintreffende Neutron einzufangen.
Folgende Gleichung lässt sich so aufstellen,
wenn $n$ Neutronen pro Sekunde  $\SI{1}{\square\meter}$ einer dünnen Folie (Dicke $d$, $K$ Atome/$\si{\cubic\meter}$) treffen und dabei $u$ Einfänge auftreten.
\begin{equation}
  \sigma = \frac{u}{nKd}
\end{equation}
Im Falle der Neutronenabsorption ist der Wirkungsquerschnitt stark von der Geschwindigkeit $v$ bzw. der kinetischen Energie der Neutronen $E$ abhängig.
Für langsame Neutronen betsteht folgender proportionaler Zusammenhang zwischen Neutronenenergie $E$ und Wirkungsquerschnitt $\sigma$:
\begin{equation}
  \sigma \propto 1/\sqrt{E} \propto 1/v
\end{equation}
Anschaulich kann die größere Absorptionswahrscheinlichkeit der langsamen Neutronen mit ihrer längeren Aufenthaltszeit im Einwirkungsbereich des Kerns erkärt werden. \cite{V702}

\subsection{Erzeugung niederenergetischer Neutronen}

Aufgrund seiner Instabilität kommt das Neutron nicht natürlich vor, sondern muss künstlich freigesetzt werden.
In diesem Versuch wird das zum Beispiel durch Beschuss von $\ce{^{9}_4Be}$-Kernen mit $\alpha$-Strahlung realisiert, die aus dem Zerfall von $\ce{^{226}_88Ra}$ entsteht.
\begin{equation}
  \ce{^{9}_4Be + ^{4}_2\text{\alpha} -> ^{12}_6C + ^{1}_0n}
\end{equation}
Zur Abbremsung werden die Neutronen, die ein kontinuierliches Energiespektrum besitzen, durch eine Material mit leichten Kernen geschickt (in diesem Experiment Paraffin).
Zwischen den Neutronen und den Kernen finden elasitsche Stöße statt, bei denen die Neutronen durch Energieübertragung an die Kerne langsamer werden.
Das funktioniert umso besser, je leichter die Kerne sind, weshalb Moleküle benutzt werden, die Wasserstoff enthalten, wie das Paraffin (gesättigte Kohlenwasserstoffe).
Die Neutronen erreichen so annähernd die mittlere kinetische Energie der Moleküle von $\SI{0.025}{\eV}$, was einer Geschwindigkeit von $\SI{2.2}{\kilo\meter\per\second}$ entspricht.
Diese Neutronen werden thermische Neutronen genannt. \cite{V702}

\subsection{Zerfall instabiler Isotope}

Indium und Silber können durch die in Kapitel \ref{sec:kr} erwähnten Prozesse in instabile Isotope umgewandelt werden, die in einem $\beta$-Zerfall wieder in stabile Nuklide übergehen.
Für einen radioaktiven Zerfall gilt allgemein für die Anzahl der noch nicht zerfallenen Kerne folgenede Gesetzmäßigkeit.
\begin{equation}
  N(t) = N_0 \text{e}^{-\lambda t}
  \label{eqn:Zerfall}
\end{equation}
$N_0$ ist dabei die anfängliche Anzahl der Atome und $\lambda$ die Zerfallskonstante.
In der Praxis werden die Zerfälle $N_{\increment t}(t) = N(t) - N(t + \increment t)$ in einem Zeitintervall $\increment t$ gemessen, um die Zerfallskonstante $\lambda$ zu bestimmen.
Gleichung \eqref{eqn:Zerfall} nimmt so dann folgende Form an.
\begin{align}
  N_{\increment t}(t) &= N_0 \text{e}^{-\lambda t} - N_0 \text{e}^{-\lambda (t + \increment t)} = N_0 (1 -\text{e}^{-\lambda \increment t}) \text{e}^{-\lambda t} \\
  \ln(N_{\increment t}(t)) &= \ln(N_0 (1 -\text{e}^{-\lambda \increment t})) - \lambda t
\end{align}
Dabei muss auf eine geeignete Wahl von $\increment t$ geachtet werden.
Denn ist $\increment t$ zu klein wird der statistische Fehler zu groß und ist $\increment t$  zu groß ist das radioaktive Präparat größtenteils schon zerfallen.
Für Indium sieht der Zerfall folgendermaßen aus.
\begin{equation}
  \ce{^{115}_49In + ^{1}_0n -> ^{116}_49In -> ^{116}_50Sn + \beta^- + \bar{\nu}_e}
\end{equation}
Da natürliches Silber aus den Isotopen $\ce{^{107}_47Ag}$ und $\ce{^{109}_47Ag}$ besteht, laufen zwei Zerfälle gleichzeitig ab, wenn das Silber durch Neutronenbeschuss aktiviert wird.
\begin{align}
  \ce{^{107}_47Ag + ^{1}_0n -> ^{108}_47Ag -> ^{108}_48Cd + \beta^- + \bar{\nu}_e} \\
  \ce{^{109}_47Ag + ^{1}_0n -> ^{110}_47Ag -> ^{110}_48Cd + \beta^- + \bar{\nu}_e}
\end{align}
Eine Bestimmung der einzelnen Halbwertszeiten ist trotzdem möglich, denn nach ausreichender Zeit ist das kurzlebige Isotop $\ce{^{110}_47Ag}$ praktisch komplett zefallen.
Die gemessenen Zerfälle stammen nur noch vom langlebigen Isotop $\ce{^{108}_47Ag}$. \cite{V702}

\section{Durchführung}
\label{sec:Durchführung}

Wie in Kapitel \ref{sec:molwärme} erwähnt wird die Molwärme $C_V$ indirekt über die Messung der spezifische Wärmekapazität $c_\text{k}$ bestimmt.
Diese Messung kann mit einem Kalorimeter wie in Abbildung \ref{fig:abb1} durchgeführt werden.
\begin{figure}
\centering
\includegraphics[height=6.0cm]{data/abb1.png}
\caption{Schematischer Aufbau eines Kalorimeters. \cite{V201}}
\label{fig:abb1}
\end{figure}

\subsection{Bestimmung der Wärmekapazität des Kalorimeters}
\label{sec:Durchführung_Kalo}

Die Wärmekapazität des Kalorimeters $c_\text{g} m_\text{g}$ muss bekannt sein, um die spezifische Wärmekapazität eines Materials bestimmen zu können.
Deshalb wird eine zusätliche Messung nur mit Wasser durchgeführt.
Die Massen $m_\text{x}$ und $m_\text{y}$ (zwei ungefähr gleichgroßer Wassermengen) werden bestimmt.
Es sollte ein ähnlich hoher Wasserstand des Kalorimeters erreicht werden wie bei den folgenden Messungen.
Die eine Wassermenge wird in das Kalorimeter gegeben und nach einer kurzen Zeit des Wärmeausgleichs die Temperatur $T_\text{x}$ gemessen.
Die andere Wassermenge wird auf etwa $\SI{90}{\celsius}$ erhitzt und die Temperatur $T_\text{y}$ gemessen.
Anschließend wird das warme Wasser in das Kalorimeter gegeben und nachdem sich ein Gleichgewicht einstellt, die Mischtemperatur $T_\text{m}$ gemessen.

\subsection{Bestimmung der spezifischen Wärmekapazität verschiedener Stoffe}
\label{sec:Durchführung_Stoffe}

Die Messung läuft ähnlich ab wie im vorherigen Abschnitt \ref{sec:Durchführung_Kalo}.
Das Kalorimeter wird mit Wasser befüllt, dessen Masse vorher $m_\text{w}$ bestimmt wird.
Nach kurzer Zeit des Temperaturausgleichs wird die Temperatur des Wassers $T_\text{w}$ gemessen.
Die Materialprobe, deren Wärmekapazität $c_\text{k}$ bestimmt werden soll, wird gewogen und auf etwa \SI{90}{\celsius} erhitzt.
Bevor die erhitzte Probe in das Kalorimeter getaucht wird, wird die Temperatur $T_\text{k}$ gemessen.
Nachdem der Wärmeaustausch zwischen Probe und Wasser abgeschlossen ist, wird die Mischtemperatur $T_\text{m}$ gemessen.

\section{Auswertung}
\label{sec:Auswertung}

\begin{figure}
  \centering
  \includegraphics{plot1.pdf}
  \caption{Gravitation.}
  \label{fig:plot}
\end{figure}

\begin{figure}
  \centering
  \includegraphics{plot2.pdf}
  \caption{Schwingung.}
  \label{fig:plot}
\end{figure}

\begin{figure}
  \centering
  \includegraphics{plot3.pdf}
  \caption{Präzession.}
  \label{fig:plot}
\end{figure}

\section{Diskussion}
\label{sec:Diskussion}

Die berechneten mittleren Reichweite weichen um \SI{57.35}{\percent} voneinander ab.
Für den Energieverlust liegt die Abweichung bei \SI{11.69}{\percent}.
Diese große Abweichung kann einerseits mit der statistischen Unisicherheit der Zählrate erklärt werden.
Wie an den Messungen aus Kapitel \ref{sec:Zufall} zu erkennen ist, ist der radioaktive Zerfall ein zufälliges Ereignis,
das einer Poissonverteilung folgt.
Außerdem spielt die Wahl von $N_\text{max}$ eine entscheidende Rolle für das Ergebnis.
Durch die geringe Anzahl an Messwerten hat dieser statistische Effekt einen großen Einfluss auf das Ergebnis.
Weitere Fehlerquellen sind die begrenzte Ablesegenauigkeit des Abstandes des $\alpha$-Strahlers und
die ebenfalls begrenzte Einstellgenauigkeit des Druckes.


\printbibliography{}

\end{document}
