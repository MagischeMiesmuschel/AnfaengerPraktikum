\section{Auswertung}
\label{sec:Auswertung}

\subsection{Impuls-Echo-Verfahren}

In Tabelle \ref{tab:echo} sind für die verschieden langen Zylinder mit dem A-Scan gemessenen Laufzeiten und Amplituden des ersten und zweiten Pulses aufgelistet.
\begin{table}
  \centering
  \caption{Laufzeiten und Amplituden für das Impuls-Echo-Verfahren.}
  \label{tab:echo}
  \begin{tabular}{c | c c | c c | c}
    \toprule
    & \multicolumn{2}{c}{Puls 1} & \multicolumn{2}{c}{Puls 2} & \\
    Länge [\si{\centi\meter}] & $U$ [\si{\volt}] & $t$ [\si{\micro\second}] & $U$ [\si{\volt}] & $t$ [\si{\micro\second}] & $\increment t$ [\si{\micro\second}] \\
    \midrule
    11,9 & 1,486 & 0,4 & 0,922 & 89,2 & 88,8 \\
     7,9 & 1,486 & 0,4 & 1,385 & 60,2 & 59,8 \\
     3,7 & 1,486 & 0,4 & 1,457 & 29,7 & 29,3 \\
      10 & 1,486 & 0,4 & 1,173 & 76,6 & 76,6 \\
     5,9 & 1,486 & 0,4 & 1,415 &   46 & 45,4 \\
       8 (kombiniert) & 1,486 & 0,4 & 1,366 &   60 & 59,6 \\
     3,1 & 1,486 & 0,4 & 1,456 & 23,6 & 23,2 \\
    \bottomrule
  \end{tabular}
\end{table}

\subsubsection{Schallgeschwindigkeit}
\label{sec:geschw}

Die Schallgeschwindigkeit kann nicht direkt mit Gleichung \eqref{eqn:gl4} besimmt werden, da die Laufzeiten aufgrund Anpassungsschicht einen systematischen Fehler besitzen.
Deshalb wird eine lineare Ausgleichsrechnung durchgeführt (siehe Abb. \ref{fig:plot1}).
\begin{equation}
  y(x) = A \cdot x + B
\end{equation}
\begin{figure}
  \includegraphics{build/plot1.pdf}
  \caption{linearer Zusammenhang zwischen Laufzeit und der Länge der Zylinder für das Impuls-Echo-Verfahren.}
  \label{fig:plot1}
\end{figure}
Die Steigung dieser Geraden enspricht der Schallgeschwindigkeit, deren y-Achsenabschnitt der doppelten Dicke der Anpassungsschicht
und deren Nullstelle der doppelten Laufzeit durch die Anpassungsschicht.
\begin{align*}
  A_\text{Echo} &= c_\text{Echo} = \SI{2697(40)}{\meter\per\second} \\
  B_\text{Echo} &= \SI{0.42(24)}{\centi\meter} \\
  d_\text{Echo} &= \frac{B_\text{Echo}}{2} = \SI{0.21}{\centi\meter} \\
  t_\text{Anpassung,Echo} &= - \frac{B_\text{Echo}}{2 \cdot A_\text{Echo}} =\SI{0.77}{\micro\second}
\end{align*}
\FloatBarrier

\subsubsection{Dämpfung}

Um die Dämpfung zu berechnen, wird Gleichung \eqref{eqn:gl3} umgestellt werden.
\begin{equation}
 \alpha*x = -\ln(I(x)/I_0))
\end{equation}
Nun kann mit den Werten aus Tabelle \ref{tab:echo} über eine lineare Ausgleichsrechnung (siehe Abb. \ref{fig:plot3}) der Absorptionskoeffizient berechnet werden.
\begin{figure}
  \includegraphics{build/plot3.pdf}
  \caption{halblogarithmische Darstellung des Intensitätsabfalls.}
  \label{fig:plot3}
\end{figure}
Für den Absorptionskoeffizienten berechnet sich folgender Wert.
\begin{align*}
  A_\text{Int} &= \alpha = \SI{4.51(116)}{\per\meter} \\
  B_\text{Int} &= \num{18.91(908)} \\
\end{align*}
\FloatBarrier

\subsection{Durchschallungsverfahren}

Mit dem Durchschallungsverfahren soll ebenfalls die Schallgeschwindigkeit bestimmt werden, folgende Laufzeiten werden dabei gemessen (siehe Tabelle \ref{tab:Durch}).
\begin{table}
  \centering
  \caption{Laufzeiten für das Durchschallungsverfahren.}
  \label{tab:Durch}
  \begin{tabular}{c c}
    \toprule
    Länge [\si{\centi\meter}] & $t$ [\si{\micro\second}] \\
    \midrule
    11,9  & 45,9 \\
     7,9  & 31,5 \\
     3,7  & 15,9 \\
      10  & 40,5 \\
     5,9  & 24,2 \\
       8 (kombiniert) & 31,3 \\
     3,1  & 12,6 \\
    \bottomrule
  \end{tabular}
\end{table}
Analog zum Impuls-Echo-Verfahren wird eine Ausgleichsrechnung mit Gleichung \eqref{eqn:gl4} durchgeführt (siehe Abb. \ref{fig:plot2}).
\begin{figure}
  \includegraphics{build/plot2.pdf}
  \caption{linearer Zusammenhang zwischen Laufzeit und der Länge der Zylinder für das Durchschallungsverfahren.}
  \label{fig:plot2}
\end{figure}
Allerdings fällt der Faktor $1/2$ weg, da der Schall den Zylinder nur einmal durchläuft.
Die Steigung dieser Geraden enspricht der Schallgeschwindigkeit und deren Nullstelle der Laufzeit durch die Anpassungsschicht.
\begin{align*}
  A_\text{Durchschallung} &= c_\text{Durchschallung} = \SI{2627(64)}{\meter\per\second} \\
  B_\text{Durchschallung} &= \SI{0.36(20)}{\centi\meter} \\
  d_\text{Durchschallung} &= \frac{B_\text{Durchschallung}}{2} = \SI{0.18}{\centi\meter} \\
  t_\text{Anpassung,Durchschallung} &= - \frac{B_\text{Durchschallung}}{2 \cdot A_\text{Durchschallung}} =\SI{0.69}{\micro\second}
\end{align*}
\FloatBarrier

\subsection{Spekrale Analyse und Cepstrum}
\label{sec:cep}

Aus dem Cepstrum der Ultraschallsonde (siehe Abb. \ref{fig:cep}) lassen sich drei Reflexionen erkennen, aus deren Laufzeiten sich mit Gleichung \eqref{eqn:gl4} die Dicke der Acrylscheiben bestimmen lässt.
Zur Berechnung wird die Schallgeschwindigkeit aus \ref{sec:geschw} benutzt.
Die Ergebnisse sind in Tabelle \ref{tab:cep} aufgeführt.
\begin{table}
  \centering
  \caption{Aus dem Cepstrum berechnete Dicke der Scheiben.}
  \label{tab:cep}
  \begin{tabular}{c c c}
    \toprule
    & $t$ [\si{\micro\second}] & Dicke der Scheiben $d$ [\si{\centi\meter}] \\
    \midrule
    Scheibe oben & 4,23 & 0,57\\
    Scheibe unten & 7,16 & 0,97\\
    Scheiben zusammen & 11,41 & 1,54 \\
    \bottomrule
  \end{tabular}
\end{table}
Die Dicke der Scheiben kann auch aus den Laufzeiten der Mehrfachreflexionen bestimmt werden.
In Tabelle \ref{tab:mehr} sind die Ergebnisse dargestellt.
\begin{table}
  \centering
  \caption{Aus den Laufzeiten der Mehrfachreflexionen berechnete Dicke der Scheiben.}
  \label{tab:mehr}
  \begin{tabular}{c c c}
    \toprule
    & $t$ [\si{\micro\second}] & Dicke der Scheiben $d$ [\si{\centi\meter}] \\
    \midrule
    Scheibe oben & $|30,6-35,1|$ & 0,61\\
    Scheibe unten & $|35,1-41,9|$ & 0,92\\
    Scheiben zusammen & $|30,6-41,9|$ & 1,52 \\
    \bottomrule
  \end{tabular}
\end{table}
\begin{figure}
  \centering
  \includegraphics[height=8cm]{data/FFT.png}
  \caption{Spektrum der Ultraschallsonde.}
  \label{fig:FFT}
\end{figure}
\begin{figure}
  \centering
  \includegraphics[height=8cm]{data/Cepstrum.png}
  \caption{Cepstrum der Ultraschallsonde.}
  \label{fig:cep}
\end{figure}

\FloatBarrier

\subsection{Abmessungen des Auges}

Im richtigen Einschallwinkel lassen sich 4 Reflexionen im Auge messen (siehe Tabelle \ref{tab:Auge1}), die an den Grenzflächen der verschiedenen Augenbestandteile auftreten.
\begin{table}
  \centering
  \caption{Laufzeiten der Echos im Auge.}
  \label{tab:Auge1}
  \begin{tabular}{c c c c}
    \toprule
    $t_1$ (Iris) & $t_2$ (Linse Eingang) & $t_3$ (Linse Ausgang) & $t_4$ (Retina) \\
    \midrule
    \SI{11.7}{\micro\second} & \SI{16.2}{\micro\second} & \SI{23.6}{\micro\second} & \SI{73}{\micro\second} \\
    \bottomrule
  \end{tabular}
\end{table}
Die Schallgeschwindigkeit der Glaskörperflüssigkeit beträgt $c_\text{GK} = \SI{1410}{\meter\per\second}$ und in der Linse $c_\text{L} = \SI{2500}{\meter\per\second}$.
Bevor die Abstände mit Gleichung \eqref{eqn:gl4} berechnet werden können, muss die Laufzeit bis zur Iris zweimal um die Laufzeit der Anpassungsschicht korrigiert werden.
\begin{table}
  \centering
  \caption{Laufzeiten der Echos im Auge.}
  \label{tab:Auge2}
  \begin{tabular}{c c c c c}
    \toprule
    von & zu & Laufzeit $\increment t$ & Abmessung Augenmodell & echtes Auge (1:3) \\
    \midrule
    Hornhaut & Iris & \SI{10.16}{\micro\second} & \SI{7.16}{\milli\meter} & \SI{2.39}{\milli\meter}  \\
    Iris & Linse Eingang & \SI{4.5}{\micro\second} & \SI{3.17}{\milli\meter} & \SI{1.06}{\milli\meter} \\
    Linse Eingang & Linse Ausgang & \SI{7.4}{\micro\second} & \SI{9.25}{\milli\meter} & \SI{3.08}{\milli\meter} \\
    Linse Ausgang & Retina & \SI{49.4}{\micro\second} & \SI{3.48}{\centi\meter} & \SI{1.16}{\centi\meter} \\
    Hornhaut & Retina & & \SI{5.44}{\centi\meter} & \SI{1.81}{\centi\meter} \\
    \bottomrule
  \end{tabular}
\end{table}
