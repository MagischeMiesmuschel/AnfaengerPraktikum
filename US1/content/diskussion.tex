\section{Diskussion}
\label{sec:Diskussion}

Der Literaturwert von Acryl beträgt \SI{2730}{\meter\per\second}. \cite{Acryl}
Die Schallgeschwindigkeit mit dem Impuls-Echo-Verfahren $c_\text{Echo}$ weicht damit um \SI{1.21}{\percent} und die des Durchschallungsverfahren $c_\text{Durchschallung}$ um \SI{3.77}{\percent} ab.
Für die Scheiben aus \ref{sec:cep} wird eine Dicke von \SI{1}{\centi\meter} (untere) und \SI{0.6}{\centi\meter} (obere) gemessen.
Die über das Cepstrum berechneten Dicken weichen für die obere Scheibe um \SI{5}{\percent} und für die untere um \SI{3}{\percent} ab.
Die über die Laufzeiten der Mehrfachreflexionen berechneten Dicken weichen für die obere Scheibe um \SI{1.67}{\percent} und für die untere um \SI{8}{\percent} ab.
Für das Abmessungen des Auge lassen sich Referenzwerte finden, die ungefähr den berechneten Abständen entsprechen,
Dicke der Linse \SI{3.5}{\milli\meter}, Durchmesser des Augapfels 22-23 mm. \cite{Auge}
Mögliche Gründe für Abweichungen sind zum einen die begrenzte Ablesemöglichkeit von der Schieblehre und den Grafiken des Auswertungsprogrammes.
Die Längen der Körper und die Laufzeiten können also nur mit einer Unsicherheit bestimmt werden.
Außerdem können die Ultraschallsonden nur mit einer begrenzten Genauigkeit die Amplituden und Laufzeiten messen.
