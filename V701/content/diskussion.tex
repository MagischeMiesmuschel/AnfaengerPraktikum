\section{Diskussion}
\label{sec:Diskussion}

Die berechneten mittleren Reichweite weichen um \SI{57.35}{\percent} voneinander ab.
Für den Energieverlust liegt die Abweichung bei \SI{11.69}{\percent}.
Diese große Abweichung kann einerseits mit der statistischen Unisicherheit der Zählrate erklärt werden.
Wie an den Messungen aus Kapitel \ref{sec:Zufall} zu erkennen ist, ist der radioaktive Zerfall ein zufälliges Ereignis,
das einer Poissonverteilung folgt.
Außerdem spielt die Wahl von $N_\text{max}$ eine entscheidende Rolle für das Ergebnis.
Durch die geringe Anzahl an Messwerten hat dieser statistische Effekt einen großen Einfluss auf das Ergebnis.
Weitere Fehlerquellen sind die begrenzte Ablesegenauigkeit des Abstandes des $\alpha$-Strahlers und
die ebenfalls begrenzte Einstellgenauigkeit des Druckes.
