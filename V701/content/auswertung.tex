\section{Auswertung}
\label{sec:Auswertung}

\section{Auswertung}
\label{sec:Auswertung}

\subsection{Bestimmung der Reichweite von a-Strahlung}

\subsubsection{Messung im Abstand von 16 mm}

In Tabelle \ref{tab:M1} sind der gemessene Druck und die entsprechenden Zählraten und der Channel der maximalen Energie dargestellt.
\begin{table}
  \centering
  \caption{Messwerte für einen Abstand von 16 mm.}
  \label{tab:M1}
  \begin{tabular}{c c c}
    \toprule
    $N$ & Channel & $p$ [mbar] \\
    \midrule
    135236 &  483 &  0 \\
    127618 &  463 &  50 \\
    126181 &  440 &  100 \\
    125278 &  431 &  150 \\
    125342 &  423 &  200 \\
    124656 &  423 &  250 \\
    123702 &  415 &  300 \\
    123934 &  415 &  350 \\
    121258 &  399 &  400 \\
    121358 &  399 &  450 \\
    119117 &  387 &  500 \\
    117457 &  379 &  550 \\
    115477 &  367 &  600 \\
    112616 &  363 &  650 \\
    110857 &  352 &  700 \\
    108448 &  348 &  750 \\
    103277 &  335 &  800 \\
    96026  & 323  & 850 \\
    89745  & 312  & 900 \\
    \bottomrule
  \end{tabular}
\end{table}
Die nach Gleichung \eqref{eqn:gl4} berechnete effektive Länge wird gegen die Zählrate aufgetragen.
Mit einer linearen Ausgleichsrechnung (siehe Abb. \ref{fig:plot1}) kann aus diesen Werten die mittlere Reichweite bestimmt werden.
\begin{figure}
  \centering
  \includegraphics{build/plot1.pdf}
  \caption{Zählrate in Abhängigkeit der effektiven Länge.}
  \label{fig:plot1}
\end{figure}
Wird der Ansatz
\begin{equation*}
  f(x) = A*x + B
\end{equation*}
gewählt ergeben sich folgende Parameter
\begin{align*}
  A &= \SI{-2411594(123456)}{\per\meter} \\
  B &= \num{137674(1102)}
\end{align*}
Für die mittlere Reichweite gilt:
\begin{equation}
  R_m = \frac{N_\text{max}/2 - B}{A}
  \label{eqn:r1}
\end{equation}
Mit den zuvor berechneten Parametern und $N_\text{max} = 135236$ foglt also:
\begin{equation*}
  R_m = \SI{0.0290(16)}{\meter}
\end{equation*}
Der dazugehörige Fehler wird über die Gaußsche Fehlerfortpflanzung berechnet.
\begin{equation}
  \increment R_m = \sqrt{(-\frac{1}{A} \cdot \increment B)^2+(-\frac{N_\text{max}/2 - B}{A^2} \cdot \increment A)^2}
  \label{eqn:r2}
\end{equation}
Die mittlere Reichweite entspricht nach Gleichung \eqref{eqn:gl3} einer Energie von:
\begin{equation*}
  E_\alpha = \SI{44.40}{\kilo\eV}
\end{equation*}

Zur Bestimmung des Energieverlustes wird die effektive Länge gegen die Energie, die aus den Channeln abgelesen werden kann, aufgetragen.
Aus diesen Wertepaaren wird mit einer linearen Ausgleichsrechnung (siehe Abb. \ref{fig:plot2}) der Energieverlust besimmt.
\begin{figure}
  \centering
  \includegraphics{build/plot3.pdf}
  \caption{Energie in Abhängigkeit der effektiven Länge.}
  \label{fig:plot2}
\end{figure}
\begin{align*}
  A &= \frac{\text{d}E_\alpha}{\text{d}x} = \SI{-86.74(305)}{\mega\eV\per\meter} \\
  B &= \SI{3.87(3)}{\mega\eV}
\end{align*}
\FloatBarrier

\subsubsection{Messung im Abstand von 22 mm}

Die vorherigen Messungen und Berechnungen werden für einen anderen Abstand des $\alpha$-Strahlers wiederholt.
In folgender Tabelle \ref{tab:M2} sind die Messwerte aufgelistet.
\begin{table}
  \centering
  \caption{Messwerte für einen Abstand von 22 mm.}
  \label{tab:M2}
  \begin{tabular}{c c c}
    \toprule
    $N$ & Channel & $p$ [mbar] \\
    \midrule
    91901 & 467 &   0 \\
    92910 & 456 &  50 \\
    92045 & 442 & 100 \\
    91330 & 435 & 150 \\
    91081 & 423 & 200 \\
    91055 & 415 & 250 \\
    90774 & 408 & 300 \\
    90951 & 398 & 350 \\
    90017 & 387 & 400 \\
    89691 & 382 & 450 \\
    89082 & 370 & 500 \\
    88990 & 360 & 550 \\
    87405 & 352 & 600 \\
    86777 & 339 & 650 \\
    84426 & 326 & 700 \\
    80633 & 310 & 750 \\
    \bottomrule
  \end{tabular}
\end{table}
Für die Bestimmung der mittleren Reichweite wird wie zuvor eine Ausgleichsrechnung durchgeführt (siehe Abb. \ref{fig:plot3}),
die folgende Parameter liefert:
\begin{align*}
  A &= \SI{-754542(97433)}{\per\meter} \\
  B &= \num{97020(1137)}
\end{align*}
\begin{figure}
  \centering
  \includegraphics{build/plot2.pdf}
  \caption{Zählrate in Abhängigkeit der effektiven Länge.}
  \label{fig:plot3}
\end{figure}
Mit Gleichung \label{eqn:r1} und \label{eqn:r1} und $N_\text{max} 92045$ wird die Reichweite und der Fehler wieder berechnet:
\begin{equation*}
  R_m = \SI{0.068(9)}{\meter}
\end{equation*}
Die mittlere Reichweite entspricht nach Gleichung \eqref{eqn:gl3} einer Energie von:
\begin{equation*}
  E_\alpha = \SI{78.36}{\kilo\eV}
\end{equation*}

Die Ausgleichsrechnung für den Energieverlust (siehe Abb. \ref{fig:plot4}) ergibt folgende Werte:
\begin{align*}
  A &= \frac{\text{d}E_\alpha}{\text{d}x} = \SI{-77.49(146)}{\mega\eV\per\meter} \\
  B &= \SI{3.99(1)}{\mega\eV}
\end{align*}
\begin{figure}
  \centering
  \includegraphics{build/plot4.pdf}
  \caption{Energie in Abhängigkeit der effektiven Länge.}
  \label{fig:plot4}
\end{figure}

\subsection{Statistik des radioaktiven Zerfalls}
\label{sec:Zufall}

Zur Untersuchung der Statistik des radioaktiven Zerfalls wird 100 Mal für 10 s die Zählrate gemessen.
Die Messergebnisse sind in Tabelle \ref{tab:M3} aufgeführt.
\begin{table}
  \centering
  \caption{gemessene Zählrate bei konstanten Bedingungen (0 mbar).}
  \label{tab:M3}
  \begin{tabular}{c c c c c c}
    \toprule
    \midrule
  85 & 00 &   86 &  68 & 110 &  90 \\
  84 & 83 &   73 &  74 &  89 &  73 \\
  96 & 75 &   85 &  87 &  65 &  69 \\
  80 & 93 &   98 &  90 &  68 &  86 \\
  98 & 84 &   75 &  77 &  73 &  83 \\
  85 & 75 &   77 &  71 &  90 &  94 \\
 106 & 82 &   91 & 103 &  91 &  73 \\
  79 & 70 &   64 &  87 &  78 & 102 \\
  74 & 93 &   84 &  86 &  81 & 101 \\
 107 & 68 &   96 &  79 &  79 & 121 \\
  74 & 86 &   83 &  88 &  95 &  84 \\
  76 & 68 &   83 & 110 &  80 &  95 \\
  60 & 84 &   85 &  82 & 108 &  96 \\
  86 & 71 &  108 & 100 &  75 &  99 \\
  97 & 68 &   71 &  83 & 101 &  62 \\
  84 & 69 &   89 &  98 &  70 &  78 \\
  81 & 83 &   87 & & & \\
    \bottomrule
  \end{tabular}
\end{table}
Im anschließenden Histogramm (siehe Abb. \ref{fig:plot5}) sind die Verteilung der gemessenen Zählraten im Vergleich zur Poissonverteilung und Gaußverteilung abgebildet.
\begin{figure}
  \centering
  \includegraphics{build/plot5.pdf}
  \caption{Histogramm der Messwerte, der Poisson- und Gaußverteilung.}
  \label{fig:plot5}
\end{figure}
Mit folenden Gleichungen berechnen sich Mittelwert und Standardabweichung:
\begin{align}
  \bar N &= \frac{1}{100} \sum_{i=1}^100 N_i = 84.66 \\
  \increment \bar N &= \sqrt{\frac{1}{100(100-1)} \sum_{i=1}^{100} (N_i- \bar N)^2} = 12.22
\end{align}
