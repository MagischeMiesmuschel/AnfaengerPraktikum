\section{Theorie}
\label{sec:Theorie}

Durch Messung der Reichweite von $\alpha$-Strahlung kann dessen Energie bestimmt werden.
Neben Ionisationsprozessen können die $\alpha$-Teilchen ihre Energie auch durch Anregung oder Dissoziation von Molekülen verlieren.
Der Energieverlust $-dE_{\alpha}/dx$ hängt dabei von der Energie der $\alpha$-Strahlung und der Dichte des durchlaufenen Materials ab, 
wobei bei kleinen Geschwindigkeiten die Wahrscheinlichkeit zur Wechselwirkung zunimmt.
Für hinreichend große Energien läßt sich der Energieverlust der $\alpha$-Teilchen durch die Bethe-Bloch-Gleichung \ref{eqn:gl1} beschreiben
\begin{equation}
    -\frac{dE_{\alpha}}{dx} = \frac{z^2 e^4}{4 \pi \epsilon_0 m_e} \frac{n Z}{v^2} \text{ln} \left(\frac{2 m_e v^2}{I}\right)
    \label{eqn:gl1}
\end{equation}
$z$ ist die Ladung und $v$ die Geschwindigkeit der $\alpha$-Strahlung.
$Z$ ist die Ordnungszahl, $n$ die Teilchendichte und $I$ die Ionisierungsenergie des Targetgases.
Die Gültigkeit der Gleichung \ref{eqn:gl1} endet bei sehr kleinen Energien.
Die Wegstrecke bis zur vollständigen Abbremsung ist die Reichweite der $\alpha$-Teilchen und lässt sich mit 
\begin{equation}
    R = \int_0^{E_{\alpha}} \frac{dE_{\alpha}}{dE_{\alpha}/dx}
    \label{eqn:gl2}
\end{equation}
berechnen.
Weil bei niedriger werdenen Energien die Bethe-Bloch-Gleichung \ref{eqn:gl1} nicht mehr gilt, durch vermehrte Ladungsaustauschprozesse,
werden zur Bestimmung der mittleren Reichweite empirisch gewonnene Kurven genutzt.
Für die mittlere Reichweite von $\alpha$-Strahlung in Luft mit Energien $E_{\alpha} \leq 2,5$MeV kann die Beziehung 
\begin{equation}
    R_m = 3,1 \cdot E_{\alpha}^{3/2}
    \label{eqn:gl3}
\end{equation}
verwendet werden.
Die Reichweite von $\alpha$-Teilchen in Gasen ist bei konstanter Temperatur und konstantem Volumen proportional zum Druck $p$.
Deswegen kann durch eine Absorptionsmessung, bei der man den Druck variiert die Reichweite bestimmt werden.
Für die effektive Länge $x$ gilt bei festem Abstand $x_0$ zwischen Detektor und $\alpha$-Strahler die Beziehung
\begin{equation}
    x = x_0 \frac{p}{p_0}
\end{equation}
Mit $p_0 = 1013$ mbar für den Normaldruck.